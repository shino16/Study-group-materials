\documentclass[xelatex,ja=standard,a4paper,14pt,everyparhook=compat]{bxjsarticle}
\usepackage{amsmath, amssymb, amsthm}
\usepackage{enumitem, mleftright, mathtools, hyperref, bm}
\usepackage{caption, wrapfig}

\setenumerate{label=(\alph*)}

\usepackage{concmath}
\usepackage[OT1]{fontenc}
\setsansfont{CMU Concrete}
\setmainfont{CMU Concrete}
\setCJKmainfont{Noto Sans CJK JP}
\setCJKsansfont{Noto Sans CJK JP Bold}

\newcommand{\bbN}{\mathbb{N}}
\newcommand{\bbZ}{\mathbb{Z}}
\newcommand{\bbC}{\mathbb{C}}
\newcommand{\fS}{\mathfrak{S}}
\newcommand{\inv}[1]{#1^{-1}}

\newcommand{\paren}[1]{\mleft(#1\mright)}
\newcommand{\norm}[1]{\mleft|#1\mright|}
\newcommand{\bbinom}[2]{\mleft(\binom{#1}{#2}\mright)}
\newcommand{\ifthen}[3]{\text{$#1$ ? $#2$ : $#3$}}
\newcommand{\bool}[1]{[#1]}
\newcommand{\dotcup}{\mathbin{\dot\cup}}
\newcommand{\qbinom}[2]{\pmb{\binom{#1}{#2}}}

\theoremstyle{definition}
\newtheorem{exercise}{演習}
\newtheorem{theorem}{定理}[subsection]
\newtheorem*{theorem*}{定理}
\newtheorem{example}[theorem]{例}
\newtheorem{proposition}[theorem]{命題}
\newtheorem{corollary}[theorem]{系}
\newtheorem{definition}[theorem]{定義}
\newtheorem*{claim}{主張}
\renewcommand{\proofname}{\textup{証明}}

\newenvironment{subproof}[1][\proofname]{%
    \begin{proof}[#1]%
    \renewcommand*{\qedsymbol}{$\blacksquare$}
}{%
    \end{proof}
}

\begin{document}

\setcounter{section}{12}
\section{Matroids}
\setcounter{subsection}{1}
\subsection{Other Matroid Axioms}

\subsubsection*{復習}

\begin{definition}
    \textbf{集合システム}$(E, \mathcal{F})$:非空な有限集合$E$と$\mathcal{F} \subseteq 2^E$の対.

    \textbf{独立性システム}$(E, \mathcal{F})$:次を満たす集合システム \begin{enumerate}
        \item[(M1)] $\emptyset \in \mathcal{F}$,
        \item[(M2)] $X \subseteq Y \in \mathcal{F}$ $\Rightarrow$ $X \in \mathcal{F}$.
    \end{enumerate}
    $\mathcal{F}$の元は\textbf{独立},$2^E \setminus \mathcal{F}$の元は\textbf{従属}.極小な従属集合は\textbf{サーキット},極大な独立集合は\textbf{基}.$X \subseteq E$の極大な独立部分集合は,$X$の基と呼ばれる.
\end{definition}

\begin{definition}
    独立性システム$(E, \mathcal{F})$と$X \subseteq E$について,

    \textbf{ランク}:$r(X) \coloneqq \max\{|Y| : Y \subseteq X, Y \in \mathcal{F}\}$ ($X$の基の位数の最大値).

    \textbf{閉包}:$\omega(X) \coloneqq \{y \in E : r(X \cup \{y\}) = r(X))\}$.
\end{definition}

\begin{definition}[定理13.5]
    \textbf{マトロイド}$(E, \mathcal{F})$:次(のいずれか)を満たす独立性システム \begin{enumerate}
        \item[(M3)] $X, Y \in \mathcal{F}$, $|X| > |Y|$ $\Longrightarrow$ $\exists x \in X \setminus Y$ s.t. $Y \cup \{x\} \in \mathcal{F}$,
        \item[(M3')] $X, Y \in \mathcal{F}$, $|X| = |Y| + 1$ $\Longrightarrow$ $\exists x \in X \setminus Y$ s.t. $Y \cup \{x\} \in \mathcal{F}$,
        \item[(M3'')] 任意の$X \subseteq E$について,$X$の基の位数はどれも等しい.
    \end{enumerate}
\end{definition}

\newpage

\subsubsection*{基の族に関する公理系}

\setcounter{theorem}{8}
\begin{theorem}
    $E$を有限集合とし,$\mathcal{B} \subseteq 2^E$とする.$\mathcal{B}$が何らかのマトロイド$(E, \mathcal{F})$の基の族であることは,以下が同時に成立することと同値. \begin{enumerate}
        \item[(B1)] $\mathcal{B} \neq \emptyset$,
        \item[(B2)] $\forall B_1, B_2 \in \mathcal{B}$, $\forall x \in B_1 \setminus B_2$, $\exists y \in B_2 \setminus B_1$ s.t. $(B_1 \setminus \{x\}) \cup \{y\} \in \mathcal{B}$.
    \end{enumerate}
\end{theorem}
\begin{proof}
    ($\Longrightarrow$)
    (B1):(M1)よりよい.

    (B2):(M2)より$B_1 \setminus \{x\} \in \mathcal{F}$.

    (M3)より$\exists y \in B_2 \setminus (B_1 \setminus \{x\})$ s.t. $(B_1 \setminus \{x\}) \cup \{y\} \in \mathcal{F}$.

    (M3'')よりこれは基.

    また$y \in B_2$より$x \neq y$,したがって$y \in B_2 \setminus B_1$.

    ($\Longleftarrow$)
    (B2)のイメージ:$B_1 \setminus B_2 \neq \emptyset$のとき,$B_1$を$(B_1 \setminus \{x\}) \cup \{y\}$に置き換えると$|B_1 \cap B_2|$が$1$増える.

    まず,$\mathcal{B}$の元の位数がどれも等しいことを示す.$|B_1| > |B_2|$なる$B_1$,$B_2$が存在すると仮定する.(B2)より$|B_1|$を保ったまま$|B_1 \cap B_2|$をいくらでも大きくでき,矛盾.

    ここで,
    \begin{equation*}
        \mathcal{F} \coloneqq \{F \subseteq E : \text{$\exists B \in \mathcal{B}$ s.t. $F \subseteq B$}\},
    \end{equation*}
    とする.$(E, \mathcal{F})$がマトロイドであることを示す.(M1),(M2)はよい.

    (M3)を示す.((M3'')で終わりでは??)

    $|X| > |Y|$なる$X, Y \in \mathcal{F}$を考える.$X \subseteq B_1 \in \mathcal{B}$,$Y \subseteq B_2 \in \mathcal{B}$を満たす$B_1$,$B_2$を,$|B_1 \cap B_2|$が最大になるようにとる.

    $B_2 \cap (X \setminus Y) \neq \emptyset$である場合,$x \in B_2 \cap (X \setminus Y)$とすると,$Y \cup \{x\} \subseteq B_2$より$Y \cup \{x\} \in \mathcal{F}$.

    $B_2 \cap (X \setminus Y) = \emptyset$と仮定して矛盾を示す.目標:$B_2 \setminus B_1 \neq \emptyset$ \begin{align*}
               & |B_1 \cap B_2| + |Y \setminus B_1| + |(B_2 \setminus B_1) \setminus Y|        \\
        ={}    & |B_2| = |B_1|                                                                 \\
        \geq{} & |B_1 \cap B_2| + |X \setminus Y| \quad (B_2 \cap (X \setminus Y) = \emptyset) \\
        >{}    & |B_1 \cap B_2| + |Y \setminus X| \quad (|X| > |Y|)                            \\
        \geq{} & |B_1 \cap B_2| + |Y \setminus B_1|.
    \end{align*}
    したがって$(B_2 \setminus B_1) \setminus Y \neq \emptyset$.$y \in (B_2 \setminus B_1) \setminus Y$をとると,$\exists x \in B_1 \setminus B_2$ s.t. $(B_2 \setminus \{y\}) \cup \{x\} \in \mathcal{B}$となり,$|B_1 \cap B_2|$の最大性に矛盾.
\end{proof}

\setcounter{exercise}{7}
\begin{exercise}
    $\mathcal{B}$が何らかのマトロイド$(E, \mathcal{F})$の基の族であることは,以下が同時に成立することと同値であることを示せ. \begin{enumerate}
        \item[(B1)] $\mathcal{B} \neq \emptyset$,
        \item[(B2)] $\forall B_1, B_2 \in \mathcal{B}$, $\forall y \in B_2 \setminus B_1$, $\exists x \in B_1 \setminus B_2$ s.t. $(B_1 \setminus \{x\}) \cup \{y\} \in \mathcal{B}$.
    \end{enumerate}
\end{exercise}
\begin{proof}
    \begin{equation*}
        \overline{\mathcal{B}} \coloneqq \{\overline{B} : B \in \mathcal{B}\} \quad (\overline{B} = E \setminus B),
    \end{equation*}
    とする.(B2)は \begin{equation*}
        \text{$\forall \overline{B_1}, \overline{B_2} \in \overline{\mathcal{B}}$,
            $\forall y \in \overline{B_1} \setminus \overline{B_2}$,
            $\exists x \in \overline{B_2} \setminus \overline{B_1}$
            s.t. $(\overline{B_1} \setminus \{y\}) \cup \{x\} \in \overline{\mathcal{B}}$},
    \end{equation*}
    と同値.また,(M3'')より$\mathcal{B}$がマトロイドの基の族であることと$\overline{\mathcal{B}}$が基の族であることは同値.
\end{proof}

\newpage

\subsubsection*{ランク関数に関する公理系}

\begin{theorem}
    $E$を有限集合とし,$r : 2^E \to \bbZ_+$とする.次の主張は同値. \begin{enumerate}
        \item $\mathcal{F} \coloneqq \{F \subseteq E : r(F) = |F|\}$とするとき,$r$はマトロイド$(E, \mathcal{F})$のランク関数.
        \item 各$X, Y \subseteq E$について, \begin{enumerate}
                  \item[(R1)] $r(X) \leq |X|$,
                  \item[(R2)] $X \subseteq Y$ $\Longrightarrow$ $r(X) \leq r(Y)$,
                  \item[(R3)] $r(X \cup Y) + r(X \cap Y) \leq r(X) + r(Y)$ (劣モジュラ性).
              \end{enumerate}
        \item 各$X \subseteq E$と$x,y \in E$について, \begin{enumerate}
                  \item[(R1')] $r(\emptyset) = 0$,
                  \item[(R2')] $r(X) \leq r(X \cup \{y\}) \leq r(X) + 1$,
                  \item[(R3')] $r(X \cup \{x\}) = r(X \cup \{y\}) = r(X)$ $\Longrightarrow$ $r(X \cup \{x,y\}) = r(X)$.
              \end{enumerate}
    \end{enumerate}
\end{theorem}
\begin{proof}
    (a) $\Longrightarrow$ (b): $r$はランク関数なので,(R1)と(R2)は明らか.(R3)を示す.

    $X, Y \subseteq E$について,$X \cap Y$の基$A$をとる.基$A$を拡張して,$A \dotcup B$を$X$の基,$A \dotcup B \dotcup C$を$X \cup Y$の基とする.$C \subseteq Y$より$A \cup C \subseteq Y$.また$A \cup B \cup C \in \mathcal{F}$より$A \cup C \in \mathcal{F}$.したがって \begin{align*}
        r(X) + r(Y)
         & \geq |A \cup B| + |A \cup C| \\
         & = |A \cup B \cup C| + |A|    \\
         & = r(X \cup Y) + r(X \cap Y).
    \end{align*}

    (b) $\Longrightarrow$ (c): (R1')は(R1)より,(R2')の前半は(R2)よりよい.また(R3)より \begin{equation*}
        r(X \cup \{y\}) \leq r(X) + r(\{y\}) - r(X \cap \{y\}) \leq r(X) + 1.
    \end{equation*}

    \newpage

    (R3')を示す.$x = y$のときは明らか,$x \neq y$のとき, \begin{align*}
        2r(X) & \leq r(X) + r(X \cup \{x, y\})         &  & \because \ (R2) \\
              & \leq r(X \cup \{x\}) + r(X \cup \{y\}) &  & \because \ (R3) \\
              & = 2r(X),
    \end{align*}
    より$r(X) = r(X \cup \{x,y\})$.

    (c) $\Longrightarrow$ (a): \begin{equation*}
        \mathcal{F} \coloneqq \{F \subseteq E : r(F) = |F|\},
    \end{equation*}
    とする.(M1)は(R1')よりよい.(M2)を示す.$Y \in \mathcal{F}$,$y \in Y$について,$X \coloneqq Y \setminus \{y\}$とすると, \begin{equation*}
        |X| + 1 = |Y| = r(Y) = r(X \cup \{y\}) \leq r(X) + 1 \leq |X| + 1,
    \end{equation*}
    より$X \in \mathcal{F}$.

    (M3')を示す.$X, Y \in \mathcal{F}$,$|X| = |Y| + 1$とする.\begin{equation*}
        X \setminus Y = \{x_1, \ldots, x_k\},
    \end{equation*}
    とおく.$\exists i$ s.t. $Y \cup \{x_i\} \in \mathcal{F}$を示したい.そうでないと仮定する.(R2)より \begin{equation*}
        r(Y \cup \{x_i\}) = r(Y) \quad (i = 1, \ldots, k).
    \end{equation*}
    さらに(R3)より \begin{equation*}
        r(Y \cup \{x_1, x_i\}) = r(Y) \quad (i = 2, \ldots, k).
    \end{equation*}
    そこで,$Y \gets Y \cup \{x_1\}$とし,同じ議論を繰り返すと, \begin{equation*}
        r(Y \cup \{x_1, \ldots, x_k\}) = r(X \cup Y) = r(Y).
    \end{equation*}
    これは$r(X \cup Y) \geq r(X) = |X| = |Y| + 1 > r(Y)$に矛盾.

    \newpage

    最後に$r$がマトロイド$(E, \mathcal{F})$のランク関数であることを示す.$X \subseteq E$について,$X$の基$Y$を$|Y|$が最大になるようにとる.$\forall x \in X \setminus Y$, $r(Y \cup \{x\}) = r(Y)$なので,先ほどと同じ議論により$r(X) = r(Y) = |Y|$.
\end{proof}

\subsubsection*{閉包演算子に関する公理系}

\begin{theorem}
    $E$を有限集合とし,$\sigma : 2^E \to 2^E$とする.$\sigma$があるマトロイド$(E, \mathcal{F})$の閉包演算子であることは,任意の$X, Y \subseteq E$と$x, y \in E$について以下が同時に成立することと同値. \begin{enumerate}[label=(S\arabic*)]
        \item $X \subseteq \sigma(X)$,
        \item $X \subseteq Y \subseteq E$ $\Longrightarrow$ $\sigma(X) \subseteq \sigma(Y)$,
        \item $\sigma(X) = \sigma(\sigma(X))$,
        \item $y \notin \sigma(X)$, $y \in \sigma(X \cup \{x\})$ $\Longrightarrow$ $x \in \sigma(X \cup \{y\})$.
    \end{enumerate}
\end{theorem}
\begin{proof}
    ($\Longrightarrow$) (S1)はよい.$X \subseteq Y$,$z \in \sigma(X)$について \begin{align*}
        r(Y \cup \{z\})
         & = r(X \cup Y \cup \{z\})                                                    \\
         & \leq r(X \cup \{z\}) + r(Y) - r((X \cup \{z\}) \cap Y) &  & \because \ (R3) \\
         & \leq r(X) + r(Y) - r(X)                                &  & \because \ (R2) \\
         & = r(Y),
    \end{align*}
    より(S2)もよい.

    (R3')を使って$X$に$\sigma(X) \setminus X$の元を追加していくと$r(\sigma(X)) = r(X)$が成り立つ.このランクを保ったまま$\sigma(X)$に新たな元を追加することはできないので,(S3)もよい.

    $y \notin \sigma(X)$,$y \in \sigma(X \cup \{x\})$,$x \notin \sigma(X \cup \{y\})$と仮定する.(R2')より$r(X \cup \{y\}) = r(X) + 1$,$r(X \cup \{x, y\}) = r(X \cup \{x\})$,$r(X \cup \{x, y\}) = r(X \cup \{y\}) + 1$.ゆえに$r(X \cup \{x\}) = r(X) + 2$となり矛盾.よって(S4)もよい.

    \newpage

    ($\Longleftarrow$) \begin{equation*}
        \mathcal{F} \coloneqq \{X \subseteq E : \forall x \in X, \ x \notin \sigma(X \setminus \{x\})\},
    \end{equation*}
    とする.$(E, \mathcal{F})$がマトロイドであることを示す.

    (M1)はよい.$X \subseteq Y \in \mathcal{F}$と$x \in X$に対して \begin{equation*}
        x \notin \sigma(Y \setminus \{x\}) \supseteq \sigma(X \setminus \{x\}),
    \end{equation*}
    より$X \in \mathcal{F}$.よって(M2)もよい.

    \begin{claim}
        $X \in \mathcal{F}$と$Y \subseteq E$が$|X| > |Y|$を満たすとき,$X \not\subseteq \sigma(Y)$.
    \end{claim}
    \begin{subproof}
        $|Y \setminus X|$に関する帰納法で示す.$Y \setminus X = \emptyset$のとき,$Y \subset X$.$x \in X \setminus Y$をとる.\begin{equation*}
            x \notin \sigma(X \setminus \{x\}) \supseteq \sigma(Y),
        \end{equation*}
        より$x \notin \sigma(Y)$,したがって$X \not\subseteq \sigma(Y)$.

        $|Y \setminus X| > 0$とする.$y \in Y \setminus X$をとる.帰納法の仮定より, \begin{equation*}
            x \in X \setminus \sigma(Y \setminus \{y\}),
        \end{equation*}
        がとれる.$x \notin \sigma(Y)$ならよい.そうでないとき,$x \notin \sigma(Y \setminus \{y\})$かつ \begin{equation*}
            x \in \sigma(Y) = \sigma((Y \setminus \{y\}) \cup \{y\}),
        \end{equation*}
        なので,(S4)より \begin{equation*}
            y \in \sigma((Y \setminus \{y\}) \cup \{x\}).
        \end{equation*}
        $y$以外の$Y$の元は$(Y \setminus \{y\}) \cup \{x\} \subseteq \sigma((Y \setminus \{y\}) \cup \{x\})$に属するので, \begin{equation*}
            Y \subseteq \sigma((Y \setminus \{y\}) \cup \{x\}).
        \end{equation*}
        (S2)より \begin{equation*}
            \sigma(Y) \subseteq \sigma(\sigma((Y \setminus \{y\}) \cup \{x\})) = \sigma((Y \setminus \{y\}) \cup \{x\}).
        \end{equation*}
        帰納法の仮定より$X \not\subseteq \sigma((Y \setminus \{y\}) \cup \{x\})$なので,$X \not\subseteq \sigma(Y)$.
    \end{subproof}

    \newpage

    先ほどの主張を用いて(M3)を示す.$X, Y \in \mathcal{F}$が$|X| > |Y|$を満たすとする.主張より,$x \in X \setminus \sigma(Y)$がとれる.ここで$Y \cup \{x\} \in \mathcal{F}$が成り立つことを示す.

    $z \in Y \cup \{x\}$とする.$z \in Y$のとき,$Y \in \mathcal{F}$より$z \notin \sigma(Y \setminus \{z\})$.$z = x$のとき,$z \notin \sigma(Y)$より$z \notin \sigma(Y \setminus \{z\})$.

    $z \notin \sigma(Y \setminus \{z\})$,$x \notin \sigma(Y)$から,(S4)より \begin{equation*}
        z \notin \sigma((Y \setminus \{z\}) \cup \{x\}) \supseteq \sigma((Y \cup \{x\}) \setminus \{z\}).
    \end{equation*}
    したがって$z \notin \sigma((Y \cup \{x\}) \setminus \{z\})$.よって$Y \cup \{x\} \in \mathcal{F}$.

    以上より,$(E, \mathcal{F})$はマトロイド.マトロイド$(E, \mathcal{F})$のランク関数を$r$,閉包演算子を$\sigma'$とする.$\sigma = \sigma'$を示す.

    まず,$X \subseteq E$と$z \in \sigma'(X)$について,$z \in \sigma(X)$を示す.$z \in X \subseteq \sigma(X)$ならよい.そうでない場合を考える.$X$の基$Y$をとる. \begin{equation*}
        r(Y \cup \{z\}) \leq r(X \cup \{z\}) = r(X) = |Y| < |Y \cup \{z\}|,
    \end{equation*}
    より$Y \cup \{z\} \notin \mathcal{F}$.したがって \begin{equation*}
        \text{$\exists y$ s.t. $y \in \sigma((Y \cup \{z\}) \setminus \{y\})$.}
    \end{equation*}
    $y = z$ならば$z \in \sigma(Y \setminus \{z\}) \subseteq \sigma(X)$.$y \neq z$ならば,$y \notin \sigma(Y \setminus \{y\})$なので(S2)より$z \in \sigma(Y) \subseteq \sigma(X)$.

    次に,$z \notin \sigma'(X)$ (すなわち$r(X \cup \{z\}) > r(X)$)として$z \notin \sigma(X)$を示す.$X \cup \{z\}$の基$Y$をとる.$z \notin Y$だとすると$Y \subseteq X$となって$r(X \cup \{z\}) > r(X)$に矛盾するため,$z \in Y$.また \begin{equation*}
        |Y \setminus \{z\}| = |Y| - 1 = r(X \cup \{z\}) - 1 = r(X),
    \end{equation*}
    より$Y \setminus \{z\}$は$X$の基.ゆえに \begin{equation*}
        X \subseteq \sigma'(Y \setminus \{z\}) \subseteq \sigma(Y \setminus \{z\}),
    \end{equation*}
    よって$\sigma(X) \subseteq \sigma(Y \setminus \{z\})$.$z \notin \sigma(Y \setminus \{z\})$より$z \notin \sigma(X)$.
\end{proof}

\subsubsection*{サーキット族に関する定義}

\begin{theorem}
    $E$を有限集合とし,$\mathcal{C} \subseteq 2^E$とする.\begin{equation*}
        \mathcal{F} = \{F \subset E : \text{$C \subseteq F$なる$C \in \mathcal{C}$が存在しない}\},
    \end{equation*}
    とするとき,$\mathcal{C}$が独立性システム$(E, \mathcal{F})$のサーキットの族であることは,以下が同時に成立することと同値. \begin{enumerate}
        \item[(C1)] $\emptyset \notin \mathcal{C}$,
        \item[(C2)] $C_1, C_2 \in \mathcal{C}$, $C_1 \subseteq C_2$ $\Longrightarrow$ $C_1 = C_2$.
    \end{enumerate}
    さらに,$\mathcal{C}$が独立性システム$(E, \mathcal{F})$のサーキットの族であるとき,以下は同値. \begin{enumerate}
        \item[(a)] $(E, \mathcal{F})$はマトロイド.
        \item[(b)] $X \in \mathcal{F}$,$e \in E$について,$X \cup \{e\}$は高々$1$個のサーキットしか持たない.
        \item[(C3)] 任意の相異なる$C_1, C_2 \in \mathcal{C}$と$e \in C_1 \cap C_2$について,$C_3 \subseteq (C_1 \cup C_2) \setminus \{e\}$なる$C_3 \in \mathcal{C}$が存在する.
        \item[(C3')] 任意の$C_1, C_2 \in \mathcal{C}$,$e \in C_1 \cap C_2$,$f \in C_1 \setminus C_2$について,$f \in C_3 \subseteq (C_1 \cup C_2) \setminus \{e\}$なる$C_3 \in \mathcal{C}$が存在する.
    \end{enumerate}
    \begin{proof}[\textup{前半の証明}]
        ($\Longrightarrow$) $\emptyset \in \mathcal{F}$より(C1)はよい.(C2)はサーキットの極小性よりよい.

        ($\Longleftarrow$) (M1)は(C1)よりよい.(M2)は定義より明らか.任意のサーキットが$\mathcal{C}$に属することは定義から確認できる.(C2)より$\mathcal{C}$はサーキットの族.
    \end{proof}
    \begin{proof}[\textup{後半の証明}]
        (a) $\Longrightarrow$ (C3'): $\mathcal{C}$をマトロイドのサーキットの族とする.$C_1, C_2 \in \mathcal{C}$,$e \in C_1 \cap C_2$,$f \in C_1 \setminus C_2$について,(R3)(ランク関数の劣モジュラ性)より \begin{align*}
            & |C_1| - 1 + r((C_1 \cup C_2) \setminus \{e,f\}) + |C_2| - 1 \\
            ={}& r(C_1) + r((C_1 \cup C_2) \setminus \{e,f\}) + r(C_2) \\
            \geq{}& r(C_1) + r((C_1 \cup C_2) \setminus \{f\}) + r(C_2 \setminus \{e\}) \\
            \geq{}& r(C_1 \setminus \{f\}) + r(C_1 \cup C_2) + r(C_2 \setminus \{e\}) \\
            ={}& |C_1| - 1 + r(C_1 \cup C_2) + |C_2| - 1.
        \end{align*}
        これより$r((C_1 \cup C_2) \setminus \{e,f\}) = r(C_1 \cup C_2)$.$(C_1 \cup C_2) \setminus \{e,f\}$の基$B$をとる.$|B \cup \{f\}| > r(C_1 \cup C_2)$より,$B \cup \{f\}$はサーキット$C_3$を含む.よって$f \in C_3 \subseteq (C_1 \cup C_2) \setminus \{e\}$.

        (C3') $\Longrightarrow$ (C3): 明らか.

        (C3) $\Longrightarrow$ (b): $X \in \mathcal{F}$について$X \cup \{e\}$が相異なるサーキット$C_1, C_2$を含むとき,(C3)より$(C_1 \cup C_2) \setminus \{e\} \notin \mathcal{F}$.しかし$(C_1 \cup C_2) \setminus \{e\} \subseteq X$なので,$X \notin \mathcal{F}$.

        (b) $\Longrightarrow$ (a): 定理13.8と命題13.7よりよい.
    \end{proof}
\end{theorem}

\end{document}
