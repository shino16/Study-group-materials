\documentclass[xelatex,ja=standard,a4paper,14pt,everyparhook=compat]{bxjsarticle}
\usepackage{amsmath, amssymb, amsthm}
\usepackage{enumitem, mleftright, mathtools, bm}
\usepackage{ascmac}

\setenumerate{label=(\alph*)}

\usepackage{concmath}
\usepackage[OT1]{fontenc}
\setsansfont{CMU Concrete}
\setmainfont{CMU Concrete}
\setCJKmainfont{Noto Sans JP}
\setCJKsansfont{Noto Sans JP Bold}

\newcommand{\bbN}{\mathbb{N}}
\newcommand{\bbZ}{\mathbb{Z}}
\newcommand{\bbR}{\mathbb{R}}
\newcommand{\bbC}{\mathbb{C}}
\newcommand{\mcF}{\mathcal{F}}
\newcommand{\mcB}{\mathcal{B}}
\newcommand{\mcC}{\mathcal{C}}

\DeclareMathOperator{\OPT}{OPT}

\newcommand{\paren}[1]{\mleft(#1\mright)}
\newcommand{\norm}[1]{\mleft|#1\mright|}
\newcommand{\ifthen}[3]{\text{$#1$ ? $#2$ : $#3$}}
\newcommand{\bool}[1]{[#1]}
\newcommand{\dotcup}{\mathbin{\dot\cup}}

\theoremstyle{definition}
\newtheorem{exercise}{演習}
\newtheorem{theorem}{定理}[section]
\newtheorem*{theorem*}{定理}
\newtheorem{example}[theorem]{例}
\newtheorem*{example*}{例}
\newtheorem*{recall}{復習}
\newtheorem{proposition}[theorem]{命題}
\newtheorem{corollary}[theorem]{系}
\newtheorem{definition}[theorem]{定義}
\newtheorem*{claim}{主張}
\renewcommand{\proofname}{\textup{証明}}

\newenvironment{subproof}[1][\proofname]{%
    \begin{proof}[#1]%
    \renewcommand*{\qedsymbol}{$\blacksquare$}
}{%
    \end{proof}
}

\begin{document}

\setcounter{section}{12}
\section{Matroids}
\setcounter{subsection}{3}
\subsection{The Greedy Algorithm}

\begin{itembox}[l]{最大化問題}
    \begin{tabular}{ll}
        インスタンス: & 独立性システム$(E, \mcF)$,$c: E \to \bbR$.                 \\
        タスク:       & $c(X) = \sum_{e \in X} c(e)$を最大化する$X \in \mcF$の発見.
    \end{tabular}
\end{itembox}

$(E, \mcF)$が何らかのオラクルで与えられる状況を考える.

\textbf{独立性オラクル}:$F \subseteq E$について,$F \in \mcF$か否かを返す.

\textbf{basis-supersetオラクル}:$F \subseteq E$について,$F$が基を含むか否かを返す.

\begin{example*}
    TSP.\textbf{完全}グラフ$G$について \begin{equation*}
        E = V(G), \quad \mcF = \{F \subseteq E : \text{$F$は$G$のハミルトン閉路の一部}\}.
    \end{equation*}
    \begin{itemize}
        \item $(E, \mcF)$の独立性オラクル:簡単
        \item $(E, \mcF)$のbasis-supersetオラクル:NP-完全
    \end{itemize}
\end{example*}

\begin{example*}
    最短経路問題.グラフ$G$について \begin{equation*}
        E = V(G), \quad \mcF = \{F \subseteq E : \text{$F$は$s$-$t$-パスの一部}\}.
    \end{equation*}
    \begin{itemize}
        \item $(E, \mcF)$の独立性オラクル:NP-完全
        \item $(E, \mcF)$のbasis-supersetオラクル:簡単
    \end{itemize}
\end{example*}

\newpage

\begin{itembox}[l]{Best-In-Greedyアルゴリズム}
    \begin{tabular}{ll}
        入力: & 独立性オラクルで与えられる独立性システム$(E, \mcF)$, \\
               & 重み$c : E \to \bbR_+$.                              \\
        出力: & $F \in \mcF$.
    \end{tabular}
\end{itembox}

\begin{enumerate}[label=\arabic*.]
    \item $E=\{e_1,\ldots,e_n\}$を$c(e_1) \geq \cdots \geq c(e_n)$となるようソート.
    \item $F \gets \emptyset$.
    \item \textbf{for} $i \gets 0$ \textbf{to} $n$ \textbf{do}: \textbf{if} $F \cup \{e_i\} \in \mcF$ \textbf{then} $F \gets F \cup \{e_i\}$.
\end{enumerate}

\begin{example*}
    Kruskal法.
\end{example*}

\begin{itembox}[l]{Worst-Out-Greedyアルゴリズム}
    \begin{tabular}{ll}
        入力: & basis-supersetオラクルで与えられる独立性システム$(E, \mcF)$, \\
               & 重み$c : E \to \bbR_+$.                              \\
        出力: & $(E, \mcF)$の基$F$.
    \end{tabular}
\end{itembox}

\begin{enumerate}[label=\arabic*.]
    \item $E=\{e_1,\ldots,e_n\}$を$c(e_1) \leq \cdots \leq c(e_n)$となるようソート.
    \item $F \gets E$.
    \item \textbf{for} $i \gets 0$ \textbf{to} $n$ \textbf{do}: \textbf{if} $F \setminus \{e_i\}$が基を含む \textbf{then} $F \gets F \setminus \{e_i\}$.
\end{enumerate}

\begin{claim}
    $(E, \mcF, c)$に対するBest-In-Greedyアルゴリズムは,$(E, \mcF^*, -c)$に対するWorst-Out-Greedyアルゴリズムと対応する.
\end{claim}
\begin{proof}
    \begin{align*}
        &F \in \mcF \\
        \Longleftrightarrow{}& \text{$\exists B:\text{$(E,\mcF)$の基}$ s.t. $F \subseteq B$} \\
        \Longleftrightarrow{}& \text{$\exists B:\text{$(E,\mcF)$の基}$ s.t. $F^c \supseteq B^c$} \quad (\text{$B^c$は$(E,\mcF^*)$の基})
    \end{align*}
\end{proof}

\newpage

\setcounter{theorem}{18}
\begin{theorem}
    $c:E \to \bbR_+$に対する最大化問題について, \begin{align*}
        G(E,\mcF,c) &= (\text{Best-In-Greedyが発見した解のコスト}), \\
        \OPT(E,\mcF,c) &= (\text{最適解のコスト}),
    \end{align*}
    とすると \begin{equation*}
        q(E,\mcF) \leq \frac{G(E,\mcF,c)}{\OPT(E,\mcF,c)} \leq 1.
    \end{equation*}
    ある$c$は下界を達成する.
\end{theorem}
\begin{recall}
    Rank quotient: \begin{equation*}
        q(E,\mcF) = \min_{F \subseteq E} \frac{\rho(F)}{r(F)}
    \end{equation*}
    Lower rank: \begin{equation*}
        \rho(F) = \min\{|B|: \text{$B$は$F$の基}\}
    \end{equation*}
\end{recall}
\begin{proof}
    $E = \{e_1,\ldots,e_n\}$,$c(e_1) \geq \cdots \geq c(e_n)$とする.これに対するBest-In-Greedyの解を$G_n$とし,最適解を$O_n$とする. \begin{align*}
        E_j &= \{e_1,\ldots,e_j\}, \\
        G_j &= G_n \cap E_j, \\
        O_j &= O_n \cap E_j,
    \end{align*}
    とする.また, \begin{align*}
        d_n &= c(e_n), \\
        d_j &= c(e_j) - c(e_{j+1}),
    \end{align*}
    とする.このとき \begin{align*}
        |O_j| &\leq r(E_j) \qquad (\because\ O_j \in \mcF) \\
        |G_j| &\geq \rho(E_j) \qquad (\because\ \text{$G_j$は$E_j$の基}).
    \end{align*}
    したがって, \begin{align*}
        \frac{c(G_n)}{c(O_n)}
        &= \frac{\sum_{j=1}^n (|G_j|-|G_{j-1}|) c(e_j)}{\sum_{j=1}^n (|O_j|-|O_{j-1}|) c(e_j)} \\
        &= \frac{\sum_{j=1}^n |G_j| d_j}{\sum_{j=1}^n |O_j| d_j} \\
        &\geq \frac{\sum_{j=1}^n \rho(E_j) d_j}{\sum_{j=1}^n r(E_j) d_j} \\
        &\geq q(E,\mcF).
    \end{align*}
    ここで,
    \begin{equation*}
        q(E,\mcF) = \frac{\rho(F)}{r(F)} = \frac{|B_1|}{|B_2|}
    \end{equation*}
    となるように$F \subseteq E$と$F$の基$B_1$,$B_2$をとる. \begin{equation*}
        c(e) = \begin{cases}
            1 & (e \in F) \\
            0 & (e \notin F)
        \end{cases}
    \end{equation*}
    とし, \begin{gather*}
        c(e_1) \geq \cdots \geq c(e_n), \\
        B_1 = \{e_1, \ldots, e_{|B_1|}\},
    \end{gather*}
    となるように$E$の元を並べると, \begin{align*}
        G(E,\mcF,c) &= |B_1|, \\
        \OPT(E,\mcF,c) &\geq |B_2|.
    \end{align*}
\end{proof}

\newpage

\begin{theorem}[Edmonds-Radoの定理]
    次は同値. \begin{enumerate}
        \item $(E,\mcF)$はマトロイド.
        \item 任意の$c:E\to\bbR_+$について,Best-In-Greedyは$(E,\mcF,c)$に関する最大化問題の最適解を与える.
    \end{enumerate}
\end{theorem}
\begin{proof}
    \begin{align*}
        & \text{$(E,\mcF)$はマトロイド} \\
        \Longleftrightarrow{}& q(E,\mcF) = 1 \\
        \Longleftrightarrow{}& \text{Best-In-Greedyは常に最適解を与える.}
    \end{align*}
\end{proof}

\begin{theorem}
    マトロイド$(E,\mcF)$,ランク関数$r:2^E \to \bbZ_+$をとる.

    $(E,\mcF)$の\textbf{マトロイド超多面体(matroid polytope)}($\mcF$の元の接続ベクトルの凸包)は \begin{equation*}
        \left\{ x \in \bbR^E : \text{$x \geq 0$, $\sum_{e \in A} x_e \leq r(A)$ for all $A \subseteq E$} \right\}
    \end{equation*}
\end{theorem}

\begin{proof}
    ($\subseteq$) は明らか.

    ($\supseteq$) $e \in E$について$r(\{e\}) \leq 1$より,超多面体の各頂点$x$について$0 \leq x \leq 1$.これに$x \in \bbZ^E$という条件が加われば,$x$はある$F \subseteq E$の接続ベクトルとなり, \begin{equation*}
        \sum_{e \in F} x_e = |F| \leq r(F)
    \end{equation*} より$F \in \mcF$,したがって$x \in (\text{matroid polytope})$.したがって,超多面体の頂点がすべて整数ベクトルであればよい.そこで, \begin{equation*}
        \max \left\{ cx : \text{$x \geq 0$, $\sum_{e \in A} x_e \leq r(A)$ for all $A \subseteq E$} \right\}
    \end{equation*}
    が任意の$c \in \bbR^E$について整数解を持つことを示す.

    $c: E \to \bbR$をとる.ある$e \in E$について$c(e) < 0$ならば最適解$x$は$x_e = 0$を満たすので,そのような$e$は考慮から外してよい.そこで$c \geq 0$を仮定し,$(E,\mcF,c)$に対する最大化問題を考える.

    LPの最適解$x$について, \begin{equation*}
        \text{$s_j = x_{e_1} + \cdots + x_{e_j}$ ($\leq r(\{e_1,\ldots,e_j\}) = r(E_j)$)},
    \end{equation*}
    とすると,
     \begin{align*}
        \frac{c(G_n)}{\sum_{e \in E} c(e) x_e}
        &= \frac{\sum_{j=1}^n (|G_j| - |G_{j-1}|) c(e_j)}{\sum_{j=1}^n (s_j - s_{j-1}) c(e_j)} \\
        &= \frac{\sum_{j=1}^n |G_j| d_j}{\sum_{j=1}^n s_j d_j} \\
        &\geq \frac{\sum_{j=1}^n \rho(E_j) d_j}{\sum_{j=1}^n r(E_j) d_j} \\
        &\geq q(E, \mcF) = 1.
    \end{align*}
    したがって$c(G_n) \geq cx$なので,$G_n$もLPの最適解.
\end{proof}

\begin{theorem}
    独立性システム$(E, \mcF)$,$c: E \to \bbR_+$に関する\textbf{最小}化問題について, \begin{align*}
        G(E,\mcF,c) &= (\text{Worst-Out-Greedyが発見した解のコスト}), \\
        \rho^* &= (\text{$(E,\mcF^*)$のlower rank関数}), \\
        r^* &= (\text{$(E,\mcF^*)$のランク関数})
    \end{align*}
    とすると, \begin{equation*}
        1 \leq \frac{G(E,\mcF,c)}{\OPT(E,\mcF,c)} \leq \max_{F \subseteq E} \frac{|F| - \rho^*(F)}{|F| - r^*(F)}.
    \end{equation*}
    ある$c$は上界を達成する.
\end{theorem}

\begin{proof}
    分子側($|G_j| \leq |E_j| - \rho^*(E_j)$)を示す.$E_j \setminus G_j$が$(E,\mcF^*)$における$E_j$の基であると言えれば, \begin{equation*}
        |E_j| - |G_j| = |E_j \setminus G_j| = r^*(E_j) \geq \rho^*(E_j).
    \end{equation*}
    \begin{enumerate}
        \item $E_j \setminus G_j \in \mcF^*$:$G_n$は$(E,\mcF)$の基であり,$(E_j \setminus G_j) \cap G_n = \emptyset$.
        \item $(E_j \setminus G_j) \cup \{e\} \notin \mcF^*$ for all $e \in G_j$:$\forall B$(基) $G_j \setminus \{e\} \not\supseteq B$より,$\forall B$(基) $((E_j \setminus G_j) \cup \{e\}) \cap B \neq \emptyset$.
    \end{enumerate}

    分母側($|O_j| \geq |E_j| - r^*(E_j)$)を示す.$O_n$は$(E,\mcF)$の基であり,$(E_j \setminus O_j) \cap O_n = \emptyset$なので,$E_j \setminus O_j \in \mcF^*$,よって \begin{equation*}
        |E_j| - |O_j| = |E_j \setminus O_j| \leq r^*(E_j \setminus O_j) \leq r^*(E_j).
    \end{equation*}

    よって同様の計算により不等式が示される.

    上界を達成する$c$を構成する. \begin{equation*}
        \max_{F \subseteq E} \frac{|F| - \rho^*(F)}{|F| - r^*(F)} = \frac{|F| - \rho^*(F)}{|F| - r^*(F)} = \frac{|F| - |B_1|}{|F| - |B_2|}
    \end{equation*}
    となるように$F \subseteq E$と$F$の($(E,\mcF^*)$での)基$B_1$,$B_2$をとり, \begin{equation*}
        c(e) = \begin{cases}
            1 & (e \in F) \\
            0 & (e \notin F)
        \end{cases}
    \end{equation*}
    とする.$E$の元を$c(e_1) \geq \cdots c(e_n)$,$B_1 = \{e_1,\ldots,e_{|B_1|}\}$となるように並べると, \begin{align*}
        G(E,\mcF,c) &= |F|-|B_1|, \\
        \OPT(E,\mcF,c) &= |F|-|B_2|.
    \end{align*}
\end{proof}

\newpage

\begin{theorem}
    マトロイド$(E,\mcF)$,$C: E \to \bbR$,$X \in \mcF$,$k = |X|$について,$c(X) = \max\{c(Y) : Y \in \mcF,\ |Y| = k\}$であることは,以下が同時に成り立つことと同値. \begin{enumerate}
        \item $y \in E \setminus X$,$X \cup \{y\} \notin \mcF$,$x \in C(X,y)$ $\Longrightarrow$ $c(x) \geq c(y)$.\footnote{$C(X,y)$は$X\cup\{y\}$が含むただ一つの閉路のこと.}
        \item $y \in E \setminus X$,$X \cup \{y\} \in \mcF$,$x \in X$ $\Longrightarrow$ $c(x) \geq c(y)$.
    \end{enumerate}
\end{theorem}

\begin{example*}
    連結グラフ$G$のグラフィックマトロイド$(E,\mcF)$,$k=r(E)=|V|-1$.
\end{example*}

\begin{proof}
    $(\Longrightarrow)$いずれかが成立しない場合,$(X \cup \{y\}) \setminus \{x\}$がより大きな重みを持つ.

    $(\Longleftarrow)$ WLOG $c \geq 0$と仮定する.$\mcF' \coloneqq \{F \in \mcF : |F| \leq k\}$.

    $E = \{e_1,\ldots,e_n\}$の元を重みの降順で並べ,同じ重みのもの同士では$X$の元が先に来るようにする.

    $(E,\mcF')$はマトロイドなので,最大化問題に対するBest-In-Greedyの貪欲解$X'$をとると$c(X) = \max\{c(Y) : Y \in \mcF\}$.

    $X = X'$を示す.$X \neq X'$を仮定する.

    $X'$は基なので$|X| = k = |X'|$.$e_i \in X' \setminus X$なる最小の$i$をとる.このとき$j < i$について$e_j \in X$ $\Longleftrightarrow$ $e_j \in X'$.

    $X \cup \{e_i\} \notin \mcF$ならば,(a)より各$e_j \in C(X,e_i)$は$j \leq i$を満たす.$e_j \in X$または$e_j = e_i$より$e_j \in X'$.したがって$C(X,e_i) \subseteq X'$となり,$X' \in \mcF'$に矛盾.

    $X \cup \{e_i\} \in \mcF$ならば,(b)より各$e_j \in X$は$j < i$を満たす.したがって$e_j \in X'$より,$X \subset X'$.$|X| = |X'|$に矛盾.
\end{proof}

\end{document}
