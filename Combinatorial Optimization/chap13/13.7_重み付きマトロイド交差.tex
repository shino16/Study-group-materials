\documentclass[xelatex,ja=standard,a4paper,14pt,everyparhook=compat]{bxjsarticle}
\usepackage{amsmath, amssymb, amsthm}
\usepackage{enumitem, mleftright, mathtools, bm}
\usepackage{ascmac}

\setenumerate{label=(\alph*)}

% \usepackage{concmath}
% \usepackage[OT1]{fontenc}
% \setsansfont{CMU Concrete}
% \setmainfont{CMU Concrete}
% \setCJKmainfont{Noto Sans JP}
% \setCJKsansfont{Noto Sans JP Bold}

\newcommand{\bbN}{\mathbb{N}}
\newcommand{\bbZ}{\mathbb{Z}}
\newcommand{\bbR}{\mathbb{R}}
\newcommand{\bbC}{\mathbb{C}}
\newcommand{\mcF}{\mathcal{F}}
\newcommand{\mcB}{\mathcal{B}}
\newcommand{\mcC}{\mathcal{C}}

\DeclareMathOperator{\OPT}{OPT}

\newcommand{\paren}[1]{\mleft(#1\mright)}
\newcommand{\norm}[1]{\mleft|#1\mright|}
\newcommand{\ifthen}[3]{\text{$#1$ ? $#2$ : $#3$}}
\newcommand{\bool}[1]{[#1]}
\newcommand{\dotcup}{\mathbin{\dot\cup}}

% \newcommand{\A1}{A^{(1)}}
% \newcommand{\A2}{A^{(2)}}
% \newcommand{\barA1}{\bar A^{(1)}}
% \newcommand{\barA2}{\bar A^{(2)}}

\theoremstyle{definition}
\newtheorem{theorem}{定理}
\newtheorem*{theorem*}{定理}
\newtheorem{lemma}{補題}
\newtheorem*{lemma*}{補題}
\newtheorem{example}[theorem]{例}
\newtheorem*{example*}{例}
\newtheorem{definition}[theorem]{定義}
\newtheorem*{definition*}{定義}
\newtheorem{proposition}[theorem]{命題}
\newtheorem*{proposition*}{命題}
\newtheorem{corollary}[theorem]{系}
\newtheorem{problem}{問題}
\newtheorem*{answer}{解答}
\renewcommand{\proofname}{\textup{証明}}

\usepackage{tcolorbox}
\tcbuselibrary{breakable,skins,theorems}

\tcolorboxenvironment{definition}{
    coltitle = black,
    % colback = white,
    colframe = green!35!black,
    fonttitle = \bfseries,
    breakable = true
}

\tcolorboxenvironment{definition*}{
    coltitle = black,
    % colback = white,
    colframe = green!35!black,
    fonttitle = \bfseries,
    breakable = true
}

\tcolorboxenvironment{theorem}{
    coltitle = black,
    % colback = black!10!white,
    colframe = blue!35!black,
    fonttitle = \bfseries,
    breakable = true
}

\tcolorboxenvironment{theorem*}{
    coltitle = black,
    % colback = black!10!white,
    colframe = blue!35!black,
    fonttitle = \bfseries,
    breakable = true
}

\tcolorboxenvironment{proposition}{
    coltitle = black,
    % colback = black!10!white,
    colframe = green!35!black,
    fonttitle = \bfseries,
    breakable = true
}

\tcolorboxenvironment{proposition*}{
    coltitle = black,
    % colback = black!10!white,
    colframe = green!35!black,
    fonttitle = \bfseries,
    breakable = true
}

\tcolorboxenvironment{lemma}{
    coltitle = black,
    % colback = black!10!white,
    colframe = gray!35!black,
    fonttitle = \bfseries,
    breakable = true
}

\tcolorboxenvironment{example}{
    coltitle = black,
    % colback = black!10!white,
    colframe = purple!35!black,
    fonttitle = \bfseries,
    breakable = true
}

\tcolorboxenvironment{example*}{
    coltitle = black,
    % colback = black!10!white,
    colframe = purple!35!black,
    fonttitle = \bfseries,
    breakable = true
}

\tcolorboxenvironment{lemma*}{
    coltitle = black,
    % colback = black!10!white,
    colframe = gray!35!black,
    fonttitle = \bfseries,
    breakable = true
}

\tcolorboxenvironment{proof}{
    blanker,
    breakable,
    left=5mm,
    before skip=10pt,
    after skip=10pt,
    borderline west={1mm}{0pt}{black}
}

\tcolorboxenvironment{proof*}{
    blanker,
    breakable,
    left=5mm,
    before skip=10pt,
    after skip=10pt,
    borderline west={1mm}{0pt}{black}
}

\tcolorboxenvironment{problem}{
    coltitle = black,
    % colback = black!10!white,
    colframe = black!35!black,
    fonttitle = \bfseries,
    breakable = true
}

\tcolorboxenvironment{answer}{
    blanker,
    breakable,
    left=5mm,
    before skip=10pt,
    after skip=10pt,
    borderline west={1mm}{0pt}{black}
}

\begin{document}

\setcounter{section}{12}
\section{Matroids}
\setcounter{subsection}{6}
\subsection{Weighted Matroid Intersection}

\begin{itembox}[l]{重み付きマトロイド交差問題}
    \begin{tabular}{ll}
        インスタンス: & 独立性オラクルで与えられるマトロイド$(E,\mcF_1)$, \\
                       & $(E,\mcF_2)$,重み$c:e\to\bbR$.                   \\
        タスク:       & $c(X)$を最大化する$X \in \mcF_1 \cap \mcF_2$.
    \end{tabular}
\end{itembox}

\setcounter{theorem}{34}
\begin{lemma}
    マトロイド$(E,\mcF)$,重み$c:E\to\bbR$,独立集合$X \in \mcF$について,$x_1,\ldots,x_l \in X$と$y_1,\ldots,y_l \notin X$が \begin{enumerate}
        \item $x_j \in C(X, y_j)$ and $c(x_j) = c(y_j)$ for all j,
        \item $x_i \notin C(X, y_j)$ or $c(x_i) > c(y_j)$ for all i < j,
        \item $x_i \notin C(X, y_j)$ or $c(x_i) \geq c(y_j)$ for all j < i,
    \end{enumerate}
    を満たすとき,$(X \setminus \{x_1,\ldots,x_l\}) \cup \{y_1,\ldots,y_l\} \in \mcF$.
\end{lemma}
\begin{proof}
    $l$で帰納法.

    $l=1$のとき,$x_1 \in C(X,y_1)$より$(X \setminus \{x_1\}) \cup \{y_1\}$.

    $\mu \coloneqq \min c(x_i)$,$h \coloneqq \min \{i : c(x_i) = \mu\}$,$X' \coloneqq (X \setminus \{x_h\}) \cup \{y_h\}$.

    (a)より$X' \in \mcF$.$C(X',y_j) = C(X,y_j)$ ($\forall j \neq h$)が言えれば,$X'$と$(x_i), (y_i)$の残りに帰納法の仮定を適用して終了.

    $h < j$のとき,$c(x_h) = \mu \leq c(x_j) = c(y_j)$なので$x_h \notin C(X,y_j)$.

    $j < h$のとき,$c(x_h) = \mu < c(x_j) = c(y_j)$なので$x_h \notin C(X,y_j)$.

    したがって,$j \neq h$について$x_h \notin C(X,y_j)$.

    ゆえに$C(X,y_j) = C(X \setminus \{x_h\}, y_j) = C(X', y_j)$.
\end{proof}

\newpage

\begin{itembox}[l]{アルゴリズムの概略}
    $X = X_0 = \emptyset$から始めて, \begin{itemize}
        \item 増加パス$P$をとり,$X_{k+1} \gets X_k \mathbin{\bigtriangleup} V(P)$
    \end{itemize}
    を繰り返す(重みなしver.と同様).

    これで得られる$X_0,\ldots,X_m \in \mcF_1 \cap \mcF_2$ ($|X_k| = k$)について, \begin{equation*}
        c(X_k) = \max \{c(X) : X \in \mcF_1 \cap \mcF_2,\ |X| = k\} \quad (\forall k)
    \end{equation*}
    を保証.$\max_k c(X_k)$を達成する$X_k$が答え.
\end{itembox}

\begin{lemma*}[\#]
    $X_k$と$c_1,c_2 : E \to \bbR$が \begin{itemize}
        \item $c_1 + c_2 = c$, and
        \item $c_i(X_k) = \max \{c_i(X) : X \in \mcF_i,\ |X| = k\}$ ($\forall i$)
    \end{itemize}
    を満たすとき,$c(X_k) = \max \{c(X) : X \in \mcF_1 \cap \mcF_2,\ |X| = k\}$.
\end{lemma*}
\begin{proof}
    $\max$を達成する$X$をとると, \begin{align*}
        c(X_k) &\leq c(X) \\
        &= c_1(X) + c_2(X) \\
        &\leq c_1(X_k) + c_2(X_k) \\
        &= c(X_k).
    \end{align*}
\end{proof}

\newpage

補題(\#)の条件を満たす$X_k,c_1,c_2$について,二部グラフを構築する. \begin{align*}
    A^{(1)} &= \{(x,y) : x \in C_1(X_k,y) \setminus \{y\},\ y \in E \setminus X\}, \\
    A^{(2)} &= \{(y,x) : x \in C_2(X_k,y) \setminus \{y\},\ y \in E \setminus X\}, \\
    S &= \{y \in E \setminus X : X_k \cup \{y\} \in \mcF_1\}, \\
    T &= \{y \in E \setminus X : X_k \cup \{y\} \in \mcF_2\}
\end{align*}
とする.ここで \begin{align*}
    m_1 &= \max \{c_1(y) : y \in S\}, \\
    m_2 &= \max \{c_2(y) : y \in T\}, \\
    \bar S &= \{y \in S : c_1(y) = m_1\}, \\
    \bar T &= \{y \in T : c_2(y) = m_2\}, \\
    \bar A^{(1)} &= \{(x,y) \in A^{(1)} : c_1(x) = c_1(y)\}, \\
    \bar A^{(2)} &= \{(y,x) \in A^{(2)} : c_2(x) = c_2(y)\}, \\
\end{align*}
とし,グラフ$\bar G = (E, \bar A^{(1)} \cup \bar A^{(2)})$を考える.

\begin{lemma*}
    \begin{enumerate}
        \item[(I)] $c_1(x) \geq c_1(y)$ ($\forall (x,y) \in A^{(1)}$).
        \item[(II)] $c_2(x) \geq c_2(y)$ ($\forall (y,x) \in A^{(2)}$).
        \item[(i)] $c_1(x) > c_1(y)$ ($\forall (x,y) \in A^{(1)} \setminus \bar A^{(1)}$).
        \item[(ii)] $c_2(x) > c_2(y)$ ($\forall (y,x) \in A^{(2)} \setminus \bar A^{(2)}$).
    \end{enumerate}
\end{lemma*}
\begin{proof}
    (I), (II)は定理13.23よりよい.(i), (ii)は$\bar A^{(1)}, \bar A^{(2)}$の定義よりよい.
\end{proof}

\newpage

\begin{lemma*}
    最短$\bar S$-$\bar T$-パス$P$について,$X_{k+1} = X_k \mathbin{\bigtriangleup} V(P)$とすると \begin{enumerate}[label=(\Alph*)]
        \item $X_{k+1} \in \mcF_1 \cap \mcF_2$, and
        \item $X_{k+1},c_1,c_2$は補題(\#)の条件を満たす.
    \end{enumerate}
\end{lemma*}
\begin{proof}
    (A) $P = y_0x_1y_1\cdots x_ly_l$とおく.

    $X_{k+1} \in \mcF_1$を示す.$y_0 \in S$より$X_k \cup \{y_0\} \in \mcF_1$,また$C(X_k,y_j) = C(X_k \cup \{y_0\},y_j)$.

    $X_k \cup \{y_0\}$,$c_1$,$x_1,\ldots,x_l$と$y_1,\ldots,y_l$に対して補題13.35を適用. \begin{enumerate}
        \item $(x_j,y_j) \in \bar A^{(1)}$より$x_j \in C_1(X_k,y_j)$ and $c_1(x_j) = c_1(y_j)$.
        \item $i < j$とする.$P$は最短なので$(x_i,y_j) \notin \bar A^{(1)}$
    \end{enumerate}
\end{proof}

\end{document}
