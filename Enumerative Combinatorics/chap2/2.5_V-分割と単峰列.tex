\documentclass[xelatex,ja=standard,a4paper,14pt,everyparhook=compat]{bxjsarticle}
\usepackage{amsmath, amssymb, amsthm}
\usepackage{enumitem, mleftright, mathtools, hyperref, bm}
\usepackage{caption, wrapfig}
\usepackage{multirow}

\newcommand{\up}[1]{\smash{\raisebox{.5\normalbaselineskip}{$#1$}}}

\setenumerate{label=(\alph*)}

\usepackage{concmath}
\usepackage[OT1]{fontenc}
\setsansfont{CMU Concrete}
\setmainfont{CMU Concrete}
\setCJKmainfont{Noto Sans CJK JP}
\setCJKsansfont{Noto Sans CJK JP}

\newcommand{\bbN}{\mathbb{N}}
\newcommand{\bbZ}{\mathbb{Z}}
\newcommand{\bbC}{\mathbb{C}}
\newcommand{\bbP}{\mathbb{P}}
\newcommand{\fS}{\mathfrak{S}}
\newcommand{\inv}[1]{#1^{-1}}

\newcommand{\paren}[1]{\mleft(#1\mright)}
\newcommand{\norm}[1]{\mleft|#1\mright|}
\newcommand{\bbinom}[2]{\mleft(\binom{#1}{#2}\mright)}
\newcommand{\ifthen}[3]{\text{$#1$ ? $#2$ : $#3$}}
\newcommand{\bool}[1]{[#1]}
\newcommand{\dcup}{\mathbin{\dot\cup}}
\newcommand{\qbinom}[2]{\pmb{\binom{#1}{#2}}}

\theoremstyle{definition}
\newtheorem{theorem}{定理}[subsection]
\newtheorem*{theorem*}{定理}
\newtheorem{example}[theorem]{例}
\newtheorem{proposition}[theorem]{命題}
\newtheorem{corollary}[theorem]{系}
\renewcommand{\proofname}{\textup{証明}}

\begin{document}

\setcounter{section}{1}
\section{Sieve Methods}
\setcounter{subsection}{4}
\subsection{$V$-Partitions and Unimodal Sequences}

\textbf{重み$n$の単峰列 ($n$-stack)}:次を満たす正整数の列$d_1 d_2 \cdots d_m$ \begin{enumerate}[label=\alph*.]
    \item $\sum d_i = n$,
    \item $d_1 \leq d_2 \leq \cdots \leq d_j \geq d_{j+1} \geq \cdots \geq d_m$ for some $j$.
\end{enumerate}
重み$n$の単峰数列の個数を$u(n)$とする.$u(0) = 0$とする.$u(n)$の母関数を \begin{align*}
    U(q) & = \sum_{n \geq 0} u(n) q^n                         \\
         & = q + 2q^2 + 4q^3 + 8q^4 + 15q^5 + 27q^6 + \cdots,
\end{align*}
とおく.

$[j]! = (1-q)(1-q^2) \cdots (1-q^j)$とおく.単峰列の最大の項に着目して\footnote{$\frac{1}{[k-1]!}$を無くしたものは$p(n)$の母関数にあたる.} \begin{equation*}
    U(q) = \sum_{k \geq 1} \frac{1}{[k-1]!} \cdot q^k \cdot \frac{1}{[k]!} = \sum_{k \geq 1} \frac{q^k}{[k-1]![k]!}.
\end{equation*}
分割数$p(n)$の母関数の式 \begin{equation*}
    \sum_{n \geq 0} p(n) q^n = \prod_{i \geq 1} (1-q^i)^{-1},
\end{equation*}
に対応する$U(n)$の式がほしい.そこで,$n$の分割と重み$n$の単峰列の中間的なものを考える.

\newpage

\textbf{$n$の$V$-分割}:次のa.,b.を満たす非負整数の並び \begin{equation*}
    \begin{bmatrix}
               & a_1 & a_2 & \cdots \\
        \up{c} & b_1 & b_2 & \cdots
    \end{bmatrix}
\end{equation*}
\begin{enumerate}[label=\alph*.]
    \item $c + \sum a_i + \sum b_i = n$,
    \item $c \geq a_1 \geq a_2 \geq \cdots$, $c \geq b_1 \geq b_2 \geq \cdots$.
\end{enumerate}
$n$の$V$-分割の個数を$v(n)$とする.$v(0) = 1$とする.母関数を \begin{align*}
    V(q) & = \sum_{n \geq 0} v(n) q^n                      \\
         & = 1 + q + 3q^2 + 6q^3 + 12q^4 + 21q^5 + \cdots,
\end{align*}
とおく.このとき \begin{equation*}
    V(q) = \sum_{k \geq 0} \frac{q^k}{[k]!^2}.
\end{equation*}

さらに$n$の\textbf{ダブル分割}を次のa.,b.を満たす非負整数の並びとして定義する.
\begin{equation*}
    \begin{bmatrix}
        a_1 & a_2 & \cdots \\
        b_1 & b_2 & \cdots
    \end{bmatrix}
\end{equation*}
\begin{enumerate}[label=\alph*.]
    \item $\sum a_i + \sum b_i = n$,
    \item $a_1 \geq a_2 \geq \cdots$, $b_1 \geq b_2 \geq \cdots$.
\end{enumerate}

\newpage

$n$のダブル分割の個数を$d(n)$とすると,母関数は \begin{equation*}
    \sum_{n \geq 0} d(n) q^n = \prod_{i \geq 1} (1-q^i)^{-2}.
\end{equation*}

$n$の$V$-分割全体の集合を$V_n$,ダブル分割全体の集合を$D_n$とする.Sieve methodにより,$\#V_n$を$\#D_n$を用いて表したい.

$\Gamma_1: D_n \to V_n$を \begin{equation*}
    \Gamma_1 \begin{bmatrix}
        a_1 & a_2 & \cdots \\
        b_1 & b_2 & \cdots
    \end{bmatrix} = \begin{cases}
        \begin{bmatrix}
                     & a_2 & a_3 & \cdots \\
            \up{a_1} & b_1 & b_2 & \cdots
        \end{bmatrix} & \text{if $a_1 \geq b_1$}, \\
        \begin{bmatrix}
                     & a_1 & a_2 & \cdots \\
            \up{b_1} & b_2 & b_3 & \cdots
        \end{bmatrix} & \text{if $b_1 > a_1$},
    \end{cases}
\end{equation*}
で定める.$\Gamma_1$は全射だが,単射にはならない.ちょうど$2$回カウントされる$V$-分割の集合は \begin{equation*}
    V_n^1 \coloneqq \mleft\{ \begin{bmatrix}
               & a_1 & a_2 & \cdots \\
        \up{c} & b_1 & b_2 & \cdots
    \end{bmatrix} \in V_n : c > a_1 \mright\}.
\end{equation*}
したがって \begin{equation*}
    \#V_n = \#D_n - \#V_n^1.
\end{equation*}

$\#V_n^1$を数えるため,$\Gamma_2 : D_{n-1} \to V_n^1$を次で定める. \begin{equation*}
    \Gamma_2 \begin{bmatrix}
        a_1 & a_2 & \cdots \\
        b_1 & b_2 & \cdots
    \end{bmatrix} = \begin{cases}
        \begin{bmatrix}
                         & a_2 & a_3 & \cdots \\
            \up{a_1 + 1} & b_1 & b_2 & \cdots
        \end{bmatrix} & \text{if $a_1 + 1 \geq b_1$,} \\
        \begin{bmatrix}
                     & a_1 + 1 & a_2 & \cdots \\
            \up{b_1} & b_2     & b_3 & \cdots
        \end{bmatrix} & \text{if $a_1 + 1 < b_1$.}    \\
    \end{cases}
\end{equation*}

\newpage

先ほどと同様に,$\Gamma_2$は全射だが単射ではなく, \begin{equation*}
    V_n^2 = \mleft\{ \begin{bmatrix}
               & a_1 & a_2 & \cdots \\
        \up{c} & b_1 & b_2 & \cdots
    \end{bmatrix} \in V_n : c > a_1 > a_2 \mright\},
\end{equation*}
の元を$2$回カウントする.したがって$\#V_n^1 = \#D_{n-1} - \#V_n^2$であり, \begin{equation*}
    \#V_n = \#D_n - \#D_{n-1} + \#V_n^2.
\end{equation*}

同様に,$\Gamma_3 : D_{n-3} \to V_n^2$を \begin{equation*}
    \Gamma_3 \begin{bmatrix}
        a_1 & a_2 & \cdots \\
        b_1 & b_2 & \cdots
    \end{bmatrix} = \begin{cases}
        \begin{bmatrix}
                         & a_2 + 1 & a_3 & \cdots \\
            \up{a_1 + 2} & b_1     & b_2 & \cdots
        \end{bmatrix} & \text{if $a_1 + 2 \geq b_1$,} \\
        \begin{bmatrix}
                     & a_1 + 2 & a_2 + 1 & a_3 & \cdots \\
            \up{b_1} & b_2     & b_3     & b_4 & \cdots
        \end{bmatrix} & \text{if $a_1 + 2 < b_1$,} \\
    \end{cases}
\end{equation*}
で定めると,これは \begin{equation*}
    V_n^3 = \mleft\{ \begin{bmatrix}
               & a_1 & a_2 & \cdots \\
        \up{c} & b_1 & b_2 & \cdots
    \end{bmatrix} \in V_n : c > a_1 > a_2 > a_3 \mright\},
\end{equation*}
の元を$2$回カウントする.したがって \begin{equation*}
    \#V_n = \#D_n - \#D_{n-1} + \#D_{n-3} - \#V_n^3.
\end{equation*}

このようにして$\Gamma_i : D_{n-\binom{i}{2}} \to V_n^{i-1}$を構成することを繰り返せば, \begin{equation*}
    v(n) = d(n) - d(n-1) + d(n-3) - d(n-6) + \cdots,
\end{equation*}
が得られる(ただし$m < 0$について$d(m) = 0$とする).

\newpage

\begin{proposition}
    \begin{equation*}
        V(q) = \mleft( \sum_{n \geq 0} (-1)^n q^{\binom{n+1}{2}} \mright) \prod_{i \geq 1} (1-q^i)^{-2}.
    \end{equation*}
\end{proposition}

\begin{proposition}
    \begin{equation*}
        U(q) + V(q) = \prod_{i \geq 1} (1-q^i)^{-2}.
    \end{equation*}
\end{proposition}

重み$n$の単峰列全体の集合を$U_n$とする.$D_n \to U_n \cup V_n$の全単射を次で構成できる. \begin{equation*}
    \begin{bmatrix}
        a_1 & a_2 & \cdots \\
        b_1 & b_2 & \cdots
    \end{bmatrix} \mapsto \begin{cases}
        \begin{bmatrix}
                     & a_2 & a_3 & \cdots \\
            \up{a_1} & b_1 & b_2 & \cdots
        \end{bmatrix} & \text{if $a_1 \geq b_1$}, \\
        \cdots a_2 a_1 b_1 b_2 \cdots & \text{if $b_1 > a_1$}.
    \end{cases}
\end{equation*}

\begin{corollary}
    \begin{equation*}
        U(q) = \mleft( \sum_{n \geq 1} (-1)^{n-1} q^{\binom{n+1}{2}} \mright) \prod_{i \geq 1} (1-q^i)^{-2}.
    \end{equation*}
\end{corollary}

\end{document}
