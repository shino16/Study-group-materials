\documentclass[xelatex,ja=standard,a4paper,14pt,everyparhook=compat]{bxjsarticle}
\usepackage{amsmath, amssymb, amsthm}
\usepackage{enumitem, mleftright, mathtools, hyperref}
\usepackage{caption, wrapfig}

% \usepackage{zxjatype}
% \renewcommand*\familydefault{\sfdefault}
% % なぜか Segoe UI で () が出力できない
% \setmainfont[BoldFont={Segoe UI SemiBold}]{Segoe UI}
% \setsansfont[BoldFont={Segoe UI Bold}]{Segoe UI}
% \setCJKmainfont{Noto Sans CJK JP}
% \setCJKsansfont{Noto Sans CJK JP}
% \usepackage{concmath}

\setenumerate{label=(\alph*)}

\usepackage{concmath}
\usepackage[OT1]{fontenc}
\setsansfont{CMU Concrete}
\setmainfont{CMU Concrete}
\setCJKmainfont{Noto Sans JP}
\setCJKsansfont{Noto Sans JP}

\newcommand{\bbN}{\mathbb{N}}
\newcommand{\bbZ}{\mathbb{Z}}
\newcommand{\bbC}{\mathbb{C}}
\newcommand{\fS}{\mathfrak{S}}
\newcommand{\inv}[1]{#1^{-1}}

\newcommand{\paren}[1]{\mleft(#1\mright)}
\newcommand{\bbinom}[2]{\mleft(\binom{#1}{#2}\mright)}
\newcommand{\ifthen}[3]{\text{$#1$ ? $#2$ : $#3$}}
\newcommand{\bool}[1]{[#1]}
\newcommand{\dcup}{\mathbin{\dot\cup}}

\theoremstyle{definition}
\newtheorem{theorem}{定理}[subsection]
\newtheorem*{theorem*}{定理}
\newtheorem{example}[theorem]{例}
\newtheorem{proposition}[theorem]{命題}
\newtheorem{corollary}[theorem]{系}
\renewcommand{\proofname}{\textup{証明}}

\begin{document}

\setcounter{section}{1}
\section{Sieve Methods}
\subsection{Inclusion-Exclusion}

``Sieve method''とは:有限集合$S$の要素数を求める方法

パターン(1) $\#S$を大きめに見積もり、誤差を大きめに見積もり、その誤差を…ということを繰り返し、誤差を$0$に近づけていく

パターン(2) $T \supseteq S$について、余分な元が打ち消しあうように$T$の各元を重みづけする(後の節で登場)

\begin{theorem}
    $n$元集合$S$、線形空間$V = \{f : 2^S \to K\}$ ($K$は体)について、線形写像$\phi: V \to V$を \begin{equation*}
        \phi f(T) = \sum_{Y \supseteq T} f(Y),
    \end{equation*}
    で定める。このとき$\inv \phi$が存在し、 \begin{equation*}
        \inv \phi f(T) = \sum_{Y \supseteq T} (-1)^{\#(Y-T)} f(Y).
    \end{equation*}
\end{theorem}

\begin{proof}
    $\psi : V \to V$を$\psi f(T) = \sum_{Y \subseteq T} (-1)^{\#(Y-T)} f(Y)$で定めると、 \begin{align*}
        \phi \psi f(T)
         & = \sum_{Y \supseteq T} (-1)^{\#(Y-T)} \phi f(Y) \qquad \text{($\psi\phi f(T)$では?)} \\
         & = \sum_{Y \supseteq T} (-1)^{\#(Y-T)} \sum_{Z \supseteq Y} f(Z)                       \\
         & = \sum_{Z \supseteq T} \paren{\sum_{Z \supseteq Y \supseteq T} (-1)^{\#(Y-T)}} f(Z).
    \end{align*}
    $T$、$Z$を固定したとき、$m = \#(Z-T)$とおくと \begin{equation*}
        \sum_{Z \supseteq Y \supseteq T} (-1)^{\#(Y-T)}
        = \sum_{i=0}^m (-1)^i \binom{m}{i}
        = \delta_{0m},
    \end{equation*}
    なので、$\phi \psi f(T) = f(T)$がわかる。よって$\inv \phi = \psi$。
\end{proof}

\subsubsection*{よくある定理2.1.1の適用例}

集合$A$と、$A$の元が持ったり持たなかったりする性質の集合$S$がある。

ちょうど$T \subseteq S$の性質のみを持つ$A$の元の個数$f_=(T)$\footnote{重み$w:A \to K$を決めて、元の個数の代わりに元の重みの和$\sum_x w(x)$を$f_=(T)$としてもよい}は求めにくいが、少なくとも$T \subseteq S$の性質は満たすような$A$の元の個数$f_\geq(T)$は求めやすいようなとき、 \begin{equation*}
    f_\geq(T) = \sum_{Y \supseteq T} f_=(Y),
\end{equation*}
なので、定理2.1.1より \begin{equation*}
    f_=(T) = \sum_{Y \supseteq T} (-1)^{\#(Y-T)} f_\geq(Y).
\end{equation*}
とくに、どの性質も持たないような元の個数は \begin{equation} \label{2.5}
    f_=(\emptyset) = \sum_{Y} (-1)^{\#Y} f_\geq(Y).
\end{equation}

性質を集合で言い換えることもできる。$A_1,\ldots,A_n$を$A$の部分集合とし、 \begin{equation*}
    A_T = \bigcap_{i \in T} A_i,
\end{equation*}
と定める($A_\emptyset = A$とする)。$A_i$を「性質$P_i$を満たす$A$の元の集合」と考えれば、式(\ref{2.5})に対応するのは \begin{align*}
    \#(\overline{A_1 \cup \cdots \cup A_n})
     & = \#(\overline{A_1} \cap \cdots \cap \overline{A_n}) \\
     & = S_0 - S_1 + S_2 - \cdots + (-1)^n S_n,
\end{align*}
ただし \begin{equation*}
    S_k = \sum_{\#T = k} \#A_T.
\end{equation*}

包除原理やその変種は、$\cap$と$\cup$、$\subseteq$と$\supseteq$などを入れ替えることで双対形が得られる。定理2.1.1の双対形は、 \begin{quote}
    $\widetilde\phi : V \to V$を \begin{equation*}
        \widetilde\phi f(T) = \sum_{Y \subseteq T} f(Y),
    \end{equation*}
    で定めるとき、$\inv{\widetilde\phi}$が存在して \begin{equation*}
        \inv{\widetilde\phi} f(T) \sum_{Y \subseteq T} (-1)^{\#(T-Y)} f(Y).
    \end{equation*}
\end{quote}
である。証明も同様。元の定理2.1.1について$g(T) \mapsto f(S-T)$を考えることでも双対形の主張が得られる。

\end{document}
