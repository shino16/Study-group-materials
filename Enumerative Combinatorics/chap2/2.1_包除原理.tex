\documentclass[xelatex,ja=standard,a4paper,14pt,everyparhook=compat]{bxjsarticle}
\usepackage{amsmath, amssymb, amsthm}
\usepackage{enumitem, mleftright, mathtools, hyperref, bm}
\usepackage{caption, wrapfig}

% \usepackage{zxjatype}
% \renewcommand*\familydefault{\sfdefault}
% % なぜか Segoe UI で () が出力できない
% \setmainfont[BoldFont={Segoe UI SemiBold}]{Segoe UI}
% \setsansfont[BoldFont={Segoe UI Bold}]{Segoe UI}
% \setCJKmainfont{Noto Sans CJK JP}
% \setCJKsansfont{Noto Sans CJK JP}
% \usepackage{concmath}

\setenumerate{label=(\alph*)}

\usepackage{concmath}
\usepackage[OT1]{fontenc}
\setsansfont{CMU Concrete}
\setmainfont{CMU Concrete}
\setCJKmainfont{Noto Sans CJK JP}
\setCJKsansfont{Noto Sans CJK JP}

\newcommand{\bbN}{\mathbb{N}}
\newcommand{\bbZ}{\mathbb{Z}}
\newcommand{\bbC}{\mathbb{C}}
\newcommand{\fS}{\mathfrak{S}}
\newcommand{\inv}[1]{#1^{-1}}

\newcommand{\paren}[1]{\mleft(#1\mright)}
\newcommand{\norm}[1]{\mleft|#1\mright|}
\newcommand{\bbinom}[2]{\mleft(\binom{#1}{#2}\mright)}
\newcommand{\ifthen}[3]{\text{$#1$ ? $#2$ : $#3$}}
\newcommand{\bool}[1]{[#1]}
\newcommand{\dcup}{\mathbin{\dot\cup}}
\newcommand{\qbinom}[2]{\pmb{\binom{#1}{#2}}}

\theoremstyle{definition}
\newtheorem{theorem}{定理}[subsection]
\newtheorem*{theorem*}{定理}
\newtheorem{example}[theorem]{例}
\newtheorem{proposition}[theorem]{命題}
\newtheorem{corollary}[theorem]{系}
\renewcommand{\proofname}{\textup{証明}}

\begin{document}

\setcounter{section}{1}
\section{Sieve Methods}
\subsection{Inclusion-Exclusion}

``Sieve method'':有限集合$S$の要素数を求める方法

パターン(1) $\#S$を大きめに見積もり,誤差を大きめに見積もり,その誤差を…ということを繰り返し,誤差を$0$に近づけていく

パターン(2) $T \supseteq S$について,余分な元が打ち消しあうように$T$の各元を重みづけする(後の節で登場)

\begin{theorem}
    $n$元集合$S$,線形空間$V = \{f : 2^S \to K\}$ ($K$は体)について,線形写像$\phi: V \to V$を \begin{equation*}
        \phi f(T) = \sum_{Y \supseteq T} f(Y),
    \end{equation*}
    で定める.このとき$\inv \phi$が存在し, \begin{equation*}
        \inv \phi f(T) = \sum_{Y \supseteq T} (-1)^{\#(Y-T)} f(Y).
    \end{equation*}
\end{theorem}

\begin{proof}
    $\psi : V \to V$を$\psi f(T) = \sum_{Y \supseteq T} (-1)^{\#(Y-T)} f(Y)$で定めると, \begin{align*}
        \phi \psi f(T)
         & = \sum_{Y \supseteq T} (-1)^{\#(Y-T)} \phi f(Y) \qquad \text{($\psi\phi f(T)$では?)} \\
         & = \sum_{Y \supseteq T} (-1)^{\#(Y-T)} \sum_{Z \supseteq Y} f(Z)                       \\
         & = \sum_{Z \supseteq T} \paren{\sum_{Z \supseteq Y \supseteq T} (-1)^{\#(Y-T)}} f(Z).
    \end{align*}
    $T$,$Z$を固定したとき,$m = \#(Z-T)$とおくと \begin{equation*}
        \sum_{Z \supseteq Y \supseteq T} (-1)^{\#(Y-T)}
        = \sum_{i=0}^m (-1)^i \binom{m}{i}
        = \delta_{0m},
    \end{equation*}
    なので,$\phi \psi f(T) = f(T)$がわかる.よって$\inv \phi = \psi$.
\end{proof}

\subsubsection*{よくある定理2.1.1の適用例}

集合$A$と,$A$の元が持ったり持たなかったりする性質の集合$S$がある.

ちょうど$T \subseteq S$の性質のみを持つ$A$の元の個数$f_=(T)$\footnote{重み$w:A \to K$を決めて,元の個数の代わりに元の重みの和$\sum_x w(x)$を$f_=(T)$としてもよい}は求めにくいが,少なくとも$T \subseteq S$の性質は満たすような$A$の元の個数$f_\geq(T)$は求めやすいようなとき, \begin{equation*}
    f_\geq(T) = \sum_{Y \supseteq T} f_=(Y),
\end{equation*}
なので,定理2.1.1より \begin{equation*}
    f_=(T) = \sum_{Y \supseteq T} (-1)^{\#(Y-T)} f_\geq(Y).
\end{equation*}
とくに,どの性質も持たないような元の個数は \begin{equation} \label{2.5}
    f_=(\emptyset) = \sum_{Y} (-1)^{\#Y} f_\geq(Y).
\end{equation}

性質を集合で言い換えることもできる.$A_1,\ldots,A_n$を$A$の部分集合とし, \begin{equation*}
    A_T = \bigcap_{i \in T} A_i,
\end{equation*}
と定める($A_\emptyset = A$とする).$A_i$を「性質$P_i$を満たす$A$の元の集合」と考えれば,式(\ref{2.5})に対応するのは \begin{align*}
    \#(\overline{A_1 \cup \cdots \cup A_n})
     & = \#(\overline{A_1} \cap \cdots \cap \overline{A_n}) \\
     & = S_0 - S_1 + S_2 - \cdots + (-1)^n S_n,
\end{align*}
ただし \begin{equation*}
    S_k = \sum_{\#T = k} \#A_T.
\end{equation*}

包除原理やその変種は,$\cap$と$\cup$,$\subseteq$と$\supseteq$などを入れ替えることで双対形が得られる.定理2.1.1の双対形は, \begin{quote}
    $\widetilde\phi : V \to V$を \begin{equation*}
        \widetilde\phi f(T) = \sum_{Y \subseteq T} f(Y),
    \end{equation*}
    で定めるとき,$\inv{\widetilde\phi}$が存在して \begin{equation*}
        \inv{\widetilde\phi} f(T) = \sum_{Y \subseteq T} (-1)^{\#(T-Y)} f(Y).
    \end{equation*}
\end{quote}
である.証明も同様.元の定理2.1.1について$f'(T) = f(S-T)$を考えることでも双対形の主張が得られる.また,高々$T$の性質しか満たさない$A$の元の個数を$f_\leq(T)$とするとき, \begin{align*}
    f_\leq(T) & = \sum_{Y \subseteq T} f_=(Y),                   \\
    f_=(T)    & = \sum_{Y \subseteq T} (-1)^{\#(T-Y)} f_\leq(Y).
\end{align*}

$f_=(T)$が$T$の要素数によってのみ決まる場合を考える.$i=0, \ldots, n$について \begin{align*}
    a(n-i) & = f_=(T) \quad (\#T = i),    \\
    b(n-i) & = f_\geq(T) \quad (\#T = i),
\end{align*}
とする.このとき$0 \leq m \leq n$について \begin{align*}
    b(m) & = \sum_{i=0}^m \binom{m}{i} a(i),            \\
    a(m) & = \sum_{i=0}^m \binom{m}{i} (-1)^{m-i} b(i).
\end{align*}
$n = \infty$でも上の$2$つの式が同値であることが確認できる.また,有限差分を使うと$0 \leq m \leq n$について \begin{equation*}
    a(m) = \Delta^m b(0),
\end{equation*}
とも表せる.$a(m)$,$b(m)$の値が$n$に依存しないなら,任意の$m \in \bbN$について$a(m) = \Delta^m b(0)$ (命題2.2.2).

\subsection{Examples and Special Cases}

\begin{example}[攪乱順列]

    順列$w \in \fS_n$であって,不動点を持たない($\forall i$ $w(i) \neq i$)ものはいくつあるか?そのような順列(攪乱順列)の個数を$D(n)$とおく.

    少なくとも$T \subseteq [n]$の点が不動点であるような順列の個数は \begin{equation*}
        f_\geq(T) = b(n-i) = (n-i)! \quad (\#T = i).
    \end{equation*}
    したがって,不動点を持たない順列の個数$f_=(\emptyset) = a(n) = D(n)$は \begin{align*}
        D(n) & = \sum_{i=0}^n \binom{n}{i} (-1)^{n-i} i!                                                   \\
             & = n! \paren{1 - \frac{1}{1!} + \frac{1}{2!} - \frac{1}{3!} + \cdots + (-1)^n \frac{1}{n!}}.
    \end{align*}
    これより$n!/e$が$D(n)$の良い近似であることが分かる.実際,$n > 0$について$D(n) = \operatorname{round}(n!/e)$が確認できる.\footnote{
        \begin{equation*}
            \norm{D(n) - \frac{n!}{e}} \leq \frac{1}{n+1} + \frac{1}{(n+1)(n+2)} + \frac{1}{(n+1)(n+2)(n+3)} + \cdots.
        \end{equation*}
        右辺を$g(n+1)$とおくと,$g(n+1) = n g(n) - 1$ ($n \geq 1$)より \begin{equation*}
            g(1) = e-1 \approx 1.718, \quad g(2) \approx 0.718, \quad g(3) \approx 0.436, \quad g(4) = 0.308, \ldots
        \end{equation*}
        WolframAlphaによると$g(n) = e(\Gamma(n) - \Gamma(n,1)) = \int_0^1 t^{n-1} e^{-t} dt$で,これは$n \geq 1$について単調減少.
        また$|D(1) - 1!/e| = |0 - 1!/e| < 1/2$.

        $n=0$については$D(0)=1$なので不成立.}

    また,$n \geq 1$について \begin{equation*}
        D(n) = n D(n-1) + (-1)^n,
    \end{equation*}
    これを繰り返し使って \begin{align*}
        D(n) & = n D(n-1) - (-1)^{n-1}             \\
             & = n D(n-1) - (D(n-1) - (n-1)D(n-2)) \\
             & = (n-1)(D(n-1)+D(n-2)).
    \end{align*}
    $D(n)$の指数型母関数は \begin{equation*}
        \sum_{n \geq 0} D(n) \frac{x^n}{n!} = \frac{e^{-x}}{1-x}.
    \end{equation*}
\end{example}

\setcounter{theorem}{2}
\begin{example}
    多重集合$M_n = \{1^2, 2^2, \ldots, n^2\}$の順列のうち,どの$2$つの隣接する項も等しくないものの個数を$h(n)$とおく.$h(0) = 1$,$h(1) = 0$,$h(2) = 2$.

    順列$w$に対して,性質$P_i$を「$2$つの$i$が隣接すること」とする.求める値は$f_=(\emptyset) = h(n)$.

    $g(i) = f_\geq(T)$ ($i=\#T$)とおくと \begin{equation*}
        g(i) = \frac{(2n-i)!}{2^{n-i}}.
    \end{equation*}
    したがって, \begin{align*}
        h(n) &= \sum_{i=0}^n \binom{n}{i} (-1)^{n-i} (n+i)! 2^{-i} \\
        &= \mleft. \Delta^n (n+i)! 2^{-i} \mright|_{i=0}.
    \end{align*}
\end{example}

\begin{example}
    順列$w = a_1 a_2 \cdots a_n \in \fS_n$について\textbf{descent set} $D(w)$を \begin{equation*}
        D(w) = \{i : a_i > a_{i+1}\},
    \end{equation*}
    で定める.また, \begin{align*}
        \alpha_n(S) &= \#\{w \in \fS_n : D(w) \subseteq S\}, \\
        \beta_n(S) &= \#\{w \in \fS_n : D(w) = S\},
    \end{align*}
    とする.このとき \begin{align*}
        \alpha_n(S) &= \sum_{T \subseteq S} \beta_n(T), \\
        \beta_n(S) &= \sum_{T \subseteq S} (-1)^{\#(S-T)} \alpha_n(T).
    \end{align*}
    また,$S = \{s_1, \ldots, s_k\}_< \subseteq [n-1]$のとき \begin{align*}
        \alpha_n(S) &= \binom{n}{s_1,\: s_2-s_1,\: s_3-s_2,\: \ldots,\: n-s_k}, \\
        \beta_n(S) &= \sum_{1 \leq i_1 < i_2 < \cdots < i_j \leq k} (-1)^{k-j}
         \binom{n}{s_{i_1}, s_{i_2}-s_{i_1}, \ldots, n-s_{i_j}}.
    \end{align*}
    $s_0 = 0$,$s_{k+1} = n$とすると, \begin{equation*}
        \beta_n(S) = n! \det \mleft[ \frac{1}{(s_{j+1}-s_i)!} \mright]_{(i,j) \in [0,k] \times [0,k]}
    \end{equation*}
    と書き換えられる(ただし$i \geq j$について$(i,j)$-成分は$0$とする).さらに$n!$を各列にばらすと \begin{equation*}
        \beta_n(S) = \det \mleft[ \binom{n-s_i}{s_{j+1}-s_i} \mright],
    \end{equation*}
    が得られる.
\end{example}

\begin{example}
    $S = \{s_1, \ldots, s_k\} \subseteq [n-1]$について,
    \begin{equation*}
        \sum_{\substack{w \in \fS_n \\ D(w) \subseteq S}} q^{s(w)} = \qbinom{n}{s_1, s_2-s_1, \ldots, n-s_k}
    \end{equation*}
    が成り立つような$s(w)$を考える.実は$s(w)=\operatorname{inv}(w)$とすると上の等式が成り立つ. \begin{equation*}
        t_1 = s_1,\ t_2 = s_2 - s_1,\ \ldots,\ t_{k+1} = n - s_k,
    \end{equation*}
    とし, \begin{equation*}
        M = \{1^{t_1}, 2^{t_2}, \ldots, (k+1)^{t_{k+1}}\},
    \end{equation*}
    とおく.命題1.7.1より \begin{equation*}
        \sum_{u \in \fS(M)} q^{\operatorname{inv}(u)} = \qbinom{n}{t_1, t_2, \ldots, t_{k+1}}.
    \end{equation*}

    $u \in \fS(M)$について,$u$の要素に対して小さい順に番号を振りなおして,$\operatorname{inv}(u) = \operatorname{inv}(v)$となるように$v \in \fS_n$をとる(例:$u = 12112$ $\Rightarrow$ $v=14235$).$w = v^{-1}$とおく.

    このようにしてとれる$w \in \fS_n$は$D(w) \subseteq \{s_1,\ldots,s_k\}$を満たす.逆に,$w \in \fS_n$について$D(w) \subseteq \{s_1,\ldots,s_k\}$が成り立つなら,対応する$u \in \fS(M)$がただ一つ存在する.よって示された.

    \begin{equation*}
        \beta_n(S, q) = \sum_{\substack{w \in \fS_n \\ D(w) = S}} q^{\operatorname{inv}(w)},
    \end{equation*}
    とおくと,例2.2.4と同じ議論により \begin{align*}
        \beta_n(S, q) &= \pmb{(n)!} \det \mleft[ \frac{1}{\pmb{(s_{j+1}-s_i)!}} \mright] \\
        &= \det \mleft[ \qbinom{n-s_i}{s_{j+1}-s_i} \mright].
    \end{align*}
\end{example}

上の議論を一般化して,次の結果が得られる.

\begin{proposition}
    $S = \{P_1,\ldots,P_n\}$を性質の集合とする.また$h$を$\bbN$上の関数とし,$e$を$\bbN \times \bbN$上の関数であって$e(i,i) = 1$,$e(i,j) = 0$ ($j < i$)が成り立つものとする.

    各$T \subseteq S$について,$T = \{P_{s_1}, \ldots, P_{s_k}\}$ ($1 \leq s_1 < \cdots < s_k \leq n$) とおくと \begin{equation*}
        f_\leq(T) = h(n) e(s_0,s_1) e(s_1,s_2) \cdots e(s_k,s_{k+1}), \\
    \end{equation*}
    が成り立つとする(ただし$s_0 = 0$,$s_{k+1} = n$とする).このとき \begin{equation*}
        f_=(T) = h(n) \det \mleft[ e(s_i, s_{j+1}) \mright].
    \end{equation*}
\end{proposition}

\end{document}
