\documentclass[xelatex,ja=standard,a4paper,14pt,everyparhook=compat]{bxjsarticle}
\usepackage{amsmath, amssymb, amsthm}
\usepackage{enumitem, mleftright, mathtools, hyperref, bm}
\usepackage{caption, wrapfig}
\usepackage{multirow}

\DeclareMathOperator{\Fix}{Fix}
\newcommand{\wt}[1]{\widetilde{#1}}

\setenumerate{label=(\alph*)}

\usepackage{concmath}
\usepackage[OT1]{fontenc}
\setsansfont{CMU Concrete}
\setmainfont{CMU Concrete}
\setCJKmainfont{Noto Sans CJK JP}
\setCJKsansfont{Noto Sans CJK JP}

\newcommand{\bbN}{\mathbb{N}}
\newcommand{\bbZ}{\mathbb{Z}}
\newcommand{\bbC}{\mathbb{C}}
\newcommand{\bbP}{\mathbb{P}}
\newcommand{\fS}{\mathfrak{S}}
\newcommand{\inv}[1]{#1^{-1}}

\newcommand{\paren}[1]{\mleft(#1\mright)}
\newcommand{\norm}[1]{\mleft|#1\mright|}
\newcommand{\bbinom}[2]{\mleft(\binom{#1}{#2}\mright)}
\newcommand{\ifthen}[3]{\text{$#1$ ? $#2$ : $#3$}}
\newcommand{\bool}[1]{[#1]}
\newcommand{\dcup}{\mathbin{\dot\cup}}
\newcommand{\qbinom}[2]{\pmb{\binom{#1}{#2}}}

\theoremstyle{definition}
\newtheorem{theorem}{定理}[subsection]
\newtheorem*{theorem*}{定理}
\newtheorem{example}[theorem]{例}
\newtheorem{proposition}[theorem]{命題}
\newtheorem{corollary}[theorem]{系}
\renewcommand{\proofname}{\textup{証明}}

\begin{document}

\setcounter{section}{2}
\setcounter{subsection}{5}
\subsection{Involutions}

包除原理の特別な形 \begin{equation*}
    f_=(\emptyset) = \sum_Y (-1)^{\#Y} f_\geq(Y).
\end{equation*}
を考える.これを \begin{equation*}
    f_=(\emptyset) + \sum_{\#Y\text{ odd}} f_\geq(Y) = \sum_{\#Y\text{ even}} f_\geq(Y).
\end{equation*}
と書き換える.これを組合せ的解釈を用いて示してみる.

集合$A$の元が,性質($S$の元)を持ったり持たなかったりする状況を考える. \begin{align*}
    M  & = \{x \in A: \text{$x$は$S$のどの性質も持たない}\},                                   \\
    N  & = \{(x, Y, Z): \text{$x \in A$が持つ性質が$Z \supseteq Y$に一致し,$\#Y$は奇数}\},    \\
    N' & = \{(x', Y', Z'): \text{$x \in A$が持つ性質が$Z \supseteq Y$に一致し,$\#Y'$は偶数}\},
\end{align*}
とする.$S$に全順序を入れ,$\sigma: M \cup N \to N'$を次のように定める. \begin{align*}
     & \sigma(x) = (x, \emptyset, \emptyset) \quad \text{if $x \in M$}, \\
     & \sigma(x, Y, Z) = \begin{cases}
        (x, Y - i, Z)    & \text{if $\min Y = \min Z = i$}, \\
        (X, Y \cup i, Z) & \text{if $\min Z = i < \min Y$}.
    \end{cases}
\end{align*}
$\sigma$は全単射であり, \begin{equation*}
    \sigma^{-1}(x, Y, Z) = \begin{cases}
        x \in M          & \text{if $Y = Z = \emptyset$},                          \\
        (X, Y-i, Z)      & \text{if $Z \neq \emptyset$ and $\min Y = \min Z = i$}, \\
        (X, Y \cup i, Z) & \text{if $Z \neq \emptyset$ and $\min Z = i < \min Y$},
    \end{cases}
\end{equation*}
ただし$\min \emptyset = \infty$とする.

$x \in M$を$(x, \emptyset, \emptyset) \in N'$と同一視すれば,$\tau \coloneqq \sigma \cup \sigma^{-1} : N \cup N' \to N \cup N'$は以下を満たす. \begin{enumerate}
    \item $\tau$はinvolution ($\tau^2 = \mathrm{id}$).
    \item $\tau$の不動点は$(x, \emptyset, \emptyset)$ ($x \in M$).
    \item $(x,Y,Z)$が$\tau$の不動点でないとき,$\tau(x,Y,Z) = (x,Y',Z')$とおくと \begin{equation*}
              (-1)^{\#Y} + (-1)^{\#Y'} = 0.
          \end{equation*}
\end{enumerate}
これらを踏まえて,包除原理の式 \begin{align*}
    f_=(\emptyset) & = \sum_Y (-1)^{\#Y} f_\geq(Y)                                                   \\
                   & = \sum_Y (-1)^{\#Y} \#\{x : Z \coloneqq \{\text{$x$が持つ性質}\} \supseteq Y\},
\end{align*}
の右辺を見ると,$\tau$で移りあう元$(x,Y,Z)$同士が係数$(-1)^{\#Y}$によって打ち消され,不動点$(x,\emptyset,\emptyset)$ ($\{\text{$x$が持つ性質}\} = \emptyset$)のみが残っている.

一般化する.有限集合$X$が$2$つの集合のdisjoint union $X^+ \dcup X^-$で表されるとし,$X$のinvolution $\tau$が次の条件を満たすとする. \begin{enumerate}
    \item $\tau(x) = y$,$x \neq y$ならば,$x,y$のちょうど片方のみが$X^+$に属する.
    \item $\tau(x) = x$ならば$x \in X^+$.
\end{enumerate}
$X$の元を重み関数 \begin{equation*}
    w(x) = \begin{cases*}
        1  & if $x \in X^+$, \\
        -1 & if $x \in X^-$,
    \end{cases*}
\end{equation*}
で重みづけするとき, \begin{equation*}
    \#\Fix(\tau) = \sum_{x \in X} w(x),
\end{equation*}
が成り立つ.ただし$\Fix(\tau)$は$\tau$の不動点の集合.

ここで,さらに別の有限集合のdisjoint union $\wt X = \wt X^+ \cup \wt X^-$を考え,$\wt X$のinvolution $\wt \tau$が(a),(b)を満たしているとする.さらに,全単射$f : X \to \wt X$が符号を保つ,すなわち$f(X^+) = \wt X^+$,$f(X^-) = \wt X^-$と仮定する.このとき \begin{equation*}
    \#\Fix(\tau) = \#X^+ - \#X^- = \# \wt X^+ - \# \wt X^- = \#\Fix(\wt \tau).
\end{equation*}
このとき,$\Fix(\tau)$から$\Fix(\wt \tau)$への自然な全単射$g$を構成することができる.これを\textbf{involution principle}という.

$x \in \Fix(\tau)$とする.このとき, \begin{equation*}
    f(\tau f^{-1} \wt \tau f)^n (x) \in \Fix(\wt \tau),
\end{equation*}
を満たす非負整数$n$が存在する.そのような$n$について, \begin{equation*}
    g(x) = f(\tau f^{-1} \wt \tau f)^n (x),
\end{equation*}
と定める.この$g : \Fix(\tau) \to \Fix(\wt \tau)$は全単射である.

このinvolution principleについて,``sieve-equivalence''という変種がある.$2$つのdisjointな有限集合$X, \wt X$と$Y \subseteq X$,$\wt Y \subseteq \wt X$について,全単射$f: X \to \wt X$,$g: Y \to \wt Y$がとれたとする.このとき$\#(X-Y) = \#(\wt X - \wt Y)$.ここで全単射$h: (X-Y) \to (\wt X - \wt Y)$を構成する.

$x \in X - Y$をとる.このとき, \begin{equation*}
    f (g^{-1} f)^n (x) \in \wt X - \wt Y.
\end{equation*}
を満たす非負整数$n$が存在する.そのような$n$について, \begin{equation*}
    h(x) = f(g^{-1} f)^n (x),
\end{equation*}
と定める.この$h : (X-Y) \to (\wt X-\wt Y)$は全単射.

\begin{example}
    $1$を不動点とする順列$w \in \fS_n$全体の集合を$Y$とする.また,ちょうど$1$個のサイクルからなる順列$w \in \fS_n$全体の集合を$\wt Y$とする.このとき$\#(\fS_n - Y) = \#(\fS_n - \wt Y) = n! - (n-1)!$.

    $\fS_n - Y$と$\fS_n - \wt Y$の間の全単射$h$を構成する明らかな方法はなさそうだが,全単射$g: Y \to \wt Y$を構成するのは簡単である.$f : \fS_n \to \fS_n$を恒等写像とすれば,先ほどの方法で全単射$h : (\fS_n - Y) \to (\fS_n - \wt Y)$を構成できる.

    ただし,この場合に関しては$Y$と$\wt Y$がdisjointであることから, \begin{equation*}
        h(w) = \begin{cases*}
            w         & if $w \notin \wt Y$, \\
            g^{-1}(w) & if $w \in \wt Y$,
        \end{cases*}
    \end{equation*}
    としてやれば済む.
\end{example}

\end{document}
