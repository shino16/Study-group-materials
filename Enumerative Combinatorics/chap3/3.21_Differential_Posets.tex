\documentclass[aspectratio=98, 8pt, t]{beamer}

\usepackage{luatexja-fontspec}
% \usepackage[noto-otf]{luatexja-preset}
% \setmainfont[Ligatures=TeX]{Fira Serif}
\setsansfont[Ligatures=TeX, UprightFont={* Regular}, BoldFont={* Medium}]{Fira Sans}
\usefonttheme{professionalfonts}
% \usefonttheme[onlymath]{serif}
% \usepackage{firamath-otf}
% \usepackage{FiraSans}
\setmainjfont[UprightFont={*-Regular}, BoldFont={*-Bold}]{NotoSansCJKjp}
\setsansjfont[UprightFont={*-Regular}, BoldFont={*-Bold}]{NotoSansCJKjp}
\usetheme{Darmstadt}
\AtBeginSection[]{
  \begin{frame}
  \vfill
  \centering
  \begin{beamercolorbox}[sep=8pt,center,shadow=true,rounded=true]{title}
    \usebeamerfont{title}\insertsectionhead\par%
  \end{beamercolorbox}
  \vfill
  \end{frame}
}

\usepackage{amsmath, amssymb, amsthm}
\usepackage{mathtools}
\usepackage{bm}
\usepackage{graphicx, float}

\usepackage{mleftright}
\let\oldleft\left
\let\oldright\right
\renewcommand{\left}{\mleft}
\renewcommand{\right}{\mright}

\theoremstyle{definition}
\newtheorem{proposition}{Proposition}
\newtheorem{remark}{Remark}

\newcommand{\bbP}{\mathbb{P}}
\newcommand{\bbN}{\mathbb{N}}
\newcommand{\defby}{\overset{\mathrm{def}}{\iff}}

\newcommand{\aparen}[1]{\langle #1 \rangle}
\newcommand{\paren}[1]{\left( #1 \right)}
\newcommand{\wKP}{\widehat{KP}}

\title{3.21 Differential Posets}
\author{shino16}
\date{\today}

\begin{document}
\maketitle

\begin{frame}[c]{今回の内容}
  Hasse walk (Hasse図上のwalk)の数え上げを扱う.

  $r$-differential poset $P$を考え,$\wKP$上の連続写像$U,D$に関する代数的手法を適用する.
\end{frame}

\section{復習}

\begin{frame}{定義}
  \begin{definition}
    $r \in \bbP$について,\textbf{$r$-differential poset}は以下を満たすposet $P$ \begin{itemize}
      \item[(D1)] $\hat0 \in P$, locally finite, graded
      \item[(D2)] $t \in P$が$k$元を被覆$\iff$ $k+r$元が$t$を被覆
      \item[(D3)] $s,t \in P$ ($s \neq t$)について,$s,t$が同じ$j$元を被覆$\iff$同じ$j$元が$s,t$を被覆
    \end{itemize}
  \end{definition}
  \pause
  \begin{remark}
    (D3)において$j=0,1$.(証明) $u_1,u_2 \lessdot s,t$のとき(D3)より$v_1,v_2 \lessdot u_1,u_2$.無限降下法
  \end{remark}
  \pause
  \begin{remark}
    $r$-differential lattice $L$について,(D3) $\iff$ $L$がモジュラ束
  \end{remark}
  \pause
  \begin{example}
    ヤング束$J_f(\bbN \times \bbN) = \{I: \text{有限順序イデアル}\}$:$1$-differential poset
  \end{example}
\end{frame}

\begin{frame}{線形代数との繋がり}
  \begin{definition}
    $K$:体,$P$:posetについて, \begin{align*}
      \widehat{KP}
       & \coloneqq \{\text{$P$の元の線形結合($K$係数)}\}
      = \left\{ \sum_{t \in P} c_t t : c_t \in K \right\}                 \\
      KP
       & \coloneqq \{\text{$P$の\textbf{有限個の}元の線形結合($K$係数)}\}
    \end{align*}
  \end{definition}
  \pause
  \begin{definition}
    \begin{equation*}
      \text{$\phi\colon \wKP \to \wKP$が\textbf{連続} $\defby$ }\phi\left( \sum_t c_t t \right) = \sum_t c_t \phi(t)
    \end{equation*}
  \end{definition}
  \pause
  \begin{remark}
    $|P| = \infty$で$(\forall t)$ $\phi(t) = u$などは,$\phi\left( \sum_t t \right) = \sum_{t \in P} u$の右辺が定義できずNG.
  \end{remark}
\end{frame}

\begin{frame}{Up, Down}
  \begin{definition}
    (D1)を満たす$P$について \begin{align*}
      U(s) & \coloneqq \sum_{t \gtrdot s} t, \\
      D(s) & \coloneqq \sum_{t \lessdot s} t
    \end{align*}
  \end{definition}
  \pause
  \begin{proposition}[3.21.3]
    \begin{equation*}
      \text{$P$が$r$-differential} \iff DU-UD = rI
    \end{equation*}
    ここで$I : \wKP \to \wKP$は恒等写像.
  \end{proposition}
  ($\Longrightarrow$の証明) \begin{itemize}
    \pause
    \item (D2)より$[t]UD(t) = k$ $\implies$ $[t]DU(t) = k+r$.
          \pause
    \item (D3)より$s\neq t$について$[s]UD(t) = j$ $\implies$ $[s]DU(t) = j$.
  \end{itemize}
  \pause 逆も同様.
\end{frame}
\begin{frame}
  \begin{definition}
    $X \subseteq P$について,\begin{equation*}
      \bm{X} = \sum_{t \in X} t
    \end{equation*}
    例:$\bm{P} = \sum_{t \in P} t$
  \end{definition}
  \pause
  \begin{proposition}[3.21.4]
    $P$が$r$-differentialであるとき \begin{equation*}
      D\bm P = (U+r)\bm P \quad (= U\bm P + r\bm P)
    \end{equation*}
  \end{proposition}
  \pause
  (証明) \begin{align*}
    [t] U \bm P & = \#\{s : t \gtrdot s\} = \#(\text{$t$\textbf{が}被覆する元})  \\
    [t] D \bm P & = \#\{s : t \lessdot s\} = \#(\text{$t$\textbf{を}被覆する元})
  \end{align*}
  \pause
  \begin{align*}
    \text{(D2) $t \in P$が$k$元を被覆} & \iff \text{$k+r$元が$t$を被覆} \\
    [t] U\bm P = k                     & \iff [t]D\bm P = k+r
  \end{align*}
\end{frame}

\begin{frame}{双線型形式}
  \begin{definition}
    $\langle \bullet, \bullet \rangle\colon \wKP \times KP \to K$を次で定める: \begin{equation*}
      \left\langle \sum a_t t, \sum b_t t \right\rangle = \sum a_t b_t \qquad \text{(ドット積っぽいやつ)}
    \end{equation*}
    ※$\sum b_t t \in KP$は有限個の項しかないから,右辺の項も有限個.
  \end{definition}
  \pause
  例えば$[t]U \bm P = \aparen{U\bm P, t}$と書ける.

  \pause
  例:$e(t) = \#(\text{飽和鎖$\hat 0 = t_0 \lessdot t_1 \lessdot \cdots \lessdot t_n = t$})$,$P_n = \{t : \rho(t)=n\}$について \begin{equation*}
    e(t) = \aparen{U^n \hat0, t}
  \end{equation*}
  \begin{equation*}
    \sum_{t \in P_n} e(t) = \aparen{U^n \hat0, \bm{P_n}}
  \end{equation*}
  \begin{equation*}
    \sum_{t \in P_n} e(t)^2 = \aparen{D^n U^n \hat0, \hat 0}
  \end{equation*}
\end{frame}

\begin{frame}{作用素での表現}
  関係式$DU - UD = rI$を \begin{equation*}
    \paren{r \frac{d}{dz}} z - z \paren{r \frac{d}{dz}} = r
  \end{equation*}
  で表現.すなわち$U = z$,$D = r \frac{d}{dz}$.

  \pause \begin{example}
    $f(U)$:$U$のべき級数について,$r \frac{d}{dz} (f(z) \cdot g(z)) = r \frac{d}{dz}f(z) \cdot g(z) + f(z) \cdot r\frac{d}{dz} g(z)$より \begin{equation*}
      D f(U) = r \frac{df(U)}{dU} + f(U) D
    \end{equation*}
  \end{example}

  \pause 目標:$DU$を$UD$に交換していって,$U^i D^j$の形を作ろう

  \pause 理由:$D^j \hat 0 = \delta_{j0} \hat 0$
\end{frame}

\begin{frame}{作用素係数のべき級数}
  モチベーション:$U,D$を含む等式を示すため,$U,D$の式を係数に持つ母関数を考えたい
  \pause
  \begin{definition}
    $x$のべき級数($K[U,D]$係数) $F(x)$を考える.つまり \begin{equation*}
      F(x) = \sum_{n \geq 0} p_n(U,D) x^n \qquad \text{($p_n$は多項式)}
    \end{equation*}
    $F(x)$は$\wKP$に作用する: \begin{equation*}
      \left(\sum_n p_n(U,D) x^n\right) t = \left(\sum_n p_n(U,D) t\right) x^n
    \end{equation*}
  \end{definition}
  \pause
  例:$e^{Dx} = \sum_{n \geq 0} \frac{D^nx^n}{n!}$
  \begin{align*}
    e^{Dx}\hat0 & = \sum_{n \geq 0} \left(\frac{D^n}{n!} x^n\right) \hat0 \\
                & = \sum_{n \geq 0} \frac{D^n \hat0}{n!} x^n              \\
                & = \hat0 \qquad \text{($n=0$だけ残る)}
  \end{align*}
\end{frame}

\begin{frame}{$(U+D)^n$}
  \begin{theorem}[3.21.6 (a)]
    \begin{equation*}
      e^{(U+D)x} = e^{\frac{1}{2} rx^2 + Ux} e^{Dx}
    \end{equation*}
    ※$x$は不定元
  \end{theorem}
  左辺は$(U+D)^n$のEGF.
  \begin{equation*}
    e^{(U+D)x} = \sum_{n \geq 0} (U+D)^n \frac{x^n}{n!}
  \end{equation*}
  \pause
  (証明) $J(x) \coloneqq (\text{右辺})$について \begin{align*}
    J(0)              & = 1,                                                             \\
    \frac{d}{dx} J(x) & = (D+U) J(x) \qquad \text{(ここで$D = r \tfrac{d}{dU}$を使った)}
  \end{align*}
\end{frame}

\begin{frame}{$e^{Dx}f(U,y)$}
  \begin{theorem}[3.21.6 (b)]
    \only<1>{$f(U, y) = \text{$y$のべき級数($K[U]$係数)}$とする.すなわち \begin{equation*}
        f(U, y) = \sum_{n \geq 0} p_n(U) y^n \qquad \text{(各$p_n$は多項式)}
      \end{equation*}
      このとき,}\begin{equation*}
      e^{Dx} f(U, y) = f(U+rx, y) e^{Dx}
    \end{equation*}
  \end{theorem}
  \pause
  (証明) $U=z$,$D=r\frac{d}{dz}$とし,各辺を$g(z)$に作用させる. \pause
  % \begin{align*}
  %   f(z+rx) &= \sum_{n \geq 0} \frac{(rx)^n}{n!} \frac{d}{dz} f(z) \qquad \text{(Taylor展開)} \\
  %   &= e^{x\paren{r \frac{d}{dz}}} f(z)
  % \end{align*}
  % \pause これを$g(z)$に作用させると \begin{align*}
  %   e^{x\paren{r \frac{d}{dz}}} (f(z) \cdot g(z))
  %   &= f(z+rx) e^{x\paren{r \frac{d}{dz}}} g(z)
  % \end{align*}
  \begin{align*}
    e^{\paren{r\frac{d}{dz}}x} (f(z,y) \cdot g(z))
     & = \sum_{n \geq 0} \frac{r^n x^n}{n!} \frac{d^n}{dz^n} (f(z,y) \cdot g(z))                                                     \\
     & = \sum_{n \geq 0} \frac{r^n x^n}{n!} \sum_{k=0}^n {n \choose k} \frac{d^k}{dz^k} f(z,y) \cdot \frac{d^{n-k}}{dz^{n-k}} g(z)   \\
    f(z+rx, y) e^{\paren{r\frac{d}{dz}}x} g(z)
     & = \sum_{k \geq 0} \frac{(rx)^k}{k!} \frac{d^k}{dz^k} f(z, y) \cdot e^{\paren{r\frac{d}{dz}}x} g(z) \qquad \text{(Taylor展開)} \\
     & = \sum_{k,m \geq 0} \frac{(rx)^k}{k!} \frac{d^k}{dz^k} f(z, y) \cdot \frac{r^m x^m}{m!} \frac{d^m}{dz^m} g(z)                 \\
     & = \sum_{k,m \geq 0} \frac{r^{k+m} x^{k+m}}{(k+m)!} \frac{(k+m)!}{k!\,m!} \frac{d^k}{dz^k} f(z, y) \cdot \frac{d^m}{dz^m} g(z)
  \end{align*}
\end{frame}

\section{Hasse Walk}

\begin{frame}{定義}
  \begin{definition}
    $s$から$t$への長さ$\ell$の\textbf{Hasse walk}: \begin{equation*}
      s = t_0,\,t_1,\,\ldots,\,t_\ell = t
    \end{equation*}
    ただし$t_{i-1} \lessdot t_i$ or $t_{i-1} \gtrdot t_i$
  \end{definition}
\end{frame}

\begin{frame}{閉じたHasse walkの数え上げ}
  \begin{theorem}[3.21.7]
    $P$:$r$-differential,$\kappa_\ell = \#(\text{$\hat0$から$\hat0$への長さ$\ell$のHasse walk})$について, \begin{equation*}
      \kappa_{2n-1} = 0, \qquad \kappa_{2n} = (2n-1)!! r^n = 1\cdot3\cdot5\cdots(2n-1) r^n
    \end{equation*}
  \end{theorem}
  \pause
  (証明) $\kappa_n = \aparen{(U+D)^n \hat0, \hat0}$に注意.
  \pause
  \begin{align*}
    \sum_{n \geq 0} \kappa_n \frac{x^n}{n!}
    \only<3-7>{   & = \left\langle \sum_{n \geq 0} (U+D)^n \frac{x^n}{n!} \hat0, \hat0 \right\rangle \\}
    \only<4-7>{   & = \left\langle e^{(U+D)x} \hat0, \hat0 \right\rangle \\}
    \only<5-7>{   & = \left\langle e^{\frac{1}{2}rx^2 + Ux} e^{Dx} \hat0, \hat0 \right\rangle \\}
    \only<6-7>{   & = \left\langle e^{\frac{1}{2}rx^2 + Ux} \hat0, \hat0 \right\rangle \\}
    \uncover<7->{ & = e^{\frac{1}{2} rx^2} \qquad \text{($\aparen{\bullet,\bullet}$の定義)} \\}
    \only<9->{    & = \sum_{n \geq 0} \frac{r^n x^{2n}}{2^n n!} \\}
    \only<10->{   & = \sum_{n \geq 0} r^n (2n-1)!! \frac{x^{2n}}{(2n)!}}
  \end{align*}
  \only<10->{最後の等号は$\frac{1}{2^n n!} = \frac{1}{(2n)!!} = \frac{(2n-1)!!}{(2n)!}$を使った.}
\end{frame}

\begin{frame}{上がって下がるHasse walkの数え上げ}
  \begin{theorem}[3.21.8]
    $P$:$r$-differential,$P_n = \{t : \rho(t) = n\}$について,$\hat0$から$\hat0$への$n$回上がって$n$回下がるHasse walkの個数は \begin{equation*}
      \sum_{t \in P_n} e(t)^2 = r^n n!
    \end{equation*}
  \end{theorem}
  \pause
  (証明) \begin{align*}
    \sum_{n \geq 0} \paren{\sum_{t \in P_n} e(t)^2} \frac{x^{2n}}{(n!)^2}
    \only<2-4>{   & = \sum_{n \geq 0} \aparen{D^n U^n \hat0, \hat0} \frac{x^{2n}}{(n!)^2}                                \\ }
    \only<3-4>{   & = \sum_{n \geq 0} \left\langle \frac{D^nx^n}{n!} \frac{U^nx^n}{n!} \hat0, \hat0 \right\rangle        \\ }
    \uncover<4->{ & = \left\langle e^{Dx}e^{Ux} \hat0, \hat0 \right\rangle                                               \\
                  & \qquad \qquad (\text{$m \neq n$のとき$\aparen{D^m U^n \hat0, \hat0} = 0$}) \\}
    \only<6->{    & = \left\langle e^{(U+rx)x}e^{Dx} \hat0, \hat0 \right\rangle \\}
    \only<7->{    & = \left\langle e^{(U+rx)x} \hat0, \hat0 \right\rangle \\}
    \uncover<8->{ & = e^{rx^2} \\}
    \only<9->{    & = \sum_{n \geq 0} \frac{r^n x^{2n}}{n!} \\}
  \end{align*}
\end{frame}

\begin{frame}{ここまでの振り返り}
  \begin{enumerate}
    \item $\kappa_n = \aparen{(U+D)^n \hat0, \hat0}$,$\sum_{t \in P_n} e(t)^2 = \aparen{D^n U^n \hat0, \hat0}$といった関係式をEGFで表す
    \item $DU-UD = rI$を用いて,$D$を交換し$U$の右側に追いやる \begin{itemize}
            \item ここで$U,D$を作用素$z, r\frac{d}{dz}$で表現
          \end{itemize}
    \item $D\hat0 = 0$を用いて整理する
    \item $\kappa_n$,$\sum_{t \in P_n} e(t)^2$のかんたんな式を得る
  \end{enumerate}

  ここからは数え上げ対象のEGFを変形してかんたんに表していく.
\end{frame}

\begin{frame}{$D\bm P = (U+r)\bm P$の利用}
  \begin{theorem}[3.21.9]
    $P$:$r$-differentialについて \begin{align*}
      e^{Dx} \bm P        & = e^{rx + \frac{1}{2}rx^2 + Ux} \bm P, \\
      e^{(U+D)x} \bm P    & = e^{rx + rx^2 + 2Ux} \bm P,           \\
      e^{Dx} e^{Ux} \bm P & = e^{rx + \frac{3}{2}rx^2 + 2Ux} \bm P
    \end{align*}
  \end{theorem}
  \pause
  ($1$つ目の証明) $L(x) = e^{rx+\frac{1}{2}rx^2+Ux}$について, \begin{align*}
    L(0)       & = 1,                                                                                                  \\
    \only<3->{DL(x)\bm P}
    \only<3->{ & = (rxL(x) + L(x)D)\bm P \qquad \qquad (Df(U) = r \frac{df(U)}{dU} + f(U) D) \\}
    \only<4->{ & = (rxL(x) + L(x)(U + r))\bm P \qquad (D\bm P = (U+r)\bm P) \\}
    \only<5->{ & = (rx + U + r)L(x)\bm P \\}
    \only<6->{ & = \frac{d}{dx} L(x) \bm P}
  \end{align*}
\end{frame}

\begin{frame}{$D\bm P = (U+r)\bm P$の利用}
  \begin{theorem}[3.21.9]
    $P$:$r$-differentialについて \begin{align*}
      e^{Dx} \bm P        & = e^{rx + \frac{1}{2}rx^2 + Ux} \bm P, \\
      e^{(U+D)x} \bm P    & = e^{rx + rx^2 + 2Ux} \bm P,           \\
      e^{Dx} e^{Ux} \bm P & = e^{rx + \frac{3}{2}rx^2 + 2Ux} \bm P
    \end{align*}
  \end{theorem}
  ($2$つ目の証明) \begin{align*}
    e^{(U+D)x} \bm P & = e^{\frac{1}{2}rx^2+Ux} e^{Dx} \bm P                      \\
                     & = e^{\frac{1}{2}rx^2+Ux} e^{rx + \frac{1}{2}rx^2+Ux} \bm P \\
                     & = e^{rx+rx^2+2Ux} \bm P
  \end{align*}
  \pause
  ($3$つ目の証明) \begin{align*}
    e^{Dx} e^{Ux} \bm P & = e^{(U+rx)x} e^{Dx} \bm P                    \\
                        & = e^{(U+rx)x} e^{rx+\frac{1}{2}rx^2+Ux} \bm P \\
                        & = e^{rx+\frac{3}{2}rx^2+2Ux} \bm P
  \end{align*}
\end{frame}

\begin{frame}{$\alpha(0 \to n)$,$\delta_n$のEGF}
  \only<-2>{\begin{definition}
    \begin{align*}
      \alpha(0 \to n) & = \sum_{t \in P_n} e(t)                       \\
      \delta_n        & = \#(\text{$\hat0$起点,長さ$n$のHasse walk})
    \end{align*}
  \end{definition}}
  \pause
  \begin{theorem}[3.21.10]
    $P$:$r$-differentialについて \begin{align*}
      \sum_{n \geq 0} \alpha(0 \to n) \frac{x^n}{n!} & = e^{rx+\frac{1}{2}rx^2} \\
      \sum_{n \geq 0} \delta_n \frac{x^n}{n!}        & = e^{rx+rx^2}
    \end{align*}
  \end{theorem}
  \pause
  (証明) \only<4>{ \begin{align*}
    \sum_{n \geq 0} \alpha(0 \to n) \frac{x^n}{n!}
    &= \sum_{n \geq 0} \aparen{D^n \bm P, \hat0} \frac{x^n}{n!} \\
    &= \aparen{e^{Dx} \bm P, \hat0} \\
    &= \aparen{e^{rx+\frac{1}{2}rx^2+Ux} \bm P, \hat0} \\
    &= e^{rx+\frac{1}{2}rx^2}
  \end{align*}}
  \only<5->{\begin{align*}
    \sum_{n \geq 0} \delta_n \frac{x^n}{n!}
    &= \sum_{n \geq 0} \aparen{(U+D)^n \bm P, \hat0} \frac{x^n}{n!} \\
    &= \aparen{e^{(U+D)x} \bm P, \hat0} \\
    &= \aparen{e^{rx+rx^2+2Ux} \bm P, \hat0} \\
    &= e^{rx+rx^2}
  \end{align*}}
\end{frame}

\begin{frame}{順列との関係}
  $w \in \mathfrak{S}_n$について,$c(w) \coloneqq (\text{サイクルの個数})$
  \begin{proposition}
    \begin{equation*}
      \alpha(0 \to n) = \sum_{w^2 = 1} r^{c(w)}
    \end{equation*}
  \end{proposition}
  (証明) \pause $w^2 = 1$ $\iff$ $w$の各サイクルが長さ$\leq 2$
  \pause \begin{align*}
    \sum_{w^2=1} r^{c(w)}
    &= \sum_{2k \leq n} \frac{r^{n-k}}{k!} \binom{n}{2} \binom{n-2}{2} \cdots \binom{n-2k+2}{2} \\
    &= \sum_{2k \leq n} \frac{n! \, r^{n-k}}{k! \, 2^k \, (n-2k)!} \\
    &= n! \, [x^n] e^{rx} e^{\frac{1}{2} rx^2} \\
    &= \alpha(0 \to n)
  \end{align*}
  \pause $c_2(w) \coloneqq (\text{長さ$2$のサイクルの個数})$ \begin{align*}
    \sum_{w^2=1} r^{c(w)} 2^{c_2(w)}
    &= \sum_{2k \leq n} \frac{n! \, r^{n-k}}{k! \, (n-2k)!} \\
    &= n! \, [x^n] e^{rx} e^{rx^2} = \delta_n
  \end{align*}
\end{frame}

\begin{frame}{$\alpha(n \to n+k)$}
  $\alpha(0 \to n)$を一般化しよう

  \pause ランク母関数 \begin{equation*}
    F(P, q) \coloneqq \sum_{n \geq 0} (\#P_n) q^n
  \end{equation*}
\end{frame}

\begin{frame}{$\alpha(n \to n+k)$の母関数}
  \begin{theorem}[3.21.11]
    $P$:$r$-differential
    \begin{equation*}
      \sum_{n, k \geq 0} \alpha(n \to n+k) q^n \frac{x^k}{k!}
      = F(P, q) exp(\frac{rx}{1-q} + \frac{rx^2}{2(1-q^2)})
    \end{equation*}
  \end{theorem}
  (証明)(テクい) \pause
  $\gamma: \wKP \to K[[q]]$を次で定める:\begin{equation*}
    \gamma\paren{\sum_{t \in P} c_t t} = \sum_{t \in P} c_t q^{\rho(t)}
  \end{equation*}
\end{frame}

\end{document}
