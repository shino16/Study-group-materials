\documentclass[xelatex,ja=standard,a4paper,14pt,everyparhook=compat]{bxjsarticle}
\usepackage{amsmath, amssymb, amsthm}
\usepackage{mathtools, bm}
\usepackage{enumitem}
\setenumerate{label=(\alph*)}

% \usepackage{concmath}
% \usepackage[OT1]{fontenc}
\setsansfont{Segoe UI}
% \setmainfont{CMU Concrete}
% \usepackage{zxjatype}
\usepackage[noto-jp,oneweight]{zxjafont}
% \setCJKmainfont{Noto Serif JP}
% \setCJKsansfont{Noto Sans JP}

\usepackage{fancyhdr}
\pagestyle{fancy}
\lhead{\nouppercase{\leftmark}}
\rhead{\nouppercase{\rightmark}}
\renewcommand{\footrulewidth}{0.4pt}
\let\origtitle\title
\renewcommand{\title}[1]{\lfoot{#1}\origtitle{#1}}
\cfoot{}
\rfoot{\thepage}

\newcommand{\paren}[1]{\left(#1\right)}

\newcommand{\bbC}{\mathbb{C}}
\newcommand{\bbR}{\mathbb{R}}
\newcommand{\bbQ}{\mathbb{Q}}
\newcommand{\bbZ}{\mathbb{Z}}
\newcommand{\bbN}{\mathbb{N}}
\newcommand{\bbP}{\mathbb{P}}
\newcommand{\bbF}{\mathbb{F}}
\newcommand{\frakS}{\mathfrak{S}}
\newcommand{\mcA}{\mathcal{A}}
\newcommand{\mcB}{\mathcal{B}}
\newcommand{\umod}{{\bmod\:}}
\DeclareMathOperator{\inv}{inv}
\DeclareMathOperator{\conv}{conv}
\DeclareMathOperator{\image}{Im}
\DeclareMathOperator{\rank}{rank}
\DeclareMathOperator{\ess}{ess}
\DeclareMathOperator{\codim}{codim}

\theoremstyle{definition}
\newtheorem{theorem}{定理}
\newtheorem*{theorem*}{定理}
\newtheorem{lemma}{補題}
\newtheorem*{lemma*}{補題}
\newtheorem{example}[theorem]{例}
\newtheorem*{example*}{例}
\newtheorem{definition}[theorem]{定義}
\newtheorem*{definition*}{定義}
\newtheorem{proposition}[theorem]{命題}
\newtheorem*{proposition*}{命題}
\newtheorem{corollary}[theorem]{系}
\newtheorem*{corollary*}{系}
\newtheorem{problem}{問題}
\newtheorem*{answer}{解答}
\renewcommand{\proofname}{\textup{証明}}

\usepackage{tcolorbox}
\tcbuselibrary{breakable,skins,theorems}

\tcolorboxenvironment{definition}{
    coltitle = black,
    % colback = white,
    colframe = green!35!black,
    fonttitle = \bfseries,
    breakable = true
}

\tcolorboxenvironment{definition*}{
    coltitle = black,
    % colback = white,
    colframe = green!35!black,
    fonttitle = \bfseries,
    breakable = true
}

\tcolorboxenvironment{theorem}{
    coltitle = black,
    % colback = black!10!white,
    colframe = blue!35!black,
    fonttitle = \bfseries,
    breakable = true
}

\tcolorboxenvironment{theorem*}{
    coltitle = black,
    % colback = black!10!white,
    colframe = blue!35!black,
    fonttitle = \bfseries,
    breakable = true
}

\tcolorboxenvironment{proposition}{
    coltitle = black,
    % colback = black!10!white,
    colframe = blue!35!black,
    fonttitle = \bfseries,
    breakable = true
}

\tcolorboxenvironment{proposition*}{
    coltitle = black,
    % colback = black!10!white,
    colframe = blue!35!black,
    fonttitle = \bfseries,
    breakable = true
}

\tcolorboxenvironment{corollary}{
    coltitle = black,
    % colback = black!10!white,
    colframe = blue!35!black,
    fonttitle = \bfseries,
    breakable = true
}

\tcolorboxenvironment{corollary*}{
    coltitle = black,
    % colback = black!10!white,
    colframe = blue!35!black,
    fonttitle = \bfseries,
    breakable = true
}

\tcolorboxenvironment{lemma}{
    coltitle = black,
    % colback = black!10!white,
    colframe = green!35!black,
    fonttitle = \bfseries,
    breakable = true
}

\tcolorboxenvironment{example}{
    coltitle = black,
    colback = white,
    colframe = purple!35!black,
    fonttitle = \bfseries,
    breakable = true
}

\tcolorboxenvironment{example*}{
    coltitle = black,
    colback = white,
    colframe = purple!35!black,
    fonttitle = \bfseries,
    breakable = true
}

\tcolorboxenvironment{lemma*}{
    coltitle = black,
    % colback = black!10!white,
    colframe = gray!35!black,
    fonttitle = \bfseries,
    breakable = true
}

\tcolorboxenvironment{proof}{
    blanker,
    breakable,
    left=5mm,
    before skip=10pt,
    after skip=10pt,
    borderline west={1mm}{0pt}{black}
}

\tcolorboxenvironment{proof*}{
    blanker,
    breakable,
    left=5mm,
    before skip=10pt,
    after skip=10pt,
    borderline west={1mm}{0pt}{black}
}

\tcolorboxenvironment{problem}{
    coltitle = black,
    % colback = black!10!white,
    colframe = black!35!black,
    fonttitle = \bfseries,
    breakable = true
}

\tcolorboxenvironment{answer}{
    blanker,
    breakable,
    left=5mm,
    before skip=10pt,
    after skip=10pt,
    borderline west={1mm}{0pt}{black}
}

\title{3.11節の残り}
\author{shino16}
\date{\today}

\begin{document}

\maketitle

\tableofcontents

\newpage

\setcounter{section}{-1}
\section{復習}

\begin{definition}
    配置$\mcA$について, \begin{equation*}
        \text{Intersection poset $L(\mcA) \coloneqq \{\text{$0$個以上の超平面の非空な交わり}\}$}.
    \end{equation*}
    $\hat 0f = V$.$s \leq t \overset{\mathrm{def}}{\Longleftrightarrow} s \supseteq t$.
\end{definition}

\begin{definition}
    \begin{align*}
        \text{特性多項式 $\chi_\mcA(x)$ }
         & \coloneqq \sum_{t \in L(\mcA)} \mu(\hat0, t) x^{\dim(t)} \\
         & = \sum_{\substack{\mcB \subseteq \mcA                    \\ \text{central}}} (-1)^{\#\mcB} x^{n - \rank(\mcB)} && (\text{命題3.11.3}).
    \end{align*}
\end{definition}

\begin{proposition}[3.11.5, Deletion-Restriction]
    配置$\mcA$,$H_0 \in \mcA$について \begin{align*}
        \mcA'  & \coloneqq \mcA - \{H_0\},                                          \\
        \mcA'' & \coloneqq \mcA^{H_0} = \{H \cap H_0 \neq \emptyset : H \in \mcA\}.
    \end{align*}
    このとき \begin{equation*}
        \chi_A(x) = \chi_{\mcA'}(x) - \chi_{\mcA''}(x).
    \end{equation*}
\end{proposition}

\begin{definition}
    $V : \text{実線形空間}$について \begin{align*}
        r(\mcA) & \coloneqq \#(\text{領域}) = r(\mcA') + r(\mcA''),                             \\
        b(\mcA) & \coloneqq \#(\text{相対的に有界な領域}) \quad (\text{essentializeすると有界}) \\
                & = \begin{cases}
            b(\mcA') + b(\mcA'') & \text{if $\rank(\mcA) = \rank(\mcA')$,}     \\
            0                    & \text{if $\rank(\mcA) = \rank(\mcA') + 1$.}
        \end{cases}
    \end{align*}
\end{definition}

\begin{theorem}
    \begin{align*}
        r(\mcA) & = (-1)^n \chi_\mcA(-1),            \\
        b(\mcA) & = (-1)^{\rank(\mcA)} \chi_\mcA(1).
    \end{align*}
\end{theorem}

\section{General Positionと特性多項式}

\begin{definition}
    $\mcA$の超平面はgeneral positionにある

    $\overset{\mathrm{def}}{\Longleftrightarrow}$ 以下が同時に成立: \begin{enumerate}
        \item $\{H_1,\ldots,H_p\} \subseteq \mcA$, $p \leq n$ $\Longrightarrow$ $\dim(H_1 \cap \cdots \cap H_p) = n-p$,
        \item $\{H_1,\ldots,H_p\} \subseteq \mcA$, $p > n$ $\Longrightarrow$ $H_1 \cap \cdots \cap H_p = \emptyset$.
    \end{enumerate}
\end{definition}

\begin{proposition}
    $\mcA:\text{general position}$,$m = \#\mcA$について \begin{align*}
        \chi_\mcA(x) & = \sum_{k=0}^n (-1)^k \binom{m}{k} x^{n-k}                            \\
                     & = x^n - mx^{n-1} + \binom{m}{2}x^{n-2} - \cdots + (-1)^n\binom{m}{n}.
    \end{align*}
\end{proposition}
\begin{proof}
    方針:$L(\mcA)$の構造に注目.$\mu(\hat0,t)$をexplicitに求める.

    \begin{align*}
        \mcA & = \{H_1,\ldots,H_m\},                                           \\
        P    & = \{S \subseteq [m] : \#S \leq n\}\text{ を包含関係で順序付け}, \\
        \phi & : P \to L(\mcA), \quad S \mapsto \bigcap_{i \in S} H_i,
    \end{align*}
    とする\footnote{$P$はtruncated boolean algebraと呼ばれる.}.このとき \begin{enumerate}
        \item $\phi$は全射:明らか.
        \item $\phi$は単射: \begin{align*}
                                    & \phi(S) = \phi(S')                          \\
                  \Longrightarrow{} & \phi(S) = \phi(S \cup S')                   \\
                  \Longrightarrow{} & n - \#S = \dim(\phi(S)) = n - \#(S \cup S') \\
                  \Longrightarrow{} & S = S'.
              \end{align*}
        \item $\phi$と$\phi^{-1}$は順序を保つ: \begin{equation*}
                  S \subseteq S' \Longleftrightarrow \phi(S) \supseteq \phi(S').
              \end{equation*}
    \end{enumerate}
    ゆえに$L(\mcA) \cong P$.

    $[\hat0,S] \cong B_{\#S}$より$\mu(\hat0,S) = (-1)^{\#S}$,また$\dim(\phi(S)) = n - \#S$より \begin{align*}
        \chi_\mcA(x) & = \sum_{t \in L(\mcA)} \mu(\hat0,t) x^{\dim(t)} \\
                     & = \sum_{S \in P} (-1)^{\#S} x^{n-\#S}           \\
                     & = \sum_{k=0}^n (-1)^k \binom{m}{k} x^{n-k}.
    \end{align*}
\end{proof}

なお$V:\text{実線形空間}$について \begin{align*}
    r(\mcA) = (-1)^n \chi_\mcA(-1)
     & = \sum_{k=0}^n \binom{m}{k}                     \\
     & = 1 + m + \binom{m}{2} + \cdots + \binom{m}{n}.
\end{align*}
$m < n$のとき \begin{align*}
    b(\mcA) & = (-1)^m \chi_\mcA(1)                     \\
            & = (-1)^m \sum_{k=0}^m (-1)^k \binom{m}{k} \\
            & = \delta_{0m},
\end{align*}
$m \geq n$のとき$\rank(\mcA) = n$より \begin{align*}
    b(\mcA) & = (-1)^n \chi_\mcA(1)                  \\
            & = \sum_{k=0}^n (-1)^{n-k} \binom{m}{k} \\
    % & = (-1)^n \left[ 1 - m + \binom{m}{2} - \cdots + (-1)^n \binom{m}{n} \right] \\
            & = \binom{m-1}{n}.
\end{align*}
(注)最後の等号について: \begin{enumerate}
    \item 漸化式を用いる: \begin{align*}
                  & \binom{m}{n} - \binom{m}{n-1} + \binom{m}{n-2} - \cdots                                                \\
              ={} & \left[\binom{m-1}{n}+\binom{m-1}{n-1}\right] - \left[\binom{m-1}{n-1}+\binom{m-1}{n-2}\right] + \cdots \\
              ={}& \binom{m-1}{n}.
          \end{align*}
    \item FPS: \begin{equation*}
        (-1)^n [x^n] \frac{1}{1-x}(1-x)^m = \binom{m-1}{n}.
    \end{equation*}
\end{enumerate}

\newpage

\section{有限体法}

$\bbQ$上の配置$\mcA$の特性多項式を,有限体を使って計算しよう.

\subsection{$\bbQ$から$\bbF_q$への帰着}

\begin{enumerate}[label=(\arabic*)]
    \item $\mcA$全体を同時に整数倍して分母をはらう.
    \item 素ベキ$q$をとって全体の$\umod q$をとり,$\mcA_q$を得る.
\end{enumerate}

\begin{definition}
    $\mcA$は$\umod q$で\textbf{良い帰着}を持つ $\overset{\mathrm{def}}{\Longleftrightarrow}$ $L(\mcA) \cong L(\mcA_q)$.
\end{definition}

\begin{proposition}
    $\bbZ$上の配置$\mcA$について,$\mcA$が$\umod p$で良い帰着を持たないような素数$p$は有限個のみ.
\end{proposition}
\begin{proof}
    $\bbZ$上のアフィン空間$H_1,\ldots,H_j$ ($H_i : \alpha_i \cdot x = a_i$)について, \begin{align*}
        & H_1 \cap \cdots \cap H_j \neq \emptyset \\
        \Longleftrightarrow{}& \rank\begin{pmatrix}
            \alpha_1 & a_1 \\
            \vdots & \vdots \\
            \alpha_j & a_j
        \end{pmatrix} = \rank \begin{pmatrix}
            \alpha_1 \\
            \vdots \\
            \alpha_j
        \end{pmatrix}.
    \end{align*}
\end{proof}

\end{document}
