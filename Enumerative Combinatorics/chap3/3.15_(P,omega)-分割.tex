\documentclass[xelatex,ja=standard,a4paper,14pt,everyparhook=compat]{bxjsarticle}
\usepackage{amsmath, amssymb, amsthm}
\usepackage{mathtools, bm}
\usepackage{enumitem}
\usepackage{hyperref}
\setenumerate{label=(\alph*)}

\DeclareMathOperator{\ao}{ao}

% \usepackage{concmath}
% \usepackage[OT1]{fontenc}
\setsansfont{Segoe UI}
% \setmainfont{CMU Concrete}
% \usepackage{zxjatype}
\usepackage[noto-jp,oneweight]{zxjafont}
% \setCJKmainfont{Noto Serif JP}
% \setCJKsansfont{Noto Sans JP}

\usepackage{fancyhdr}
\pagestyle{fancy}
\lhead{\nouppercase{\leftmark}}
\rhead{\nouppercase{\rightmark}}
\renewcommand{\footrulewidth}{0.4pt}
\let\origtitle\title
\renewcommand{\title}[1]{\lfoot{#1}\origtitle{#1}}
\cfoot{}
\rfoot{\thepage}

\newcommand{\defby}{\overset{\mathrm{def}}{\iff}}

\newcommand{\bbC}{\mathbb{C}}
\newcommand{\bbR}{\mathbb{R}}
\newcommand{\bbQ}{\mathbb{Q}}
\newcommand{\bbZ}{\mathbb{Z}}
\newcommand{\bbN}{\mathbb{N}}
\newcommand{\bbP}{\mathbb{P}}
\newcommand{\bbF}{\mathbb{F}}
\newcommand{\mkS}{\mathfrak{S}}
\newcommand{\mkm}{\mathfrak{m}}
\newcommand{\mcA}{\mathcal{A}}
\newcommand{\mcB}{\mathcal{B}}
\newcommand{\mcC}{\mathcal{C}}
\newcommand{\mcH}{\mathcal{H}}
\newcommand{\mcL}{\mathcal{L}}
\newcommand{\mcS}{\mathcal{S}}
\newcommand{\umod}{{\bmod\:}}
\DeclareMathOperator{\inv}{inv}
\DeclareMathOperator{\conv}{conv}
\DeclareMathOperator{\image}{Im}
\DeclareMathOperator{\rank}{rank}
\DeclareMathOperator{\nullity}{null}
\DeclareMathOperator{\ess}{ess}
\DeclareMathOperator{\codim}{codim}

\theoremstyle{definition}
% \newtheorem{theorem}{定理}
\newtheorem*{theorem}{定理}
% \newtheorem{lemma}{補題}
\newtheorem*{lemma}{補題}
% \newtheorem{example}[theorem]{例}
\newtheorem*{example}{例}
% \newtheorem{definition}[theorem]{定義}
\newtheorem*{definition}{定義}
% \newtheorem{proposition}[theorem]{命題}
\newtheorem*{proposition}{命題}
% \newtheorem{corollary}[theorem]{系}
\newtheorem*{corollary}{系}
\newtheorem{problem}{問題}
\newtheorem*{answer}{解答}
\renewcommand{\proofname}{\textup{証明}}

\usepackage{tcolorbox}
\tcbuselibrary{breakable,skins,theorems}

\tcolorboxenvironment{definition}{
    coltitle = black,
    % colback = white,
    colframe = green!35!black,
    fonttitle = \bfseries,
    breakable = true
}

\tcolorboxenvironment{definition*}{
    coltitle = black,
    % colback = white,
    colframe = green!35!black,
    fonttitle = \bfseries,
    breakable = true
}

\tcolorboxenvironment{theorem}{
    coltitle = black,
    % colback = black!10!white,
    colframe = blue!35!black,
    fonttitle = \bfseries,
    breakable = true
}

\tcolorboxenvironment{theorem*}{
    coltitle = black,
    % colback = black!10!white,
    colframe = blue!35!black,
    fonttitle = \bfseries,
    breakable = true
}

\tcolorboxenvironment{proposition}{
    coltitle = black,
    % colback = black!10!white,
    colframe = blue!35!black,
    fonttitle = \bfseries,
    breakable = true
}

\tcolorboxenvironment{proposition*}{
    coltitle = black,
    % colback = black!10!white,
    colframe = blue!35!black,
    fonttitle = \bfseries,
    breakable = true
}

\tcolorboxenvironment{corollary}{
    coltitle = black,
    % colback = black!10!white,
    colframe = blue!35!black,
    fonttitle = \bfseries,
    breakable = true
}

\tcolorboxenvironment{corollary*}{
    coltitle = black,
    % colback = black!10!white,
    colframe = blue!35!black,
    fonttitle = \bfseries,
    breakable = true
}

\tcolorboxenvironment{lemma}{
    coltitle = black,
    % colback = black!10!white,
    colframe = green!35!black,
    fonttitle = \bfseries,
    breakable = true
}

\tcolorboxenvironment{example}{
    coltitle = black,
    colback = white,
    colframe = purple!35!black,
    fonttitle = \bfseries,
    breakable = true
}

\tcolorboxenvironment{example*}{
    coltitle = black,
    colback = white,
    colframe = purple!35!black,
    fonttitle = \bfseries,
    breakable = true
}

\tcolorboxenvironment{lemma*}{
    coltitle = black,
    % colback = black!10!white,
    colframe = gray!35!black,
    fonttitle = \bfseries,
    breakable = true
}

\tcolorboxenvironment{proof}{
    blanker,
    breakable,
    left=5mm,
    before skip=10pt,
    after skip=10pt,
    borderline west={1mm}{0pt}{black}
}

\tcolorboxenvironment{proof*}{
    blanker,
    breakable,
    left=5mm,
    before skip=10pt,
    after skip=10pt,
    borderline west={1mm}{0pt}{black}
}

\tcolorboxenvironment{problem}{
    coltitle = black,
    % colback = black!10!white,
    colframe = black!35!black,
    fonttitle = \bfseries,
    breakable = true
}

\tcolorboxenvironment{answer}{
    blanker,
    breakable,
    left=5mm,
    before skip=10pt,
    after skip=10pt,
    borderline west={1mm}{0pt}{black}
}

\title{3.11節の残り}
\author{shino16}
\date{\today}

\begin{document}

\maketitle

\tableofcontents

\newpage

\setcounter{section}{-1}
\section{復習}

\begin{definition}
    $P$:ランク$n$の有限階層的半順序集合,$\rho$:ランク関数,$S \subseteq [0,n]$について \begin{align*}
        P_S       & \coloneqq \{t \in P : \rho(t) \in S\},                   \\
        \alpha(S) & \coloneqq \#(\text{$P_S$の極大鎖}),                      \\
        \beta(S)  & \coloneqq \sum_{T \subseteq S} (-1)^{\#(S-T)} \alpha(T).
    \end{align*}
\end{definition}
\begin{definition}
    $P$:$\hat0,\hat1$を持つ有限階層的半順序集合.

    任意の区間$[s,t]$について,次を満たす極大鎖$s = t_0 \lessdot \cdots \lessdot t_\ell = t$が一意に存在するとき,$\lambda : \{(s,t) : s \lessdot t\} \to \bbZ$は\textbf{R-ラベリング}: \begin{equation*}
        \lambda(t_0,t_1) \leq \cdots \leq \lambda(t_{\ell-1}, t_\ell).
    \end{equation*}
    R-ラベリングが存在する半順序集合は\textbf{R-poset}.
\end{definition}
\begin{theorem}[3.14.2]
    $S \subseteq [n-1]$について, \begin{equation*}
        \beta(S) = \#\{\text{極大鎖$\mkm$} : D(\lambda(\mkm)) = S \}.
    \end{equation*}
    ただし$\mkm$:$\hat0=t_0 \lessdot \cdots \lessdot t_n = \hat1$について \begin{align*}
        \lambda(\mkm)    & \coloneqq (\lambda(t_0,t_1),\ldots,\lambda(t_{n-1},t_n)),      \\
        D(\lambda(\mkm)) & \coloneqq \{i : \lambda(t_{i-1},t_i) > \lambda(t_i,t_{i+1})\}.
    \end{align*}
\end{theorem}

\section{半モジュラ束のR-ラベリング}

$L$:有限半モジュラ束(階層的かつ$\rho(s) + \rho(t) \geq \rho(s \land t) + \rho(s \vee t)$).

$P = \{s \in L : \text{結び既約}\}$ ($s \neq \hat0$,$t,u < s$を用いて$s = t \vee u$と表せない).

$\omega: P \to [\#P]$:order-preservingな全単射.ここで \begin{align*}
    t_i          & = \omega^{-1}(i),                                  \\
    \lambda(s,t) & = \min\{i : s \vee t_i = t\} \quad (s \lessdot t),
\end{align*}
とすると$\lambda$は$L$のR-ラベリング.

\begin{proof}[概略]
    ($\lambda(\mkm)$が単調増加な極大鎖$\mkm$の存在)区間$[s,t]$の長さで帰納法.$s = t$のときはよい.
    \begin{align*}
        i & \coloneqq \min \{i : s < s \vee t_i \leq t\},                           \\
        w & \coloneqq \bigvee \{t_j : t_j < t_i\} \quad (\text{空のときは$\hat0$}),
    \end{align*}
    とすると,$w \lessdot t_i$.

    また各$t_j < t_i$について,$i$の最小性より$s = s \vee t_j$,したがって$s \geq w$.これより$s \wedge t_i = w$.
    \begin{align*}
        \rho(s) + \rho(t_i) & \geq \rho(s \vee t_i) + \rho(w)     \\
                            & = \rho(s \vee t_i) + \rho(t_i) - 1,
    \end{align*}
    より$\rho(s) + 1 \geq \rho(s \vee t_i)$.$s < s \vee t_i$より$s \lessdot (s \vee t_i)$.

    帰納法の仮定を使って$s \vee t_i$から$t$への極大鎖をとり,$s$をprepend.

    (極大鎖$\mkm$の一意性) $\lambda(\mkm), \lambda(\mkm')$がともに単調増加な極大鎖 \begin{align*}
        \mkm  & : s = s_0 \lessdot s_1 \lessdot \cdots \lessdot s_\ell = t,    \\
        \mkm' & : s = s_0' \lessdot s_1' \lessdot \cdots \lessdot s_\ell' = t,
    \end{align*}
    について,$i = \lambda(s_0, s_1) < \lambda(s_0', s_1')$を仮定する.

    $j = \min \{j : t_i \leq s_j'\} > 0$をとると$s_{j-1}' \vee t_i = s_j'$より$\lambda(s_{j-1}', s_j') \leq i$,矛盾.
\end{proof}

\section{$(P,\omega)$-分割}
\subsection{主要な母関数}

\begin{definition}
    $P$:半順序集合,$p = \#P$,全単射$\omega : P \to [p]$とする.

    \textbf{$(P,\omega)$-分割}$\sigma : P \to \bbN$は次を満たす: \begin{enumerate}
        \item $s \leq t$ $\implies$ $\sigma(s) \geq \sigma(t)$, and
        \item $s < t$ and $\omega(s) > \omega(t)$ $\implies$ $\sigma(s) > \sigma(t)$.
    \end{enumerate}
    $\sum_{t \in P} \sigma(t) = n$とすると,$\sigma$は$n$の$(P,\omega)$-分割.
\end{definition}

$s < t$ $\implies$ $\omega(s) < \omega(t)$のとき$\omega$は$P$の\textbf{自然なラベリング}.このとき条件(b)は無関係で,このときの$\sigma$は\textbf{$P$-分割}.

$s < t$ $\implies$ $\omega(s) > \omega(t)$のとき$\omega$は$P$の\textbf{双対自然なラベリング}.条件(a)で定まる順序関係が全てstrictになる.このときの$\sigma$は\textbf{狭義$P$-分割}.

$P = \{t_1,\ldots,t_p\}$とおく.$(P,\omega)$-分割たちに関連する基本母関数は \begin{equation*}
    F_{P,\omega} = F_{P,\omega}(x_1,\ldots,x_p) \coloneqq \sum_{\text{$\sigma$: $(P,\omega)$-分割}} x_1^{\sigma(t_1)} \cdots x_p^{\sigma(t_p)}.
\end{equation*}
$\omega$が自然なラベリングであるとき,$\omega$を省略して$F_P$と書く.

\begin{example}
    $t_1 < \cdots < t_p$からなる$P$と自然なラベリング$\omega$について, \begin{align*}
        F_P & = \sum_{a_1 \geq \cdots \geq a_p \geq 0} x_1^{a_1} x_2^{a_2} \cdots x_p^{a_p} \\
            & = \frac{1}{(1-x_1)(1-x_1x_2)\cdots(1-x_1x_2\cdots x_p)}.
    \end{align*}
\end{example}

\begin{example}
    $t_1 < \cdots < t_p$からなる$P$と双対自然なラベリング$\omega$について, \begin{align*}
        F_{P,\omega} & = \sum_{a_1 > \cdots > a_p \geq 0} x_1^{a_1} x_2^{a_2}\cdots x_p^{a_p}                  \\
                     & = \frac{x_1^{p-1}x_2^{p-2}\cdots x_{p-1}}{(1-x_1)(1-x_1x_2)\cdots(1-x_1x_2\cdots x_p)}.
    \end{align*}
\end{example}

\begin{example}
    $p$元反鎖$P$について \begin{align*}
        F_P & = \sum_{a_1,\ldots,a_p \geq 0} x_1^{a_1} x_2^{a_2} \cdots x_p^{a_p} \\
            & = \frac{1}{(1-x_1)(1-x_2)\cdots(1-x_p)}.
    \end{align*}
\end{example}

\begin{example}
    $t_1 < t_2$,$t_1 < t_3$,$t_2 \parallel t_3$からなる$P$と$\omega(t_1) = 2$,$\omega(t_2)=1$,$\omega(t_3)=3$について, \begin{equation*}
        F_{P,\omega} = \sum_{b < a \geq c} x_1^a x_2^b x_3^c.
    \end{equation*}
\end{example}

\begin{definition}
    ラベル付き半順序集合$(P,\omega)$について,\textbf{Jordan-H\"older集合} \begin{equation*}
        \mcL(P, \omega) \coloneqq \{\text{$P$のトポロジカルソートをラベルで並べたもの}\} \subseteq \mkS_p.
    \end{equation*}
\end{definition}

\begin{definition}
    \begin{equation*}
        \mcA(P,\omega) \coloneqq \{\sigma : \text{$P \to \bbN$, $(P,\omega)$-分割}\}.
    \end{equation*}
    $\sigma \in \mcA(P,\omega)$に対して,$\sigma' : [p] \to \bbN$を$\sigma'(\omega(t)) = \sigma(t)$で定める.
\end{definition}

\begin{definition}
    $w \in \mkS_n$について,次の条件を満たす$f : [n] \to \bbN$は$w$-compatible. \begin{itemize}
        \item $f(w_1) \geq \cdots \geq f(w_n)$, and
        \item $f(w_i) = f(w_{i+1})$ $\implies$ $w_i < w_{i+1}$.
    \end{itemize}
    $f$が$w$-compatibleとなる$w$は一意に存在.
\end{definition}

\begin{definition}
    \begin{equation*}
        S_w = \{\sigma : \text{$P \to \bbN$, $\sigma'$が$w$-compatible}\}.
    \end{equation*}
\end{definition}

\begin{lemma}[3.15.3]
    $\sigma : P \to \bbN$について, \begin{equation*}
        \text{$\sigma$は$(P,\omega)$-分割$\iff$ $\exists w \in \mcL(P,\omega)$ s.t. $\sigma'$が$w$-compatible.}
    \end{equation*}
    言い換えると, \begin{equation*}
        \mcA(P,\omega) = \bigcup_{w \in \mcL(P,\omega)} S_w \quad (\text{非交和}).
    \end{equation*}
\end{lemma}

\end{document}
