\documentclass[xelatex,ja=standard,a4paper,14pt,everyparhook=compat]{bxjsarticle}
\usepackage{amsmath, amssymb, amsthm}
\usepackage{mathtools, bm}

% \usepackage{concmath}
% \usepackage[OT1]{fontenc}
% \setsansfont{CMU Concrete}
% \setmainfont{CMU Concrete}
% \setCJKmainfont{Noto Sans JP Light}
% \setCJKsansfont{Noto Sans JP Light}

\newcommand{\paren}[1]{\left(#1\right)}

\newcommand{\bbC}{\mathbb{C}}
\newcommand{\bbN}{\mathbb{N}}
\newcommand{\bbP}{\mathbb{P}}
\newcommand{\bbF}{\mathbb{F}}
\newcommand{\frakS}{\mathfrak{S}}
\DeclareMathOperator{\inv}{inv}

\theoremstyle{definition}
\newtheorem{theorem}{定理}[subsection]
\newtheorem*{theorem*}{定理}
\newtheorem{example}[theorem]{例}
\newtheorem{proposition}[theorem]{命題}
\newtheorem{corollary}[theorem]{系}
\renewcommand{\proofname}{\textup{証明}}

\begin{document}

\subsection*{記法}

本の末尾``List of Notations''が便利.

$\bbN$:非負整数の集合

$\bbP$:正整数の集合

$\# S$:有限集合$S$の要素数

$S \subset T$:$S$は$T$の真部分集合($S \subsetneq T$)

$[n] = \{1,\ldots,n\}$

$\frakS_n$:$[n]$上の順列の集合

$\big( \binom{n}{k} \big)$:$[n]$の$k$-元部分多重集合の個数

$\binom{S}{k}$:$S$の$k$-元部分集合の集合

$\big( \binom{S}{k} \big)$:$S$の$k$-元部分多重集合の集合

\subsection*{形式的冪級数}

通常型母関数: \begin{equation*}
    \sum_{n \geq 0} f(n) x^n.
\end{equation*}

指数型母関数: \begin{equation*}
    \sum_{n \geq 0} f(n) \frac{x^n}{n!}.
\end{equation*}
これらは形式的冪級数(formal power series).

$F(x) = \sum_{n \geq 0} a_n x^n$について,$x^n$の係数は \begin{equation*}
    [x^n] F(x) = a_n.
\end{equation*}
また,形式的に$a_0 = F(0)$と表す.

和・積: \begin{align*}
    \paren{\sum_{n \geq 0} a_n x^n} + \paren{\sum_{n \geq 0} b_n x^n}
     & = \sum_{n \geq 0} (a_n + b_n) x^n,                                              \\
    \paren{\sum_{n \geq 0} a_n x^n} \paren{\sum_{n \geq 0} b_n x^n}
     & = \sum_{n \geq 0} \paren{\sum_{i=0}^n a_i b_{n-i}} x^n.
\end{align*}
指数型母関数については, \begin{align*}
    \paren{\sum_{n \geq 0} a_n \frac{x^n}{n!}} \paren{\sum_{n \geq 0} b_n \frac{x^n}{n!}}
    & = \sum_{n \geq 0} \paren{\sum_{i=0}^n \frac{a_i}{i!} \frac{b_{n-i}}{(n-i)!}} x^n \\
     & = \sum_{n \geq 0} \paren{\sum_{i=0}^n \binom{n}{i} a_i b_{n-i}} \frac{x^n}{n!}.
\end{align*}
これらは形式的冪級数環$\bbC[[x]]$をなす.

$F(x) G(x) = 1$であるとき,$G(x) = F(x)^{-1} = 1/F(x)$と表す.これは$F(0) \neq 0$ならば必ず存在する.

\setcounter{section}{1}
\setcounter{subsection}{1}
\setcounter{theorem}{4}
\begin{example}
    $\alpha \in \bbC \setminus \{0\}$について,
    \begin{equation*}
        (1-\alpha x)^{-1} = \sum_{n \geq 0} \alpha^n x^n.
    \end{equation*}
\end{example}

原則として,形式的冪級数を関数とみなしたときに成り立つ等式は,形式的冪級数の等式としても成立する.ただし式が形式的冪級数としてwell-definedである場合に限る.

\begin{example}
    \begin{equation*}
        F(x) = \sum_{n \geq 0} a_n x^n = \paren{\sum_{n \geq 0} \frac{x^n}{n!}} \paren{\sum_{n \geq 0} (-1)^n \frac{x^n}{n!}}
    \end{equation*}
    とおく.

    $F(x)$を(通常の意味での)関数とみなしてみる.任意の$x \in \bbC$について$F(x) = 1$が成り立つ($e^x e^{-x} = 1$)ので,係数比較により \begin{equation*}
        a_0 = 1, \qquad a_n = 0 \quad (n \geq 1).
    \end{equation*}
    したがって, \begin{equation*}
        F(x) = \sum_{n \geq 0} a_n x^n = 1
    \end{equation*}
    が形式的冪級数の等式としても成立する.
\end{example}

\begin{example}
    等式 \begin{equation*}
        \sum_{n \geq 0} \frac{(x+1)^n}{n!} = e \sum_{n \geq 0} \frac{x^n}{n!}
    \end{equation*}
    は関数論的には正しい($e^{x+1} = e \cdot e^x$)が,左辺は形式的冪級数として収束しないためNG.
\end{example}

$F(x) = \sum_{n \geq 0} a_n x^n$と$G(0) = 0$を満たす$G(x)$について, \begin{equation*}
    F(G(x)) = \sum_{n \geq 0} a_n G(x)^n.
\end{equation*}

$\lambda \in \bbC$について \begin{equation*}
    (1 + x)^\lambda = \sum_{n \geq 0} \binom{\lambda}{n} x^n.
\end{equation*}
ただし$\binom{\lambda}{n} = \frac{\lambda(\lambda-1)\cdots(\lambda-n+1)}{n!}$.

$F(x) = \sum_{n \geq 0} a_n x^n$について, \begin{equation*}
    F'(x) = \sum_{n \geq 0} n a_n x^{n-1} = \sum_{n \geq 0} (n+1) a_{n+1} x^n.
\end{equation*}

$\exp$, $\log$はテイラー展開に沿って定義する. \begin{align*}
    \exp x &= \sum_{n \geq 0} \frac{x^n}{n!}, \\
    \log (1+x) &= \sum_{n \geq 1} (-1)^{n-1} \frac{x^n}{n!}.
\end{align*}

\setcounter{theorem}{10}
\begin{example}
   $F(0) = 1$とし,次を満たす形式的冪級数$G(x)$を考える: \begin{equation*}
        G'(x) = \frac{F'(x)}{F(x)}, \qquad G(0) = 0.
    \end{equation*}
    このとき \begin{equation*}
        F(x) = \exp G(x).
    \end{equation*}
\end{example}

\begin{proof}[\textup{証明}]
    $F(x) = 1 + \sum_{n \geq 1} a_n x^n$,$\exp G(x) = \sum b_n x^n$とおく.各$b_n$は有限個の$a_i$の値によって決まる.$F(0)=1$より各$b_n$は有限個の$a_i$の多項式, \begin{equation*}
        b_n = p_n (a_1, \ldots, a_m)
    \end{equation*}
    として表せる.ここで$a_1, \ldots, a_m$を動かして十分小さくとれば,$F(x) = 1 + \sum_{n \geq 1} a_n x^n$が収束するような$0$の近傍$U$がとれる.$\forall x \in U$について \begin{equation*}
        F(x) = \exp G(x)
    \end{equation*}
    が成り立つので,\begin{equation*}
        a_n = b_n = p_n(a_1, \ldots, a_m).
    \end{equation*}
    多項式$p_n(a_1, \ldots, a_m)$と$a_n$が何らかの$0 \in \bbC^m$の近傍で一致することから,$b_n = p_n(a_1, \ldots, a_m)$は多項式として$a_n$に一致する.
\end{proof}

\begin{example}
   $a_0 = a_1 = 1$,$a_n = a_{n-1} + a_{n-2}$ ($n \geq 2$)について,母関数$F(x) = \sum_{n \geq 0} a_n x^n$の簡単な式を求める. \begin{align*}
        \sum_{n \geq 2} a_n x^n &= \sum_{n \geq 2} a_{n-1} x^n + \sum_{n \geq 2} a_{n-2} x^n \\
&= x \sum_{n \geq 2} a_{n-1} x^{n-1} + x^2 \sum_{n \geq 2} a_{n-2} x^{n-2}
    \end{align*}
    より, \begin{equation*}
        F(x) - (1 + x) = x(F(x) - 1) + x^2 F(x).
    \end{equation*}
    したがって \begin{equation*}
        F(x) = \frac{1}{1-x-x^2}.
    \end{equation*}
\end{example}

\begin{example}
   $a_0 = 1$,$a_{n+1} = a_n + n a_{n-1}$ ($n \geq 0$)について,母関数$F(x) = \sum_{n \geq 0} a_n x^n/n!$の簡単な式を求める. \begin{align*}
        \sum_{n \geq 0} a_{n+1} \frac{x^n}{n!}
         & = \sum_{n \geq 0} a_n \frac{x^n}{n!}
        + \sum_{n \geq 0} n a_{n-1} \frac{x^n}{n!} \\
         & = \sum_{n \geq 0} a_n \frac{x^n}{n!}
        + \sum_{n \geq 1} a_{n-1} \frac{x^n}{(n-1)!},
    \end{align*}
    すなわち \begin{equation*}
        F'(x) = F(x) + xF(x).
    \end{equation*}
    例1.1.11より,$F(x) = \exp\paren{x + \frac{1}{2} x^2}$.
\end{example}

\subsection*{$q$-類似}

``$n$の$q$-類似'': \begin{equation*}
    \pmb{(n)} = 1+q+\cdots+q^{n-1} = (1-q^n)/(1-q).
\end{equation*}

``$n!$の$q$-類似'': \begin{equation*}
    \pmb{(n)!} = \pmb{(1)(2)\cdots(n)} = (1+q)(1+q+q^2)\cdots(1+q+q^2+\cdots+q^{n-1}).
\end{equation*}

``$\binom{n}{k}$の$q$-類似'': \begin{equation*}
    \pmb{\binom{n}{k}} = \frac{\pmb{(n)!}}{\pmb{(k)!(n-k)!}}.
\end{equation*}

$q \to 1$のとき$\pmb{(n)} \to n$,$\pmb{(n)!} \to n!$,$\pmb{\binom{n}{k}} \to \binom{n}{k}$.

$n!$:$[n]$の部分集合の列$\emptyset = S_0 \subset S_1 \subset S_2 \subset \cdots \subset S_n = [n]$の個数.

$\pmb{(n)!}$:線形空間$\bbF_q^n$の部分空間の列$\{0\} = V_0 \subset V_1 \subset \cdots \subset V_n = \bbF_q^n$の個数.

$\binom{n}{k}$:$[n]$の$k$-元部分集合の個数.

$\pmb{\binom{n}{k}}$:$\bbF_q^n$の$k$次元部分空間の個数.

\setcounter{subsection}{3}
\setcounter{theorem}{12}
\begin{corollary}
    順列$w \in \frakS_n$の転倒数を$\inv(w)$で表すとき, \begin{equation*}
        \sum_{w \in \frakS_n} q^{\inv(w)} = \pmb{(n)!}.
    \end{equation*}
\end{corollary}

\end{document}
