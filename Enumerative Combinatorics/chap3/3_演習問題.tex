\documentclass[xelatex,ja=standard,a4paper,14pt,everyparhook=compat]{bxjsarticle}
\usepackage{amsmath, amssymb, amsthm}
\usepackage{mathtools, bm}
\usepackage{enumitem}
\setenumerate{label=\alph*.}
\usepackage{ascmac}

% \usepackage{concmath}
% \usepackage[OT1]{fontenc}
% \setsansfont{Computer Mathematics}
% \setmainfont{CMU Concrete}
% \setCJKmainfont{Noto Sans JP Light}
% \setCJKsansfont{Noto Sans JP Light}

\usepackage{fancyhdr}
\pagestyle{fancy}
\rhead{\leftmark}
\lhead{\rightmark}
\renewcommand{\footrulewidth}{0.4pt}
\let\origtitle\title
\renewcommand{\title}[1]{\lfoot{#1}\origtitle{#1}}
\cfoot{}
\rfoot{\thepage}

\newcommand{\paren}[1]{\left(#1\right)}

\newcommand{\bbC}{\mathbb{C}}
\newcommand{\bbN}{\mathbb{N}}
\newcommand{\bbP}{\mathbb{P}}
\newcommand{\bbF}{\mathbb{F}}
\newcommand{\frakS}{\mathfrak{S}}
\DeclareMathOperator{\inv}{inv}
\DeclareMathOperator{\conv}{conv}
\DeclareMathOperator{\image}{Im}

\theoremstyle{definition}
\newtheorem{theorem}{定理}
\newtheorem*{theorem*}{定理}
\newtheorem{lemma}{補題}
\newtheorem*{lemma*}{補題}
\newtheorem{example}[theorem]{例}
\newtheorem*{example*}{例}
\newtheorem{definition}[theorem]{定義}
\newtheorem*{definition*}{定義}
\newtheorem{proposition}[theorem]{命題}
\newtheorem*{proposition*}{命題}
\newtheorem{corollary}[theorem]{系}
\newtheorem{problem}{問題}
\newtheorem*{answer}{解答}
\renewcommand{\proofname}{\textup{証明}}

\usepackage{tcolorbox}
\tcbuselibrary{breakable,skins,theorems}

\tcolorboxenvironment{definition}{
    coltitle = black,
    % colback = white,
    colframe = green!35!black,
    fonttitle = \bfseries,
    breakable = true
}

\tcolorboxenvironment{definition*}{
    coltitle = black,
    % colback = white,
    colframe = green!35!black,
    fonttitle = \bfseries,
    breakable = true
}

\tcolorboxenvironment{theorem}{
    coltitle = black,
    % colback = black!10!white,
    colframe = blue!35!black,
    fonttitle = \bfseries,
    breakable = true
}

\tcolorboxenvironment{theorem*}{
    coltitle = black,
    % colback = black!10!white,
    colframe = blue!35!black,
    fonttitle = \bfseries,
    breakable = true
}

\tcolorboxenvironment{proposition}{
    coltitle = black,
    % colback = black!10!white,
    colframe = green!35!black,
    fonttitle = \bfseries,
    breakable = true
}

\tcolorboxenvironment{proposition*}{
    coltitle = black,
    % colback = black!10!white,
    colframe = green!35!black,
    fonttitle = \bfseries,
    breakable = true
}

\tcolorboxenvironment{lemma}{
    coltitle = black,
    % colback = black!10!white,
    colframe = gray!35!black,
    fonttitle = \bfseries,
    breakable = true
}

\tcolorboxenvironment{example}{
    coltitle = black,
    % colback = black!10!white,
    colframe = purple!35!black,
    fonttitle = \bfseries,
    breakable = true
}

\tcolorboxenvironment{example*}{
    coltitle = black,
    % colback = black!10!white,
    colframe = purple!35!black,
    fonttitle = \bfseries,
    breakable = true
}

\tcolorboxenvironment{lemma*}{
    coltitle = black,
    % colback = black!10!white,
    colframe = gray!35!black,
    fonttitle = \bfseries,
    breakable = true
}

\tcolorboxenvironment{proof}{
    blanker,
    breakable,
    left=5mm,
    before skip=10pt,
    after skip=10pt,
    borderline west={1mm}{0pt}{black}
}

\tcolorboxenvironment{proof*}{
    blanker,
    breakable,
    left=5mm,
    before skip=10pt,
    after skip=10pt,
    borderline west={1mm}{0pt}{black}
}

\tcolorboxenvironment{problem}{
    coltitle = black,
    % colback = black!10!white,
    colframe = black!35!black,
    fonttitle = \bfseries,
    breakable = true
}

\tcolorboxenvironment{answer}{
    blanker,
    breakable,
    left=5mm,
    before skip=10pt,
    after skip=10pt,
    borderline west={1mm}{0pt}{black}
}

\title{第3章 演習問題}
\author{shino16}
\date{\today}

\begin{document}
\maketitle

\newpage

\setcounter{problem}{85}
\begin{problem}
$L$:有限束,$f,g : L \to K$ (標数$0$の体)について \begin{equation*}
    f(s) = \sum_{t : s \land t = \hat 0} g(t),
\end{equation*}
とする.$\mu(\hat 0, u) \neq 0$ ($\forall u \in L$)であるとき, \begin{equation*}
    g(s) = \sum_t \alpha(s,t) f(t)
\end{equation*}
を示せ.ただし \begin{equation*}
    \alpha(s,t) = \sum_u \frac{\mu(s,u) \mu(t,u)}{\mu(\hat 0, u)}.
\end{equation*}
\end{problem}

\begin{answer}
    $G(s) = \sum_{s \leq t} g(t)$ ($G = \zeta g$)とする. \begin{align*}
        f(s) = \sum_{t : s \land t = \hat 0} g(t)
         & = \sum_t g(t) \sum_{u \leq s \land t} \mu(\hat 0, u)  \\
         & = \sum_{u \leq s} \mu(\hat 0, u) \sum_{u \leq t} g(t) \\
         & = \sum_{u \leq s} \mu(\hat 0, u) G(u).
    \end{align*}
    $f(s), \mu(\hat 0, u) G(u)$に対するM\"obius反転より \begin{align*}
        \mu(\hat 0, t) G(t) = \sum_{u \leq t} f(u) \mu(u, t).
    \end{align*}
    一方,$G(s), g(t)$に対するM\"obius反転より \begin{align*}
        g(s) & = \sum_{s \leq t} \mu(s, t) G(t)                                               \\
             & = \sum_{s \leq t} \frac{\mu(s,t)}{\mu(\hat0,t)} \sum_{u \leq t} f(u) \mu(u, t) \\
             & = \sum_u f(u) \sum_t \frac{\mu(s,t) \mu(u,t)}{\mu(\hat0,t)}.
    \end{align*}
\end{answer}

\begin{problem}
\begin{enumerate}
    \item $P$:$\hat0,\hat1$つきの有限poset,$f: P \to \bbC$.
          \begin{align*}
                  & \sum (f(t_1)-1) (f(t_2)-1) \cdots (f(t_k)-1)                                                   \\
              ={} & \sum (-1)^{k+1} \mu(\hat0, t_1) \mu(t_1,t_2) \cdots \mu(t_k,\hat1) f(t_1) f(t_2) \cdots f(t_k)
          \end{align*}
          を示せ.ただし和は全ての$k$と鎖$\hat 0 < t_1 < \cdots < t_k < \hat 1$にわたってとる.
    \item 次を示せ: \begin{equation*}
              \sum_{\hat0 = t_0 < t_1 < \cdots < t_k = \hat1} (-1)^k \mu(\hat0, t_1) \mu(t_1,t_2) \cdots \mu(t_{k-1},t_k) = 1.
          \end{equation*}
          \item (b)を隣接代数を用いて示せ.
          \item
\end{enumerate}
\end{problem}

\begin{answer}
    (a) ある鎖$C: \hat0 < t_1 < \cdots < t_k < \hat1$について,$f(t_1)\cdots f(t_k)$の寄与は \begin{align*}
            & \sum_{\text{鎖$C'$} \supseteq C} (-1)^{\#(C'-C)}                                                                            \\
        ={} & \sum_{\text{鎖$C'$} \supseteq C} (-1)^{\#(C' \cap (\hat0,t_1)) + \#(C' \cap (t_1,t_2)) + \cdots + \#(C' \cap (t_k,\hat1))}.
    \end{align*}
    \begin{itembox}[l]{命題3.8.5}
        有限poset $P$について,
        \begin{align*}
            \mu_{\widehat P}(\hat0, \hat1)
             & = \sum_{k \geq 0} (-(\zeta-1))^k (\hat0,\hat1) \\
             & = c_0 - c_1 + c_2 - c_3 + \cdots
        \end{align*}
        ただし$c_i \coloneqq (\text{$\hat0$から$\hat1$への長さ$i$の鎖の個数})$.
    \end{itembox}
    $\mu(t_i, t_{i+1}) = \sum_{C} (-1)^{\#C-1}$ ($C$は$t_i$から$t_{i+1}$への鎖)などから, \begin{align*}
            & \sum_{\text{鎖$C'$} \supseteq C} (-1)^{\#(C' \cap (\hat0,t_1)) + \#(C' \cap (t_1,t_2)) + \cdots + \#(C' \cap (t_k,\hat1))} \\
        ={} & (-1)^{k+1} \mu(\hat0, t_1) \mu(t_1,t_2) \cdots \mu(t_k,\hat1).
    \end{align*}

    (b) $f(t) = 0$ ($\forall t \in P$)として得られる.
\end{answer}

\end{document}