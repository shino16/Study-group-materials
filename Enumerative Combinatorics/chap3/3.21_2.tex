\documentclass[aspectratio=98, 8pt, t]{beamer}

\usepackage{luatexja-fontspec}
% \usepackage[noto-otf]{luatexja-preset}
% \setmainfont[Ligatures=TeX]{Fira Serif}
\setsansfont[Ligatures=TeX, UprightFont={* Regular}, BoldFont={* Medium}]{Fira Sans}
\usefonttheme{professionalfonts}
% \usefonttheme[onlymath]{serif}
% \usepackage{firamath-otf}
% \usepackage{FiraSans}
\setmainjfont[UprightFont={*-Regular}, BoldFont={*-Bold}]{NotoSansCJKjp}
\setsansjfont[UprightFont={*-Regular}, BoldFont={*-Bold}]{NotoSansCJKjp}
\usetheme{Darmstadt}
\AtBeginSection[]{
  \begin{frame}
  \vfill
  \centering
  \begin{beamercolorbox}[sep=8pt,center,shadow=true,rounded=true]{title}
    \usebeamerfont{title}\insertsectionhead\par%
  \end{beamercolorbox}
  \vfill
  \end{frame}
}

\usepackage{amsmath, amssymb, amsthm}
\usepackage{mathtools}
\usepackage{bm}
\usepackage{graphicx, float}

\usepackage{mleftright}
\let\oldleft\left
\let\oldright\right
\renewcommand{\left}{\mleft}
\renewcommand{\right}{\mright}

\theoremstyle{definition}
\newtheorem{proposition}{Proposition}
\newtheorem{remark}{Remark}

\newcommand{\bbP}{\mathbb{P}}
\newcommand{\bbN}{\mathbb{N}}
\newcommand{\defby}{\overset{\mathrm{def}}{\iff}}

\newcommand{\aparen}[1]{\langle #1 \rangle}
\newcommand{\paren}[1]{\left( #1 \right)}
\newcommand{\wKP}{\widehat{KP}}

\title{3.21 Differential Posets}
\author{shino16}
\date{\today}

\begin{document}
\maketitle

\section{復習}

\begin{frame}{$r$-differential posetの定義}
  \begin{definition}
    $r \in \bbP$について,\textbf{$r$-differential poset}は以下を満たすposet $P$ \begin{itemize}
      \item[(D1)] $\hat0 \in P$, locally finite, graded
      \item[(D2)] $t \in P$が$k$元を被覆$\iff$ $k+r$元が$t$を被覆
      \item[(D3)] $s,t \in P$ ($s \neq t$)について,$s,t$が同じ$j$元を被覆$\iff$同じ$j$元が$s,t$を被覆
    \end{itemize}
  \end{definition}
\end{frame}

\begin{frame}{線形代数}
  \begin{definition}
    $K$:体,$P$:posetについて, \begin{align*}
      \widehat{KP}
       & \coloneqq \{\text{$P$の元の線形結合($K$係数)}\}
      = \left\{ \sum_{t \in P} c_t t : c_t \in K \right\}    \\
      KP
       & \coloneqq \{\text{$P$の\textbf{有限個の}元の線形結合($K$係数)}\}
    \end{align*}
  \end{definition}
  \begin{definition}
    \begin{equation*}
      \text{$\phi\colon \wKP \to \wKP$が\textbf{連続} $\defby$ }\phi\left( \sum_t c_t t \right) = \sum_t c_t \phi(t)
    \end{equation*}
  \end{definition}
\end{frame}

% \begin{frame}{Up, Down}
%   \begin{definition}
%     (D1)を満たす$P$について \begin{align*}
%       U(s) & \coloneqq \sum_{t \gtrdot s} t, \\
%       D(s) & \coloneqq \sum_{t \lessdot s} t
%     \end{align*}
%   \end{definition}
%   \pause
%   \begin{proposition}[3.21.3]
%     \begin{equation*}
%       \text{$P$が$r$-differential} \iff DU-UD = rI
%     \end{equation*}
%     ここで$I : \wKP \to \wKP$は恒等写像.
%   \end{proposition}
%   ($\Longrightarrow$の証明) \begin{itemize}
%     \pause
%     \item (D2)より$[t]UD(t) = k$ $\implies$ $[t]DU(t) = k+r$.
%           \pause
%     \item (D3)より$s\neq t$について$[s]UD(t) = j$ $\implies$ $[s]DU(t) = j$.
%   \end{itemize}
%   \pause 逆も同様.
% \end{frame}
% \begin{frame}
%   \begin{definition}
%     $X \subseteq P$について,\begin{equation*}
%       \bm{X} = \sum_{t \in X} t
%     \end{equation*}
%     例:$\bm{P} = \sum_{t \in P} t$
%   \end{definition}
%   \pause
%   \begin{proposition}[3.21.4]
%     $P$が$r$-differentialであるとき \begin{equation*}
%       D\bm P = (U+r)\bm P \quad (= U\bm P + r\bm P)
%     \end{equation*}
%   \end{proposition}
%   \pause
%   (証明) \begin{align*}
%     [t] U \bm P & = \#\{s : t \gtrdot s\} = \#(\text{$t$\textbf{が}被覆する元})  \\
%     [t] D \bm P & = \#\{s : t \lessdot s\} = \#(\text{$t$\textbf{を}被覆する元})
%   \end{align*}
%   \pause
%   \begin{align*}
%     \text{(D2) $t \in P$が$k$元を被覆} & \iff \text{$k+r$元が$t$を被覆} \\
%     [t] U\bm P = k                     & \iff [t]D\bm P = k+r
%   \end{align*}
% \end{frame}

% \begin{frame}{双線型形式}
%   \begin{definition}
%     $\langle \bullet, \bullet \rangle\colon \wKP \times KP \to K$を次で定める: \begin{equation*}
%       \left\langle \sum a_t t, \sum b_t t \right\rangle = \sum a_t b_t \qquad \text{(ドット積っぽいやつ)}
%     \end{equation*}
%     ※$\sum b_t t \in KP$は有限個の項しかないから,右辺の項も有限個.
%   \end{definition}
%   \pause
%   例えば$[t]U \bm P = \aparen{U\bm P, t}$と書ける.

%   \pause
%   例:$e(t) = \#(\text{飽和鎖$\hat 0 = t_0 \lessdot t_1 \lessdot \cdots \lessdot t_n = t$})$,$P_n = \{t : \rho(t)=n\}$について \begin{equation*}
%     e(t) = \aparen{U^n \hat0, t}
%   \end{equation*}
%   \begin{equation*}
%     \sum_{t \in P_n} e(t) = \aparen{U^n \hat0, \bm{P_n}}
%   \end{equation*}
%   \begin{equation*}
%     \sum_{t \in P_n} e(t)^2 = \aparen{D^n U^n \hat0, \hat 0}
%   \end{equation*}
% \end{frame}

% \begin{frame}{作用素での表現}
%   関係式$DU - UD = rI$を \begin{equation*}
%     \paren{r \frac{d}{dz}} z - z \paren{r \frac{d}{dz}} = r
%   \end{equation*}
%   で表現.すなわち$U = z$,$D = r \frac{d}{dz}$.

%   \pause \begin{example}
%     $f(U)$:$U$のべき級数について,$r \frac{d}{dz} (f(z) \cdot g(z)) = r \frac{d}{dz}f(z) \cdot g(z) + f(z) \cdot r\frac{d}{dz} g(z)$より \begin{equation*}
%       D f(U) = r \frac{df(U)}{dU} + f(U) D
%     \end{equation*}
%   \end{example}

%   \pause 目標:$DU$を$UD$に交換していって,$U^i D^j$の形を作ろう

%   \pause 理由:$D^j \hat 0 = \delta_{j0} \hat 0$
% \end{frame}

% \begin{frame}{作用素係数のべき級数}
%   モチベーション:$U,D$を含む等式を示すため,$U,D$の式を係数に持つ母関数を考えたい
%   \pause
%   \begin{definition}
%     $x$のべき級数($K[U,D]$係数) $F(x)$を考える.つまり \begin{equation*}
%       F(x) = \sum_{n \geq 0} p_n(U,D) x^n \qquad \text{($p_n$は多項式)}
%     \end{equation*}
%     $F(x)$は$\wKP$に作用する: \begin{equation*}
%       \left(\sum_n p_n(U,D) x^n\right) t = \left(\sum_n p_n(U,D) t\right) x^n
%     \end{equation*}
%   \end{definition}
%   \pause
%   例:$e^{Dx} = \sum_{n \geq 0} \frac{D^nx^n}{n!}$
%   \begin{align*}
%     e^{Dx}\hat0 & = \sum_{n \geq 0} \left(\frac{D^n}{n!} x^n\right) \hat0 \\
%                 & = \sum_{n \geq 0} \frac{D^n \hat0}{n!} x^n              \\
%                 & = \hat0 \qquad \text{($n=0$だけ残る)}
%   \end{align*}
% \end{frame}

% \begin{frame}{$(U+D)^n$}
%   \begin{theorem}[3.21.6 (a)]
%     \begin{equation*}
%       e^{(U+D)x} = e^{\frac{1}{2} rx^2 + Ux} e^{Dx}
%     \end{equation*}
%     ※$x$は不定元
%   \end{theorem}
%   左辺は$(U+D)^n$のEGF.
%   \begin{equation*}
%     e^{(U+D)x} = \sum_{n \geq 0} (U+D)^n \frac{x^n}{n!}
%   \end{equation*}
%   \pause
%   (証明) $J(x) \coloneqq (\text{右辺})$について \begin{align*}
%     J(0)              & = 1,                                                             \\
%     \frac{d}{dx} J(x) & = (D+U) J(x) \qquad \text{(ここで$D = r \tfrac{d}{dU}$を使った)}
%   \end{align*}
% \end{frame}

% \begin{frame}{$e^{Dx}f(U,y)$}
%   \begin{theorem}[3.21.6 (b)]
%     \only<1>{$f(U, y) = \text{$y$のべき級数($K[U]$係数)}$とする.すなわち \begin{equation*}
%         f(U, y) = \sum_{n \geq 0} p_n(U) y^n \qquad \text{(各$p_n$は多項式)}
%       \end{equation*}
%       このとき,}\begin{equation*}
%       e^{Dx} f(U, y) = f(U+rx, y) e^{Dx}
%     \end{equation*}
%   \end{theorem}
%   \pause
%   (証明) $U=z$,$D=r\frac{d}{dz}$とし,各辺を$g(z)$に作用させる. \pause
%   % \begin{align*}
%   %   f(z+rx) &= \sum_{n \geq 0} \frac{(rx)^n}{n!} \frac{d}{dz} f(z) \qquad \text{(Taylor展開)} \\
%   %   &= e^{x\paren{r \frac{d}{dz}}} f(z)
%   % \end{align*}
%   % \pause これを$g(z)$に作用させると \begin{align*}
%   %   e^{x\paren{r \frac{d}{dz}}} (f(z) \cdot g(z))
%   %   &= f(z+rx) e^{x\paren{r \frac{d}{dz}}} g(z)
%   % \end{align*}
%   \begin{align*}
%     e^{\paren{r\frac{d}{dz}}x} (f(z,y) \cdot g(z))
%      & = \sum_{n \geq 0} \frac{r^n x^n}{n!} \frac{d^n}{dz^n} (f(z,y) \cdot g(z))                                                     \\
%      & = \sum_{n \geq 0} \frac{r^n x^n}{n!} \sum_{k=0}^n {n \choose k} \frac{d^k}{dz^k} f(z,y) \cdot \frac{d^{n-k}}{dz^{n-k}} g(z)   \\
%     f(z+rx, y) e^{\paren{r\frac{d}{dz}}x} g(z)
%      & = \sum_{k \geq 0} \frac{(rx)^k}{k!} \frac{d^k}{dz^k} f(z, y) \cdot e^{\paren{r\frac{d}{dz}}x} g(z) \qquad \text{(Taylor展開)} \\
%      & = \sum_{k,m \geq 0} \frac{(rx)^k}{k!} \frac{d^k}{dz^k} f(z, y) \cdot \frac{r^m x^m}{m!} \frac{d^m}{dz^m} g(z)                 \\
%      & = \sum_{k,m \geq 0} \frac{r^{k+m} x^{k+m}}{(k+m)!} \frac{(k+m)!}{k!\,m!} \frac{d^k}{dz^k} f(z, y) \cdot \frac{d^m}{dz^m} g(z)
%   \end{align*}
% \end{frame}

\section{Hasse Walk}

\begin{frame}{定義}
  \begin{definition}
    $s$から$t$への長さ$\ell$の\textbf{Hasse walk}: \begin{equation*}
      s = t_0,\,t_1,\,\ldots,\,t_\ell = t
    \end{equation*}
    ただし$t_{i-1} \lessdot t_i$ or $t_{i-1} \gtrdot t_i$
  \end{definition}
  % \begin{frame}{$D\bm P = (U+r)\bm P$の利用}
  %   \begin{theorem}[3.21.9]
  %     $P$:$r$-differentialについて \begin{align*}
  %       e^{Dx} \bm P        & = e^{rx + \frac{1}{2}rx^2 + Ux} \bm P, \\
  %       e^{(U+D)x} \bm P    & = e^{rx + rx^2 + 2Ux} \bm P,           \\
  %       e^{Dx} e^{Ux} \bm P & = e^{rx + \frac{3}{2}rx^2 + 2Ux} \bm P
  %     \end{align*}
  %   \end{theorem}
  %   ($2$つ目の証明) \begin{align*}
  %     e^{(U+D)x} \bm P & = e^{\frac{1}{2}rx^2+Ux} e^{Dx} \bm P                      \\
  %                      & = e^{\frac{1}{2}rx^2+Ux} e^{rx + \frac{1}{2}rx^2+Ux} \bm P \\
  %                      & = e^{rx+rx^2+2Ux} \bm P
  %   \end{align*}
  %   \pause
  %   ($3$つ目の証明) \begin{align*}
  %     e^{Dx} e^{Ux} \bm P & = e^{(U+rx)x} e^{Dx} \bm P                    \\
  %                         & = e^{(U+rx)x} e^{rx+\frac{1}{2}rx^2+Ux} \bm P \\
  %                         & = e^{rx+\frac{3}{2}rx^2+2Ux} \bm P
  %   \end{align*}
  % \end{frame}

  \pause
  \begin{definition}
    \begin{align*}
      e(t)            & = \#(\text{$\hat0$から$t$へ上に上がるだけのHasse walk}) \\
      \alpha(0 \to n) & = \sum_{t \in P_n} e(t)                      \\
      \delta_n        & = \#(\text{起点$\hat0$,長さ$n$のHasse walk})
    \end{align*}
  \end{definition}
\end{frame}

\begin{frame}{定理}
  \begin{theorem}[3.21.9]
    $P$:$r$-differentialについて \begin{align*}
      e^{Dx} \bm P        & = e^{rx + \frac{1}{2}rx^2 + Ux} \bm P, \\
      e^{(U+D)x} \bm P    & = e^{rx + rx^2 + 2Ux} \bm P,           \\
      e^{Dx} e^{Ux} \bm P & = e^{rx + \frac{3}{2}rx^2 + 2Ux} \bm P
    \end{align*}
  \end{theorem}
  \pause
  \begin{theorem}[3.21.10]
    $P$:$r$-differentialについて \begin{align*}
      \sum_{n \geq 0} \alpha(0 \to n) \frac{x^n}{n!} & = e^{rx+\frac{1}{2}rx^2} \\
      \sum_{n \geq 0} \delta_n \frac{x^n}{n!}        & = e^{rx+rx^2}
    \end{align*}
  \end{theorem}
\end{frame}

% \begin{frame}{順列との関係}
%   $w \in \mathfrak{S}_n$について,$c(w) \coloneqq (\text{サイクルの個数})$
%   \begin{proposition}
%     \begin{equation*}
%       \alpha(0 \to n) = \sum_{w^2 = 1} r^{c(w)}
%     \end{equation*}
%   \end{proposition}
%   (証明) \pause $w^2 = 1$ $\iff$ $w$の各サイクルが長さ$\leq 2$
%   \pause \begin{align*}
%     \sum_{w^2=1} r^{c(w)}
%     &= \sum_{2k \leq n} \frac{r^{n-k}}{k!} \binom{n}{2} \binom{n-2}{2} \cdots \binom{n-2k+2}{2} \\
%     &= \sum_{2k \leq n} \frac{n! \, r^{n-k}}{k! \, 2^k \, (n-2k)!} \\
%     &= n! \, [x^n] e^{rx} e^{\frac{1}{2} rx^2} \\
%     &= \alpha(0 \to n)
%   \end{align*}
%   \pause $c_2(w) \coloneqq (\text{長さ$2$のサイクルの個数})$ \begin{align*}
%     \sum_{w^2=1} r^{c(w)} 2^{c_2(w)}
%     &= \sum_{2k \leq n} \frac{n! \, r^{n-k}}{k! \, (n-2k)!} \\
%     &= n! \, [x^n] e^{rx} e^{rx^2} = \delta_n
%   \end{align*}
% \end{frame}

\begin{frame}{$\alpha(n \to n+k)$}
  $\alpha(0 \to n)$の一般化,$\alpha(n \to n+k)$を調べる

  \pause ${}$

  $k=0$のとき$\alpha(n \to n) = \#P_n$
  \begin{definition}
    Graded poset $P$のランク母関数 \begin{equation*}
      F(P, q) \coloneqq \sum_{n \geq 0} (\#P_n) q^n
    \end{equation*}
  \end{definition}
\end{frame}

\begin{frame}{$\alpha(n \to n+k)$の母関数}
  \begin{theorem}[3.21.11]
    $P$:$r$-differential
    \begin{equation*}
      \sum_{n, k \geq 0} \alpha(n \to n+k) q^n \frac{x^k}{k!}
      = F(P, q) \exp\left[\frac{rx}{1-q} + \frac{rx^2}{2(1-q^2)}\right]
    \end{equation*}
  \end{theorem}
  (証明) \pause
  $\gamma: \wKP \to K[[q]]$を次で定める:\begin{equation*}
    \gamma\paren{\sum_{t \in P} c_t t} = \sum_{t \in P} c_t q^{\rho(t)}
  \end{equation*}
  \pause
  \begin{align*}
    \alpha(n \to n+k)   & = \sum_{t \in P_n} \#(\text{$t$から$k$回上がるHasse walk})                      \\
                        & = [q^n] \gamma(D^k \bm P)                                                 \\
    \uncover<4->{G(q,x) & \coloneqq (\text{左辺}) = \sum_{k \geq 0} \gamma(D^k \bm P) \frac{x^k}{k!}}
  \end{align*}
\end{frame}

\begin{frame}{証明続き}
  \begin{align*}
    G(q,x) & = \sum_{k \geq 0} \gamma(D^k \bm P) \frac{x^k}{k!}       \\
           & = \gamma(e^{Dx} \bm P)                                   \\
           & = e^{rx+\frac{1}{2}rx^2} \gamma(e^{Ux} \bm P)            \\
           & = e^{rx+\frac{1}{2}rx^2} \sum_{n \geq 0} \sum_{k \geq 0}
    \alpha(n \to n+k) q^{n+k} \frac{x^k}{k!}                          \\
           & = e^{rx+\frac{1}{2}rx^2} \sum_{n \geq 0} \sum_{k \geq 0}
    \alpha(n \to n+k) q^n \frac{(qx)^k}{k!}                           \\
           & = e^{rx+\frac{1}{2}rx^2} G(q, qx)
  \end{align*}
  \pause
  $x \gets 0$に注目すると \begin{equation*}
    G(q,0) = F(P,q)
  \end{equation*}
\end{frame}

\begin{frame}{証明続き}
  わかったこと:
  \begin{itemize}
    \item $G(q,x) = e^{rx+\frac{1}{2}rx^2} G(q,qx)$
    \item $G(q,0) = F(P,q)$
  \end{itemize}
  $G(q,x) = F(p,q) \exp \left[ \frac{rx}{1-q} + \frac{rx^2}{2(1-q^2)} \right]$がこの方程式の一意解だと示せればOK.

  \pause
  \begin{equation*}
    G(q,x) = \sum_{k \geq 0} f_k(q) \, x^k \qquad (f_k(q) \in K[[q]])
  \end{equation*}
  とおく.$x \gets 0$に注目して$f_0(q) = F(P,q)$.
  \pause \begin{equation*}
    \sum_{k \geq 0} q_k(q) \, x^k = e^{rx+\frac{1}{2}rx^2} \sum_{k \geq 0} f_k(q) \, (qx)^k
  \end{equation*}
  \pause
  ここで$x^k$の係数を比較すると, \begin{equation*}
    f_k(q) = f_k(q) \, q^k + r f_{k-1}(q) \, q^{k-1} + \cdots + ({\cdots}) f_0(q)
  \end{equation*}
  の形になり,$f_k(q)$が$f_{k-1}(q),\ldots,f_0(q)$から一意に決まる.
\end{frame}

\end{document}
