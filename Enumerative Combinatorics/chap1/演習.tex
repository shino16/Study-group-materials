\documentclass[xelatex,ja=standard,a4paper,14pt,everyparhook=compat]{bxjsarticle}
\usepackage{amsmath, amssymb, amsthm}
\usepackage{mathtools, bm}
\usepackage{enumitem}
\setenumerate{label=\alph*.}
\usepackage{ascmac}

% \usepackage{concmath}
% \usepackage[OT1]{fontenc}
% \setsansfont{Computer Mathematics}
% \setmainfont{CMU Concrete}
% \setCJKmainfont{Noto Sans JP Light}
% \setCJKsansfont{Noto Sans JP Light}

\usepackage{fancyhdr, lastpage}
\pagestyle{fancy}
\rhead{\leftmark}
\lhead{\rightmark}
\renewcommand{\footrulewidth}{0.4pt}
\let\origtitle\title
\renewcommand{\title}[1]{\lfoot{#1}\origtitle{#1}}
\cfoot{}
\rfoot{\thepage/\pageref{LastPage}}

\newcommand{\paren}[1]{\left(#1\right)}

\newcommand{\bbC}{\mathbb{C}}
\newcommand{\bbN}{\mathbb{N}}
\newcommand{\bbP}{\mathbb{P}}
\newcommand{\bbF}{\mathbb{F}}
\newcommand{\frakS}{\mathfrak{S}}
\DeclareMathOperator{\inv}{inv}
\DeclareMathOperator{\conv}{conv}
\DeclareMathOperator{\image}{Im}

\theoremstyle{definition}
\newtheorem{theorem}{定理}
\newtheorem*{theorem*}{定理}
\newtheorem{lemma}{補題}
\newtheorem*{lemma*}{補題}
\newtheorem{example}[theorem]{例}
\newtheorem*{example*}{例}
\newtheorem{definition}[theorem]{定義}
\newtheorem*{definition*}{定義}
\newtheorem{proposition}[theorem]{命題}
\newtheorem*{proposition*}{命題}
\newtheorem{corollary}[theorem]{系}
\newtheorem*{problem}{問題}
\newtheorem*{answer}{解答}
\renewcommand{\proofname}{\textup{証明}}

\usepackage{tcolorbox}
\tcbuselibrary{breakable,skins,theorems}

\tcolorboxenvironment{definition}{
    coltitle = black,
    % colback = white,
    colframe = green!35!black,
    fonttitle = \bfseries,
    breakable = true
}

\tcolorboxenvironment{definition*}{
    coltitle = black,
    % colback = white,
    colframe = green!35!black,
    fonttitle = \bfseries,
    breakable = true
}

\tcolorboxenvironment{theorem}{
    coltitle = black,
    % colback = black!10!white,
    colframe = blue!35!black,
    fonttitle = \bfseries,
    breakable = true
}

\tcolorboxenvironment{theorem*}{
    coltitle = black,
    % colback = black!10!white,
    colframe = blue!35!black,
    fonttitle = \bfseries,
    breakable = true
}

\tcolorboxenvironment{proposition}{
    coltitle = black,
    % colback = black!10!white,
    colframe = green!35!black,
    fonttitle = \bfseries,
    breakable = true
}

\tcolorboxenvironment{proposition*}{
    coltitle = black,
    % colback = black!10!white,
    colframe = green!35!black,
    fonttitle = \bfseries,
    breakable = true
}

\tcolorboxenvironment{lemma}{
    coltitle = black,
    % colback = black!10!white,
    colframe = gray!35!black,
    fonttitle = \bfseries,
    breakable = true
}

\tcolorboxenvironment{example}{
    coltitle = black,
    % colback = black!10!white,
    colframe = purple!35!black,
    fonttitle = \bfseries,
    breakable = true
}

\tcolorboxenvironment{example*}{
    coltitle = black,
    % colback = black!10!white,
    colframe = purple!35!black,
    fonttitle = \bfseries,
    breakable = true
}

\tcolorboxenvironment{lemma*}{
    coltitle = black,
    % colback = black!10!white,
    colframe = gray!35!black,
    fonttitle = \bfseries,
    breakable = true
}

\tcolorboxenvironment{proof}{
    blanker,
    breakable,
    left=5mm,
    before skip=10pt,
    after skip=10pt,
    borderline west={1mm}{0pt}{black}
}

\tcolorboxenvironment{proof*}{
    blanker,
    breakable,
    left=5mm,
    before skip=10pt,
    after skip=10pt,
    borderline west={1mm}{0pt}{black}
}

\tcolorboxenvironment{problem}{
    coltitle = black,
    % colback = black!10!white,
    colframe = black!35!black,
    fonttitle = \bfseries,
    breakable = true
}

\tcolorboxenvironment{answer}{
    blanker,
    breakable,
    left=5mm,
    before skip=10pt,
    after skip=10pt,
    borderline west={1mm}{0pt}{black}
}

\title{第1章 演習問題}
\author{shino16}
\date{\today}

\begin{document}
\maketitle

\newpage

\begin{problem}[9]
$f(m,n) = (\text{$(0,0)$から$(m,n)$へ移動する経路の個数})$,ただし使える移動は$(1,0)$,$(0,1)$,$(1,1)$のみ.
\begin{enumerate}
    \item 次を示せ: \begin{equation*}
              \sum_{m \geq 0} \sum_{n \geq 0} f(m,n) x^n y^n = \frac{1}{1-x-y-xy}.
          \end{equation*}
    \item $\sum_{n \geq 0} f(n,n) x^n$を簡単に表せ.
\end{enumerate}
\end{problem}
\begin{answer}
    (a.) \begin{equation*}
        (\text{左辺}) = 1 + (x+y+xy) + (x+y+xy)^2 + \cdots.
    \end{equation*}

    (b.) $(0,1)$の移動を$k$回使うと,$(1,0)$は$k$回,$(1,1)$は$n-k$回.したがって \begin{align*}
        \sum_{n \geq 0} f(n,n) x^n
         & = \sum_{n \geq 0} x^n \sum_k \binom{n+k}{n-k,k,k}                 \\
         & = \sum_{n \geq 0} x^n \sum_k \binom{n+k}{2k} \binom{2k}{k}        \\
         & = \sum_k \binom{2k}{k} \sum_{n \geq 0} \binom{n+k}{2k} x^n        \\
         & = \sum_k \binom{2k}{k} x^k \sum_{n+k \geq 0} \binom{n+2k}{2k} x^n \\
         & = \sum_k \binom{2k}{k} \frac{x^k}{(1-x)^{2k+1}}.
    \end{align*}
    ここで$(1-4x)^{-1/2} = \sum_n \binom{2n}{n} x^n$を使うと, \begin{equation*}
        \sum_{n \geq 0} f(n,n) x^n = \frac{1}{1-x} \paren{1-\frac{4x}{(1-x)^2}}^{-1/2} = (1-6x+x^2)^{-1/2}.
    \end{equation*}
\end{answer}

\begin{problem}[10]
$f(n,r,s) = \#\left\{S \subseteq [2n] : \begin{aligned} & \text{奇数$r$個,偶数$s$個}, \\ & \text{どの$2$つの元も隣接しない} \end{aligned} \right\}$.

$f(n,r,s) = \binom{n-r}{s} \binom{n-s}{r}$を示せ.
\end{problem}
\begin{answer}
    DP (漸化式)で解けるので,$f(n,r,s)=\binom{n-r}{s}\binom{n-s}{r}$が初期条件と漸化式を満たすことを示せばOK.
\end{answer}
\begin{answer}
    数え上げの対象$S = \{a_1,\ldots,a_{r+s}\}_< \subseteq [2n]$について, \begin{equation*}
        \phi(S) \coloneqq \{\{a_1, a_2 - 2, a_3 - 4, \ldots, a_{r+s} - 2(r+s-1)\}\} \quad (\text{多重集合}).
    \end{equation*}
    $\phi(S)$は$[2n - 2(r+s-1)]$上の多重集合で,奇数$r$個,偶数$s$個.

    $\phi$はそのような多重集合たちへの全単射なので, \begin{align*}
        f(n,r,s) & = \paren{\binom{n-r-s+1}{r}} \paren{\binom{n-r-s+1}{s}} \\
                 & = \binom{n-s}{r} \binom{n-r}{s}.
    \end{align*}
\end{answer}

\begin{problem}[15]
以下の個数を求めよ.
\begin{enumerate}
    \item 多項式$(1+x+x^2)^n$の係数で,$3$の倍数でないもの.
\end{enumerate}
\end{problem}
\begin{answer}
    (a.) $\bbF_3$上で$1+x+x^2 = 1-2x+x^2 = (1-x)^2$.

    よって$\binom{2n}{k} \not\equiv 0 \pmod 3$なる$0 \leq k \leq 2n$を数えればよい.

    Lucasの定理より,$2n = \overline{a_1a_2\cdots a_r}_{(3)}$のとき答えは$(a_1+1)\cdots(a_r+1)$.
\end{answer}

\begin{problem}[33]
\begin{enumerate}
    \item $k, n \geq 1$について,次を満たす列$\emptyset = S_0,\ldots,S_k \subseteq [n]$はいくつあるか.\begin{enumerate}[label=(\roman*)]
              \item $S_{i-1} \subset S_i$ or $S_{i-1} \supset S_i$, and
              \item $|\#S_{i-1} - \#S_i| = 1$.
          \end{enumerate}
    \item 上に条件$S_k = \emptyset$を課したときの答え$f_k(n)$が \begin{equation*}
              f_k(n) = \frac{1}{2^n} \sum_i \binom{n}{i} (n-2i)^k
          \end{equation*}
          であることを示せ.
\end{enumerate}
\end{problem}
\begin{answer}
    (a.) $S_{i-1} \mathbin{\triangle} S_i = \{a_i\}$とすると,列$(a_1,\ldots,a_k)$は$\emptyset = S_0,\ldots,S_k$を一意に決める.よって$n^k$個.
\end{answer}

\begin{problem}[63]
\begin{enumerate}
    \item 大きさが相異なる$n$個の封筒が与えられる.封筒をより大きな封筒に入れることを$0$回以上繰り返して得られる配置はいくつあるか.
    \item a.のうち,他の封筒の中に入っていない封筒が$k$個あるものはいくつあるか.

          a.のうち,他の封筒が入っていない封筒が$k$個あるものはいくつあるか.
\end{enumerate}
\end{problem}

\begin{answer}
    封筒を大きい順に$1,\ldots,n$とラベル付けする.

    他の封筒に入っていないことを,「封筒$0$に入っている」と考える.

    (a.) 各$i=1,\ldots,n$について,封筒$i$を入れる先を$0,\ldots,i-1$から自由に選べる.よって$n!$個.

    (b.前半) 順列$w_1\cdots w_n \in \frakS_n$について,封筒の配置を次のように構成する: \begin{itemize}
        \item 各$i$について,$j = \max\{j < i : w_j < w_i\}$ (存在しなければ$j=0$)とし,封筒$i$を封筒$j$に入れる.
    \end{itemize}
    これは配置を全単射的に構成する.例:57316284

    このうちleft-to-right minimaが$k$個あるものを探せば良く,これは非交なサイクル$k$個からなる順列の個数$c(n, k)$ (符号なし第一種スターリング数).

    (b.後半)順列のうちdescentが$k-1$個しかないものを探せばよく,これはEulerian number $A(n,k)$.\footnote{命題1.4.4 \begin{equation*}
            \sum_{k=1}^n A(n,k) x^k = (1-x)^{n+1} \sum_{m \geq 0} m^n x^m
        \end{equation*}より,$n$を固定すると各$k \leq n$について同時に$O(n \log n)$時間で求まる.}
\end{answer}

\begin{problem}[64]
\begin{enumerate}
    \item \begin{equation*}
              f(n) = \left\{\begin{aligned}
                   & \text{正整数列$a_1,\ldots,a_n$} :                        \\
                   & \qquad \text{($\forall k>1$, $k \in \{a_1,\ldots,a_n\}$)} \\
                   & \qquad \text{\underline{最後の} $k$は$k-1$の後ろに現れる}
              \end{aligned} \right\}.
          \end{equation*}
          $f(n) = n!$を示せ.
    \item 上のうち$\max\{a_1,\ldots,a_n\} = k$を満たすものの個数が$A(n,k)$であることを示せ\footnote{$A(n,k) = \text{(非交なサイクル$k-1$個からなる順列$w\in\frakS_n$の個数)}$}.
\end{enumerate}
\end{problem}
ヒント:各$k$の最後の出現だけに注目
\begin{answer}
    数え上げ対象$(a_1,\ldots,a_n)$から$w \in \frakS_n$を全単射的に構成する: \begin{enumerate}[label=\arabic*.]
        \item $m_1 = (\text{$1$の出現回数})$とし,$1$を${}$\underline{右}から順に$1,2,\ldots,m_1$で置き換える.
        \item $m_2 = (\text{$2$の出現回数})$とし,元々の$2$を${}$\underline{右}から順に$m_1+1,m_1+2,\ldots,m_1+m_2$で置き換える.
        \item $3$以降も同様.
    \end{enumerate}
    例:13213312
\end{answer}

\begin{problem}[97]
    整数列$(a_1,\ldots,a_n)$で,$0 \leq a_i \leq 9$かつ総和が$4$の倍数であるものの個数を求めよ.
\end{problem}
\begin{answer}
    $f(x) = 1+x+\cdots+x^9$とすると,求める答えは$\sum_i [x^{4i}] f(x)^n$.これは \begin{align*}
        & \frac{1}{4} (f(1)^n + f(i)^n + f(-1)^n + f(-i)^n) \\
        ={}& \frac{1}{4} (10^n + (1+i)^n + (1-i)^n) \\
        ={}& \begin{cases*}
            \frac{1}{4}(10^n + (-1)^k 2^{2k+1}) & if $n = 4k$, \\
            \frac{1}{4}(10^n + (-1)^k 2^{2k+1}) & if $n = 4k+1$, \\
            \frac{1}{4}10^n & if $n = 4k+2$, \\
            \frac{1}{4}(10^n + (-1)^{k+1} 2^{2k+2}) & if $n = 4k+3$, \\
        \end{cases*}
    \end{align*}
\end{answer}

\begin{problem}[108]
    \begin{enumerate}
        \item $[n]$の分割であって,どの隣接する$2$元も同じブロックに属さないものの個数がベル数$B(n-1)$であることを示せ.
    \end{enumerate}
\end{problem}
\begin{answer}
    $[n-1]$の分割$\pi$をとる.同じブロックに$i,i+1,\ldots,j-1,j$ (極大)が属するとき,そのうち$i+1,i+3,\ldots$を$n$が属するブロックに移動させる.これを繰り返すと数え上げ対象の$[n]$の分割が得られ,この構成は全単射.
\end{answer}

\begin{problem}[109]
    \begin{enumerate}
        \item 順列$a_1 \cdots a_n \in \frakS_n$のうち,$a_i < a_j < a_{j+1}$なる$i < j$が存在しないものの個数が$B(n)$であることを示せ.
        \item $a_i < a_j < a_{j+1}$の代わりに$a_i < a_{j+1} < a_j$を考えても結果は同じであることを示せ.
        \item (a.) (b.)の条件をともに満たす順列の個数は,$\frakS_n$の対合の個数と等しいことを示せ\footnote{$f : X \to X$が対合 $\iff$ $f \circ f = \mathrm{id}$}.
    \end{enumerate}
\end{problem}
\begin{answer}
    (a.) $[n]$の分割$\pi$をとる.ブロックを最小元の降順に並べ,各ブロックの元を$(\text{最小元}),(\text{残りの元の降順})$に並べる.これらを結合して$a_1\cdots a_n \in \frakS_n$を得る.この構成は全単射.例:$13569|248|7$

    (b.) (a.) と同じことをする,ただし各ブロックの元は昇順に並べる.

    (c.) (a.)で得た順列が(b.)の条件を満たすには,各ブロックの要素数が$2$以下であることが必要十分.
\end{answer}

\end{document}
