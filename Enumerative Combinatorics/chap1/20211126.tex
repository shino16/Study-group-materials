\documentclass[xelatex,ja=standard,a4paper,14pt]{bxjsarticle}
\usepackage{amsmath, amssymb, amsthm}
\usepackage{enumerate, mleftright}

\usepackage{fancyhdr}
\pagestyle{fancy}
\fancyhead{}
\renewcommand{\headrulewidth}{0pt}
\fancyfoot[C]{}
\fancyfoot[R]{\thepage}


\usepackage{zxjatype}
\renewcommand*\familydefault{\sfdefault}
% なぜか Segoe UI で () が出力できない
\setsansfont[BoldFont={Segoe UI Bold}]{Segoe UI}
\setCJKsansfont{Noto Sans CJK JP}
\usepackage{concmath}

\newcommand{\bbC}{\mathbb{C}}

\newcommand{\paren}[1]{\mleft(#1\mright)}

\theoremstyle{definition}
\newtheorem{theorem}{定理}[subsection]
\newtheorem{example}[theorem]{例}
\newtheorem{proposition}[theorem]{命題}
\renewcommand{\proofname}{\textup{証明}}

\begin{document}

\section{What Is Enumerative Combinatorics?}
\subsection{How to Count}

$i$ が動くとき,有限集合 $S_i$ の要素数 $f(i)$ を数えるとする.$f(i)$ がどのような形で表されれば「数えた」といえるだろうか?

\begin{enumerate}[1.]
    \item 完全に陽な閉じた式で表され,よく知られた関数のみを含み,総和記号を含まない形.
    \item 漸化式.
    \item $f(i)$ を計算するある程度効率的なアルゴリズム.
    \item 漸近的な評価.
    \item 母関数.
\end{enumerate}

\subsubsection{母関数の扱いについて}

通常型母関数: \begin{equation*}
    \sum_{n \geq 0} f(n) x^n.
\end{equation*}
指数型母関数: \begin{equation*}
    \sum_{n \geq 0} f(n) \frac{x^n}{n!}.
\end{equation*}
これらは形式的べき級数.

$x$ が何かの値をとるわけではなく,$x^n$ や $x^n/n!$ は $f(n)$ が書かれる場所に過ぎない.

$F(x) = \sum_{n \geq 0} a_n x^n$ であるとき,$a_n$ は $F(x)$ における $x^n$ の\textbf{係数}といい, \begin{equation*}
    a_n = [x^n] F(x)
\end{equation*}
と表す.$F(x) = \sum_{n \geq 0} a_n x^n/n!$ に対しては, \begin{equation*}
    a_n = n! [x^n] F(x)
\end{equation*}
と表す.また,形式的に $a_0 = F(0)$ と表す.

複変数の母関数の例: \begin{equation*}
    \sum_{l \geq 0} \sum_{m \geq 0} \sum_{n \geq 0} f(l,m,n) \frac{x^ly^mz^n}{n!}.
\end{equation*}
($l$,$m$ について通常型,$n$ について指数型)

変数は無限個あってもよいが,どの項も無限個の変数を含んではならない.例:$x_1+x_2+x_3+\cdots$.

和・積は次に従う. \begin{align*}
    \paren{\sum_{n \geq 0} a_n x^n} + \paren{\sum_{n \geq 0} b_n x^n}
     & = \sum_{n \geq 0} (a_n + b_n) x^n,                                              \\
    \paren{\sum_{n \geq 0} a_n \frac{x^n}{n!}} + \paren{\sum_{n \geq 0} b_n \frac{x^n}{n!}}
     & = \sum_{n \geq 0} (a_n + b_n) \frac{x^n}{n!},                                   \\
    \paren{\sum_{n \geq 0} a_n x^n} \paren{\sum_{n \geq 0} b_n x^n}
     & = \sum_{n \geq 0} \paren{\sum_{i=0}^n a_i b_{n-i}} x^n,                         \\
    \paren{\sum_{n \geq 0} a_n \frac{x^n}{n!}} \paren{\sum_{n \geq 0} b_n \frac{x^n}{n!}}
     & = \sum_{n \geq 0} \paren{\sum_{i=0}^n \binom{n}{i} a_i b_{n-i}} \frac{x^n}{n!}.
\end{align*}
$\bbC$ 係数の形式的べき級数全体が成す環を $\bbC[[x]]$ と表す.$m$ 変数 $x_1,\ldots,x_m$ の場合は $\bbC[[x_1,\ldots,x_m]]$ と表す.これらは一意分解環 (unique factorization domain) をなす.

$F(x) G(x) = 1$ であるとき,$G(x) = F(x)^{-1}$ と表す.$F(x)^{-1}$ が存在するための必要十分条件は $F(0) \neq 0$ である.$F(x)^{-1} G(x) = G(x) / F(x)$ と表す.$(F(x)G(x))^{-1} = F(x)^{-1} G(x)^{-1}$,$(F(x)^{-1})^{-1} = F(x)$ が成り立つ(それぞれ $(\cdot)^{-1}$ が存在する場合).

\setcounter{theorem}{4}
\begin{example}
    $\alpha \in \bbC \setminus \{0\}$ とする.$\paren{\sum_{n \geq 0} \alpha^n x^n}(1 - \alpha x) = 1$ より, \begin{equation*}
        \sum_{n \geq 0} \alpha^n x^n = \frac{1}{1 - \alpha x}.
    \end{equation*}
\end{example}

原則として,形式的べき級数を関数とみなしたときに成り立つ等式は,形式的べき級数の等式としても成立する(式が形式的べき級数としても well defined である場合に限る).

\begin{example}
    等式 \begin{equation} \label{ex1.1.6}
        \paren{\sum_{n \geq 0} \frac{x^n}{n!}} \paren{\sum_{n \geq 0} (-1)^n \frac{x^n}{n!}} = 1
    \end{equation}
    は関数論的には正しい($e^x e^{-x} = 1$).すなわち,任意の $x \in \bbC$ について \begin{equation*}
        \sum_{n \geq 0} \paren{\sum_{k=0}^n (-1)^k \binom{n}{k}} \frac{x^n}{n!} = 1
    \end{equation*}
    が成り立つので,係数を比較して \begin{equation*}
        \sum_{k=0}^n (-1)^k \binom{n}{k} = \delta_{0n}
    \end{equation*}
    である.したがって式 (\ref{ex1.1.6}) は形式的べき級数に関する等式としても正しい.
\end{example}

\begin{example}
    等式 \begin{equation*}
        \sum_{n \geq 0} \frac{(x+1)^n}{n!} = e \sum_{n \geq 0} \frac{x^n}{n!}
    \end{equation*}
    は関数論的には正しい($e^{x+1} = e \cdot e^x$)が,左辺は $\bbC[[x]]$ の元でないため,形式的べき級数に関する主張にはならない.例えば左辺の定数項は $\sum_{n \geq 0} 1/n!$ であり,この級数は $\bbC[[x]]$ においては収束しない.
\end{example}

次の条件が成り立つとき,形式的べき級数の列 $F_1(x), F_2(x), \ldots$ が $F(x) = \sum_{n \geq 0} a_n x^n$ に\textbf{収束する}という: \begin{equation*}
    \forall n \geq 0,\ \exists \delta(n)\ \mathrm{s.t.}\ \forall i \geq \delta(n),\ [x^n] F_i(x) = a_n.
\end{equation*}
このことを $F_i(x) \to F(x)$ や $\lim_{i \to \infty} F_i(x) = F(x)$ と表す.

非零な形式的べき級数 $F(x) = \sum_{n \geq 0} a_n x^n$ の\textbf{次数} $\deg F(x)$ を \begin{equation*}
    \deg F(x) = \min \{n : a_n \neq 0\}
\end{equation*}
で定める($\deg 0 = \infty$ っぽい).$\deg F(x) G(x) = \deg F(x) + \deg G(x)$ に注意.これを用いると,$F_i(x)$ が収束する条件は \begin{equation*}
    \lim_{i \to \infty} \deg(F_{i+1}(x) - F_i(x)) = \infty
\end{equation*}
と表せる.また,$F_i(x)$ が $F(x)$ に収束する条件は \begin{equation*}
    \lim_{i \to \infty} \deg(F(x) - F_i(x)) = \infty
\end{equation*}
とも表せる.

\begin{proposition}
    無限級数 $\sum_{j \geq 0} F_j(x)$ が収束するための必要十分条件は \begin{equation*}
        \lim_{j \to \infty} \deg F_j(x) = \infty.
    \end{equation*}
\end{proposition}
\begin{proof}
    $\widetilde{F}_i(x) = \sum_{j=0}^i F_j(x)$ とすると,$\deg(\widetilde{F}_{i+1}(x) - \widetilde{F}_i(x)) = \deg F_{i+1}(x)$.
\end{proof}

\begin{proposition}
    無限積 $\prod_{j \geq 1} (1 + G_j(x))$ ($G_j(0) = 0$) が収束するための必要十分条件は, \begin{equation*}
        \lim_{j \to \infty} \deg G_j(x) = \infty.
    \end{equation*}
\end{proposition}
\begin{proof}
    $\widetilde{G}_i = \prod_{j=1}^i (1 + G_j(x))$ とすると, \begin{align*}
        \deg(\widetilde{G}_{i+1} - \widetilde{G}_i)
        % &= \deg(\widetilde{G}_{i+1} - \widetilde{G}_i) - \deg \widetilde{G}_{i} \\
         & = \deg \frac{\widetilde{G}_{i+1} - \widetilde{G}_i}{\widetilde{G}_{i}}
        \quad \text{($\widetilde{G}_i^{-1}$ の定数項は非零)}                      \\
         & = \deg \paren{\frac{\widetilde{G}_{i+1}}{\widetilde{G}_i} - 1}         \\
         & = \deg G_{i+1}(x).
    \end{align*}
\end{proof}

$F(x) = \sum_{n \geq 0} a_n x^n$ は $F_j(x) = a_j x^j$ の無限級数ともみなせる.

$F(x) = \sum_{n \geq 0} a_n x^n$ と $G(0) = 0$ を満たす $G(x)$ について,\textbf{合成} $F(G(x))$ を \begin{equation*}
    F(G(x)) = \sum_{n \geq 0} a_n G(x)^n
\end{equation*}
で定める.

\begin{example}
    $F(x) \in \bbC[[x]]$ が $F(0) = 0$ を満たすとき,$\lambda \in \bbC$ について \begin{equation*}
        (1 + F(x))^\lambda = \sum_{n \geq 0} \binom{\lambda}{n} F(x)^n
    \end{equation*}
    と定義する.ただし $\binom{\lambda}{n} = \lambda(\lambda-1)\cdots(\lambda-n+1)/n!$.
\end{example}

$F(x) = \sum_{n \geq 0} a_n x^n$ について,形式的導関数 $F'(x)$(または $\frac{dF}{dx}$ や $DF(x)$)を \begin{equation*}
    F'(x) = \sum_{n \geq 0} n a_n x^{n-1} = \sum_{n \geq 0} (n+1) a_{n+1} x^n
\end{equation*}
で定める.次の成立が確認できる: \begin{align*}
    (F+G)'   & = F' + G',        \\
    (FG)'    & = F'G + FG',      \\
    F(G(x))' & = G'(x) F'(G(x)).
\end{align*}

\begin{example}
    $F(0) = 1$ とし,次を満たす形式的べき級数 $G(x)$ をとる: \begin{equation*}
        G'(x) = \frac{F'(x)}{F(x)}, \qquad G(0) = 0.
    \end{equation*}
    関数論的にこの条件を $F(x)$ について解くと, \begin{equation} \label{ex1.1.11}
        F(x) = \exp G(x)
    \end{equation}
    が得られる.ただし \begin{equation*}
        \exp G(x) = \sum_{n \geq 0} \frac{G(x)^n}{n!}
    \end{equation*}
    と定める.式 (\ref{ex1.1.11}) は形式的べき級数の等式としても正しい.
\end{example}

\begin{proof}
    $F(x)$ の係数を動かすことを考える. \begin{align*}
        F(x)      & = 1 + \sum_{n \geq 1} a_n x^n, \\
        G(x)      & = \sum_{n \geq 1} b_n x^n,     \\
        \exp G(x) & = 1 + \sum_{n \geq 1} c_n x^n
    \end{align*}
    とおく.各 $b_n$ は有限個の $a_i$ の多項式なので,$c_n$ も有限個の $a_i$ の多項式である. \begin{equation*}
        c_n = p_n(a_1, a_2, \ldots, a_m)
    \end{equation*}
    とおく.各 $a_i$ の値を適当に定めると,原点の近傍 $U \subseteq \bbC$ であって $U$ 上で$1 + \sum_{n \geq 1} a_n x^n$ が収束するようなものがとれる.$U$ 上で $F(x) = \exp G(x)$ が成り立つので,これらの $a_i$ の値に対して $a_n = c_n = p_n(a_1,\ldots,a_m)$.

    したがって,$\bbC^m$ の近傍で $p_n(a_1,\ldots,a_m)$ が $a_n$ に一致するので,これらは多項式として等しい.よって任意の $a_i$ について $a_n = p_n(a_1,\ldots,a_m) = c_n$ であり,$F(G) = \exp G(x)$ が形式的べき級数の等式として成立する.
\end{proof}

\begin{example}
    $a_0 = a_1 = 1$,$a_n = a_{n-1} + a_{n-2}$ ($n \geq 2$) について,母関数 $F(x) = \sum_{n \geq 0} a_n x^n$ の簡単な式を求める. \begin{align*}
        F(x) & = 1 + x + \sum_{n \geq 2} a_n x^n                 \\
             & = 1 + x + \sum_{n \geq 2} (a_{n-1} + a_{n-2}) x^n \\
             & = 1 + x + x \sum_{n \geq 2} a_{n-1} x^{n-1}
        + x^2 \sum_{n \geq 2} a_{n-2} x^{n-2}                    \\
             & = 1 + x + x (F(x) - 1) + x^2 F(x)
    \end{align*}
    より,これを $F(x)$ について解いて \begin{equation*}
        F(x) = \frac{1}{1-x-x^2}
    \end{equation*}
    を得る.
\end{example}

\begin{example}
    $a_0 = 1$,$a_{n+1} = a_n + n a_{n-1}$ ($n \geq 0$) について,母関数 $F(x) = \sum_{n \geq 0} a_n x^n/n!$ の簡単な式を求める. \begin{align*}
        \sum_{n \geq 0} a_{n+1} \frac{x^n}{n!}
         & = \sum_{n \geq 0} a_n \frac{x^n}{n!}
        + \sum_{n \geq 0} n a_{n-1} \frac{x^n}{n!} \\
         & = \sum_{n \geq 0} a_n \frac{x^n}{n!}
        + \sum_{n \geq 1} a_{n-1} \frac{x^n}{(n-1)!},
    \end{align*}
    すなわち \begin{equation*}
        F'(x) = F(x) + xF(x)
    \end{equation*}
    である.これを例 1.1.11 と同様に解いて,$F(x) = \exp\paren{x + \frac{1}{2} x^2}$.
\end{example}

\end{document}
