\documentclass[xelatex,ja=standard,a4paper,14pt,everyparhook=compat]{bxjsarticle}
\usepackage{amsmath, amssymb, amsthm}
\usepackage{enumerate, mleftright, mathtools}
\usepackage{caption, wrapfig}

% \usepackage{fancyhdr}
% \pagestyle{fancy}
% \fancyhead{}
% \renewcommand{\headrulewidth}{0pt}
% \fancyfoot[C]{}
% \fancyfoot[R]{\thepage}


% \usepackage{zxjatype}
% \renewcommand*\familydefault{\sfdefault}
% % なぜか Segoe UI で () が出力できない
% \setmainfont[BoldFont={Segoe UI SemiBold}]{Segoe UI}
% \setsansfont[BoldFont={Segoe UI Bold}]{Segoe UI}
% \setCJKmainfont{Noto Sans CJK JP}
% \setCJKsansfont{Noto Sans CJK JP}
% \usepackage{concmath}


\usepackage{zxjatype}
% なぜか Segoe UI で () が出力できない
% \setmainfont[BoldFont=* Bold, ItalicFont=* Italic, BoldItalicFont=* BoldItalic]{/usr/local/texlive/texmf-local/fonts/truetype/CMUConcrete/CMU Concrete.ttf}
\usepackage{concmath}
\usepackage[OT1]{fontenc}
\setsansfont{CMU Concrete}
\setmainfont{CMU Concrete}
\setCJKmainfont{Noto Sans CJK JP}
\setCJKsansfont{Noto Sans CJK JP}
% \newfontfamily\bfsegoe{CMU Concrete Roman}
% \renewcommand*\bfdefault{\bfsegoe}


\newcommand{\bbC}{\mathbb{C}}
\newcommand{\fS}{\mathfrak{S}}
\DeclareMathOperator*{\inv}{inv}

\newcommand{\paren}[1]{\mleft(#1\mright)}

\theoremstyle{definition}
\newtheorem{theorem}{定理}[subsection]
\newtheorem*{theorem*}{定理}
\newtheorem{example}[theorem]{例}
\newtheorem{proposition}[theorem]{命題}
\renewcommand{\proofname}{\textup{証明}}

\begin{document}

\section{What is Enumerative Combinatorics?}
\setcounter{subsection}{5}
\subsection{Alternating Permutations, Euler Numbers, and the $cd$-Index of $\fS_n$}

\subsubsection{Basic Properties}

Alternating permutations $w \in \fS_n$の個数$E_n$をオイラー数と呼ぶ.


\begin{proposition}
    \begin{equation*}
        \sum_{n \geq 0} E_n \frac{x^n}{n!} = \sec x + \tan x.
    \end{equation*}
\end{proposition}

なお$\sum_{n \geq 0} E_n \frac{(-x)^n}{n!} = \sec x - \tan x$より, \begin{align*}
    \sum_{n \geq 0} E_{2n} \frac{x^{2n}}{(2n)!} &= \sec x, \\
    \sum_{n \geq 0} E_{2n+1} \frac{x^{2n+1}}{(2n+1)!} &= \tan x.
\end{align*}
$E_{2n}$はsecant number,$E_{2n+1}$はtangent numberとも呼ばれる.

\begin{proof}
    $0 \leq k \leq n$とする.$S \in \binom{[n]}{k}$をとり,$\overline S = [n] - S$とする.$S$のreverse alternating permutation $u$と,$\overline S$のreverse alternating permutation $v$をとる.$\operatorname{rev}(u), n+1, v$の連結を$w$とする.以上の手順ですべてのalternationg permutationとreverse alternating permutation $w \in \fS_{n+1}$をちょうど$1$回ずつ生成できる.したがって,$n \geq 1$について \begin{equation*}
        2E_{n+1} = \sum_{k=0}^n \binom{n}{k} E_k E_{n-k}.
    \end{equation*}
    $y = \sum_{n \geq 0} E_n x^n / n!$とする.上の等式を$x^n/n!$倍して足し上げると,微分方程式 \begin{equation*}
        2y' = y^2 + 1,
    \end{equation*}
    が導かれる.初期条件$y(0) = E_0 = 1$により,$y = \sec x + \tan x$が唯一解.
\end{proof}

等式 \begin{equation*}
    \sum_{n \geq 0} E_{2n} \frac{x^{2n}}{(2n)!} = (\cos x)^{-1}
    = \paren{\sum_{n \geq 0} (-1)^n \frac{x^{2n}}{(2n)!}}^{-1},
\end{equation*}
は, \begin{equation*}
    \sum_{n \geq 0} \frac{x^n}{n!} = e^x = (e^{-x})^{-1} = \paren{\sum_{n \geq 0} (-1)^n \frac{x^n}{n!}}^{-1},
\end{equation*}
とあわせて次のように一般化できる.

\begin{theorem*}
    \begin{equation*}
        f_k(n) = \# \{w \in \fS_n : D(w) = \{k, 2k, 3k, \ldots\} \cap [n-1] \},
    \end{equation*}
    とするとき, \begin{equation*}
        \sum_{n \geq 0} f_k(kn) \frac{x^{kn}}{(kn)!}
        = \paren{\sum_{n \geq 0} (-1)^n \frac{x^{kn}}{(kn)!}}^{-1}.
    \end{equation*}
\end{theorem*}
\begin{proof}
    \begin{align*}
        \paren{\sum_{n \geq 0} (-1)^n \frac{x^{kn}}{(kn)!}}^{-1}
        & = \paren{1 - \sum_{n \geq 1} (-1)^{n-1} \frac{x^{kn}}{(kn)!}}^{-1} \\
        & = \sum_{j \geq 0} \paren{\sum_{n \geq 1} (-1)^{n-1} \frac{x^{kn}}{(kn)!}}^j.
    \end{align*}
    右辺の$j$乗部分を展開すると \begin{align*}
            & \sum_{j \geq 0} \paren{\sum_{n \geq 1} (-1)^{n-1} \frac{x^{kn}}{(kn)!}}^j \\
        ={} & \sum_{j \geq 0} \sum_{N \geq 0}
                \sum_{\substack{a_1 + \cdots + a_j = N \\ a_i \geq 1}}
                \binom{kN}{ka_1, \ldots, ka_j} (-1)^{N-j} \frac{x^{kN}}{(kN)!}.
    \end{align*}
    ここで \begin{align*}
        S & = \{k, 2k, \ldots, (N-1)k\}, \\
        T & = \{ka_1, k(a_1+a_2), \ldots, k(a_1+\cdots+a_{j-1})\},
    \end{align*}
    とおくと$\binom{kN}{ka_1, \ldots, ka_j} = \alpha_{kN}(T)$ ($=\#\{w \in \fS_{kN} : D(w) \subseteq T\}$) である.$\#S = N-1$,$\#T = j-1$に注意すると, \begin{align*}
            & \sum_{j \geq 0} \sum_{N \geq 0}
                \sum_{\substack{a_1 + \cdots + a_j = N \\ a_i \geq 1}}
                \binom{kN}{ka_1, \ldots, ka_j} (-1)^{N-j} \frac{x^{kN}}{(kN)!} \\
        ={} & \sum_{N \geq 0} \sum_{T \subseteq S}
                (-1)^{\#S-\#T} \alpha_{kN}(T) \frac{x^{kN}}{(kN)!} \\
        ={} & \sum_{N \geq 0} \beta_{kN}(S) \frac{x^{kN}}{(kN)!},
    \end{align*}
    ただし$\beta_{kN}(S) = \#\{w \in \fS_{kN} : D(w) = S\}$.
\end{proof}

これはさらに一般化できる.演習問題146: \begin{equation*}
    \sum_{n \geq 0} f_k(kn+i) \frac{x^{kn+i}}{(kn+i)!} = \frac{\sum_{n \geq 0} (-1)^n \frac{x^{nk+i}}{(kn+i)!}}{\sum_{n \geq 0} (-1)^n \frac{x^{kn}}{(kn)!}} \quad (1 \leq i \leq k).
\end{equation*}
$i=k$として両辺に$1$を足すと先ほどの等式が,$k=2$,$i=1$とすれば$\sum_{n \geq 0} E_{2n+1} \frac{x^{2n+1}}{(2n+1)!} = \sin x / \cos x = \tan x$が得られる.

\subsubsection{Flip Equivalence of Increasing Binary Trees}

$w$のincreasing binary tree $T(w)$について,頂点$v$の左右の子を入れ替える操作を\textbf{flip}という.$n$頂点のincreasing binary tree $T$に何度かflipを行って$T'$にできるとき,$T$と$T'$が\textbf{同値}であるとする.

% $T$の葉の数を$\epsilon(T)$とするとき,$T$と同値なincreasing binary treeの個数は$2^{n-\epsilon(T)}$.

これらの同値類の個数を$f(n)$とする.すなわち$f(n)$は子同士の順番の違いを無視したincreasing binary treeの個数.このような木は$(1,2)$-木と呼ばれる.

\begin{proposition}
    \begin{equation*}
        f(n) = E_n.
    \end{equation*}
\end{proposition}
\begin{proof}
    漸化式を考える.根が子を$1$個持つものと$2$個持つもので分けて数えると, \begin{equation*}
        f(n+1) = f(n) + \frac{1}{2} \sum_{i=1}^{n-1} \binom{n}{i} f(i) f(n-i).
    \end{equation*}
    $y = \sum_{n \geq 1} f(n) x^n/n!$とすると \begin{equation*}
        y' = 1 + y + \frac{1}{2} y^2, \quad y(0) = 0.
    \end{equation*}
    $2 (y+1)' = 1 + (y+1)^2$より,唯一解は$y=\sec x + \tan x - 1$.
\end{proof}

\end{document}
